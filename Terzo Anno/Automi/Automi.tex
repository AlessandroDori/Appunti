\documentclass{article}
\usepackage[a4paper, margin=1in]{geometry}
\usepackage{amsmath}
\usepackage{amsthm}
\usepackage{amssymb}
\usepackage{tcolorbox}
\usepackage{amsthm}
\usepackage{graphicx}
\usepackage[hidelinks]{hyperref}
\usepackage{cleveref}
\usepackage{float}
\usepackage{subcaption}
\usepackage{tikz}
\usepackage{tocloft}
\usepackage[italian]{babel}
\usepackage{makecell}
\setlength{\parindent}{0pt}
\setlength{\parskip}{6pt}

\setcounter{tocdepth}{4} % Mostra i paragrafi nell'indice

\usetikzlibrary{automata, positioning, arrows}
\definecolor{myred}{rgb}{1, 0, 0}
\definecolor{myorange}{rgb}{0.72, 0.51, 0.04}

\title{Automi}
\author{Alessandro Dori}
\date{\today}

\begin{document}

\begin{titlepage}
    \centering
    \vspace*{1in}
    
    \includegraphics[width=0.4\textwidth]{Immagini/logo-sapienza.png}\par\vspace{1cm}
    
    {\scshape\LARGE Università degli Studi di Roma "La Sapienza" \par}
    \vspace{1.5cm}
    
    {\scshape\Large Riassunti di Automi\par}
    \vspace{1.5cm}
    
    {\Large\itshape Alessandro Dori\par}
    
    \vspace{3cm}

    \begin{tcolorbox}[colback=red!10!white, colframe=red!50!black, title=ATTENZIONE!]
        \textbf{Tutti i teoremi/esempi/dimostrazioni sono richiesti all'orale per un voto 18-24, ad eccezione di quelli segnalati con (24-30) o (30L). Quindi pure se non c'è la parentesi (orale) è comunque richiesto all'orale. STUDIATE!}
    \end{tcolorbox}
    
    Docente:\par
    {\large Prof. Daniele Gorla\par}
    
    \vfill
    
    {\large Anno Accademico 2024/2025\par}
    
\end{titlepage}

\tableofcontents
\newpage


\section{Linguaggi regolari}
\subsection{Automi finiti}

\subsubsection*{Definizione formale di un automa finito}
Un automa finito è una quintupla $M = (Q, \Sigma, \delta, q_0, F)$, dove:
\begin{itemize}
    \item $Q$ è un insieme finito chiamato l'insieme degli stati;
    \item $\Sigma$ è l'insieme finito chiamato l'alfabeto;
    \item $\delta: Q \times \Sigma \rightarrow Q$ è la funzione di transizione;
    \item $q_0 \in Q$ è lo stato iniziale;
    \item $F \subseteq Q$ è l'insieme degli stati accettanti.
\end{itemize}

\begin{figure}[ht]
    \centering
    \includegraphics[width=0.5\textwidth]{Immagini/1.png}
    \caption{Esempio di automa finito}
    \label{fig:automa1}
\end{figure}

Possiamo descrivere $M$ formalmente ponendo:
\begin{itemize}
    \item $Q = \{q_1, q_2, \dots, q_n\}$
    \item $\Sigma = \{a, b, \dots, z\}$
    \item La funzione di transizione $\delta$ è descritta come segue:
        \[
        \delta(q_i, a) = q_j \quad \text{dove } i, j \in Q \text{ e } a \in \Sigma
        \]
    \item $q_0$ è lo stato iniziale
    \item $F$ è l'insieme degli stati accettanti
\end{itemize}

Se $A$ è un insieme di tutte le stringhe che la macchina $M$ accetta, diciamo che $A$ è il \textbf{linguaggio della macchina} $M$ e scriviamo $L(M) = A$. $M$ riconosce $A$.


Nel nostro esempio, sia $$ A = \{ w | w \text{ contiene almeno un 1 e un numero pari di 0 segue l'ultimo 1}\}.$$

\subsubsection{Definizione formale di computazione}
Sia $M = (Q, \Sigma, \delta, q_0, F)$ un automa finito e sia $w = w_1 w_2 \dots w_n$ una stringa, dove ogni $w_i$ è un elemento dell'alfabeto $\Sigma$. Allora $M$ accetta $w$ se esiste una sequenza di stati $r_0, r_1, \dots, r_n$ in $Q$ tale che:

\begin{itemize}
    \item $r_0 = q_0$ (la macchina inizia nello stato iniziale),
    \item $r_{i+1} = \delta(r_i, w_{i+1})$ per ogni $i = 0, \dots, n-1$ (la macchina passa da stato a stato in base alla funzione di transizione),
    \item $r_n \in F$ (la macchina accetta il suo input se termina la lettura in uno stato accettante).
\end{itemize}

Diciamo che $M$ riconosce il linguaggio $A$ se $L(M) = A$. Un linguaggio è chiamato un linguaggio regolare se un automa finito lo riconosce.

\subsubsection{Le operazioni regolari}

Definiamo tre operazioni sui linguaggi, chiamate operazioni regolari, e le usiamo per studiare le proprietà dei linguaggi regolari.

\begin{tcolorbox}[colback=yellow!10!white, colframe=yellow!50!black, title=Operazioni Regolari]
\subsubsection{Definizione 1.23}
Siano $A$ e $B$ linguaggi. Definiamo le operazioni regolari di unione, concatenazione e star come segue:
\begin{itemize}
    \item Unione: $A \cup B = \{x \mid x \in A \text{ oppure } x \in B\}$
    \item Concatenazione: $A \cdot B = \{xy \mid x \in A \text{ e } y \in B\}$
    \item Star: $A^* = \{x_1 x_2 \dots x_k \mid k \geq 0, x_i \in A\}$
\end{itemize}

\begin{itemize}
    \item Unione: prende tutte le stringhe sia in $A$ che in $B$ e le raggruppa insieme in un linguaggio.
    \item Concatenazione: antepone una stringa di $A$ ad una stringa di $B$ in tutti i modi possibili per ottenere le stringhe nel nuovo linguaggio.
    \item L'operazione di star è un'operazione unaria. Essa opera concatenando un numero qualsiasi di stringhe in $A$ insieme per ottenere una stringa nel nuovo linguaggio.
\end{itemize}
\end{tcolorbox}

\begin{tcolorbox}[colback=red!10!white, colframe=red!50!black, title=Importante]
La stringa vuota $\epsilon$ è sempre un elemento di $A^*$, indipendentemente da chi sia $A$.
\end{tcolorbox}

Una classe di oggetti è chiusa rispetto ad un'operazione se l'applicazione di questa operazione a elementi della classe restituisce un oggetto ancora nella classe.

\subsubsection{Teorema 1.25}
La classe dei linguaggi regolari è chiusa rispetto all'operazione di unione. In altre parole, se $A$ e $B$ sono linguaggi regolari, lo è anche $A \cup B$.
Costruiamo $M$ da $M_{1}$ e $M_{2}$, dove:
\begin{itemize}
    \item $M_1$ riconosce $A$
    \item $M_2$ riconosce $B$
    \item $M$ riconosce $A \cup B$
    \item Gli stati accettanti in $M$ sono quelle coppie tali che $M_{1}\circ M_{2}$ è in uno stato accettante.
\end{itemize}


\subsubsection{Teorema 1.26}
La classe dei linguaggi regolari è chiusa rispetto all'operazione di concatenazione. Se $A$ e $B$ sono linguaggi regolari, allora lo è anche $A \cdot B$.

\begin{itemize}
    \item $M_1$ riconosce $A$
    \item $M_2$ riconosce $B$
    \item $M$ deve riconoscere $M_1 \circ M_2$, ma $M$ non sa dove dividere il suo input.
\end{itemize}

Per risolvere questo problema introduciamo una nuova tecnica chiamata \textbf{non determinismo}.

\subsection{Non determinismo}

In una macchina non deterministica, possono esistere diverse scelte per lo stato successivo in ogni punto.
Il non determinismo è una generalizzazione del determinismo, quindi ogni automa finito deterministico è autonomamente anche non deterministico.

Esempio:

\begin{center}
    \includegraphics[width=0.5\textwidth]{Immagini/3.png}
\end{center}

\begin{tcolorbox}[title=Nota]
    DFA = Automa finito deterministico\\
    NFA = Automa finito non deterministico
\end{tcolorbox}

Ogni stato di un \textbf{DFA} ha sempre esattamente un arco di transizione uscente per ogni simbolo dell'alfabeto.
In un \textbf{NFA} uno stato può avere zero, uno, o più archi uscenti per ogni simbolo dell'alfabeto.
In un DFA le etichette sugli archi appartengono all'alfabeto, in un NFA troviamo anche $\varepsilon$; zero, uno o più archi possono uscire da ciascuno stato con l'etichetta $\varepsilon$.
In un NFA, nel caso in cui abbiamo due strade diverse con lo stesso simbolo, la "computazione" si divide.
Se una di queste strade termina in uno stato non accettante, essa "muore"; se una qualsiasi di queste strade (copie) accetta, allora l'NFA accetta la stringa di input.
Se incontriamo $\varepsilon$ senza leggere alcun input, la macchina si divide in più copie multiple.

\begin{tcolorbox}[title=Importante]
    Ogni NFA può essere trasformato in un DFA equivalente, e costruire un NFA a volte è più semplice che costruire direttamente un DFA.
\end{tcolorbox}

\begin{center}
    \includegraphics[width=0.4\textwidth]{Immagini/4.png}
    \includegraphics[width=0.4\textwidth]{Immagini/5.png}
\end{center}

\subsubsection{Definizione formale di automa finito non deterministico}

Per un qualsiasi insieme $Q$ denotiamo con $P(Q)$ la collezione di tutti i sottoinsiemi di $Q$.
    \begin{itemize}
        \item $P(Q)$ = \textcolor{myred}{insieme potenza} di $Q$.
        \item Per ogni alfabeto $\Sigma$, scriviamo $\Sigma_\varepsilon$ per denotare $\Sigma \cup \{\varepsilon\}$.
    \end{itemize}

Ora possiamo descrivere formalmente il tipo di funzione di transizione in un NFA come:
\[
\delta:Q\times\Sigma_\varepsilon\rightarrow P(Q)
\]
\vspace{3em}

\paragraph{Definizione 1.37}
\text{  }
\begin{tcolorbox}[colback=blue!10!white, colframe=blue!50!black, title=Definizione 1.37]
Un automa finito non deterministico è una quintupla $(Q,\Sigma,\delta,q_0,F)$, dove:
\begin{enumerate}
    \item $Q$ è un insieme finito di stati.
    \item $\Sigma$ è un alfabeto finito.
    \item $\delta:Q\times\Sigma_\varepsilon\rightarrow P(Q)$ è la funzione di transizione.
    \item $q_{0}\in Q$ è lo stato iniziale.
    \item $F \subseteq Q$ è l'insieme degli stati di accettazione.
\end{enumerate}
\end{tcolorbox}

\subsubsection{Equivalenza tra gli NFA e i DFA}
\paragraph{Teorema 1.39(orale)}
\label{teorema-1.39}
\text{  }
\begin{tcolorbox}[colback=green!10!white, colframe=green!50!black, title=Teorema 1.39 (orale)]
Ogni automa finito non deterministico può essere convertito in un automa finito deterministico equivalente.
\end{tcolorbox}

\textbf{IDEA}: Se un linguaggio è riconosciuto da un NFA, dobbiamo dimostrare l'esistenza di un DFA che lo riconosce. L'idea è di trasformare l'NFA in un DFA equivalente che simula l'NFA. Se $k$ è il numero degli stati dell'NFA, il DFA avrà $2^k$ stati, corrispondenti ai sottoinsiemi degli stati di $Q$. Ora dobbiamo determinare lo stato iniziale, gli stati accettanti del DFA e la sua funzione di transizione.
\vspace{1em}

\textbf{DIMOSTRAZIONE}:
    Sia $N = (Q, \Sigma, \delta, q_0, F)$ l'NFA che riconosce un linguaggio $A$. Costruiamo un DFA $M = (Q', \Sigma, \delta', q_0', F')$ che riconosce $A$. Prima di tutto consideriamo il caso più semplice in cui $N$ non ha $\varepsilon$-archi. In seguito considereremo gli $\varepsilon$-archi.
    
    \begin{enumerate}
        \item $Q' = P(Q)$.
            \begin{itemize}
                \item Ogni stato di $M$ è un insieme di stati di $N$. Ricorda che $P(Q)$ è l'insieme dei sottoinsiemi di $Q$.
            \end{itemize}
        
        \item Per $R \in Q'$ e $a \in \Sigma$, sia 
        \[
            \delta'(R, a) = \{q \in Q \mid q \in \delta(r, a) \text{ per qualche } r \in R\}.
        \]
            \begin{itemize}
                \item Se $R$ è uno stato di $M$, esso è anche un insieme di stati di $N$. Quando $M$ legge un simbolo $a$ nello stato $R$, mostra dove $a$ porta ogni stato in $R$. Poiché da ogni stato si può andare in un insieme di stati, prendiamo l'unione di tutti questi insiemi. Un altro modo per scrivere questa espressione è 
                \[
                    \delta'(R, a) = \bigcup_{r \in R} \delta(r, a),
                \]
                ovvero l'unione di tutti gli insiemi $\delta(r, a)$ per ogni possibile $r \in R$.
            \end{itemize}
        
        \item $q_0' = \{q_0\}$.
            \begin{itemize}
                \item $M$ inizia nello stato corrispondente alla collezione che contiene solo lo stato iniziale di $N$.
            \end{itemize}
        
        \item $F' = \{R \in Q' \mid R \text{ contiene uno stato accettante di } N\}$.
            \begin{itemize}
                \item La macchina $M$ accetta se uno dei possibili stati in cui $N$ potrebbe essere a quel punto è uno stato accettante.
            \end{itemize}
    \end{enumerate}
    
    Ora dobbiamo considerare gli $\varepsilon$-archi. Introduciamo qualche ulteriore notazione. Per ogni stato $R$ di $M$, definiamo $E(R)$ come la collezione di stati che possono essere raggiunti dagli elementi di $R$ proseguendo solo con $\varepsilon$-archi, includendo gli stessi elementi di $R$. Formalmente, per $R \subseteq Q$ sia
    \[
        E(R) = \{q \mid q \text{ può essere raggiunto da } R \text{ attraverso 0 o più } \varepsilon\text{-archi}\}.
    \]
    Poi modifichiamo la funzione di transizione di $M$ ponendo dita supplementari su tutti gli stati che possono essere raggiunti proseguendo attraverso $\varepsilon$-archi dopo ogni passo. Sostituendo $\delta(r, a)$ con $E(\delta(r, a))$ realizziamo questo effetto. Quindi
    \[
        \delta'(R, a) = \{q \in Q \mid q \in E(\delta(r, a)) \text{ per qualche } r \in R\}.
    \]
    Inoltre, dobbiamo modificare lo stato iniziale di $M$ per muovere inizialmente la dita su tutti i possibili stati che possono essere raggiunti dallo stato iniziale di $N$ attraverso gli $\varepsilon$-archi. Cambiare $q_0'$ in $E(\{q_0\})$ realizza questo risultato. Ora abbiamo completato la costruzione del DFA $M$ che simula l'NFA $N$.
    
    Ovviamente la costruzione di $M$ funziona correttamente. A ogni passo nella computazione di $M$ su un input, chiaramente entra in uno stato che corrisponde al sottoinsieme di stati in cui $N$ potrebbe essere a quel punto. Quindi la nostra prova è completa.

\paragraph{Corollario 1.40}
\text{  }
\begin{tcolorbox}[colback=purple!10!white, colframe=purple!50!black, title=Corollario 1.40]\label{corollario-1.40}
Un linguaggio è regolare se e solo se qualche automa finito non deterministico lo riconosce.
\end{tcolorbox}

\begin{figure}[H]
    \centering
    \includegraphics[width=0.4\textwidth]{Immagini/6.png}
    \includegraphics[width=0.4\textwidth]{Immagini/7.png}
    \caption{Conversione da NFA a DFA}
    \label{fig:nfa_example1}
\end{figure}

\paragraph{Chiusura rispetto alle operazioni regolari}
\vspace{1em}
\text{}
\newline
Il nostro scopo è provare che l'unione, la concatenazione e lo star di linguaggi regolari sono ancora regolari. L'uso del non determinismo rende queste prove molto più semplici.
Innanzitutto consideriamo la chiusura rispetto all'unione.

\paragraph{Teorema 1.45(orale)}
\label{teorema-1.45}
\vspace{1em}
\text{}
\newline
\begin{tcolorbox}[colback=orange!10!white, colframe=orange!50!black, title=Teorema 1.45 (orale)]
La classe dei linguaggi regolari è chiusa rispetto all'operazione di unione.
\end{tcolorbox}


\textbf{IDEA:}
Abbiamo i linguaggi regolari $A_1$ e $A_2$ e vogliamo provare che $A_1 \cup A_2$ è regolare. L'idea è prendere due NFA, $N_1$ e $N_2$, che riconoscono rispettivamente $A_1$ e $A_2$, e comporli in un nuovo NFA, $N$.

La macchina $N$ deve accettare il suo input se $N_1$ o $N_2$ accettano questo input. La nuova macchina ha un nuovo stato iniziale che si dirama negli stati iniziali delle vecchie macchine tramite $\varepsilon$-archi. In questo modo, se una delle due macchine accetta l'input, anche $N$ lo accetterà.

\begin{figure}[H]
    \centering
    \includegraphics[width=0.5\textwidth]{Immagini/8.png}
    \caption{Esempio di NFA che riconosce l'unione di due linguaggi regolari}
    \label{fig:nfa_union1}
\end{figure}

\textbf{DIMOSTRAZIONE:}
Sia $N_1 = (Q_1, \Sigma, \delta_1, q_1, F_1)$ che riconosce $A_1$ ed
$N_2 = (Q_2, \Sigma, \delta_2, q_2, F_2)$ che riconosce $A_2$.
Costruiamo $N = (Q, \Sigma, \delta, q_0, F)$ per riconoscere $A_1 \cup A_2$.
\begin{enumerate}
    \item $Q = \{q_0\} \cup Q_1 \cup Q_2.$
        \begin{itemize}
            \item Gli stati di $N$ sono tutti gli stati di $N_1$ e $N_2$, con l'aggiunta di un nuovo stato iniziale $q_0$.
        \end{itemize}
    \item Lo stato $q_0$ è lo stato iniziale di $N$.
    \item L'insieme degli stati accettanti $F = F_1 \cup F_2$.
        \begin{itemize}
            \item Gli stati accettanti di $N$ sono tutti gli stati accettanti di $N_1$ e $N_2$. In questo modo, $N$ accetta se $N_1$ accetta o $N_2$ accetta.
        \end{itemize}
    \item Definiamo $\delta$ in modo che per ogni $q \in Q$ e per ogni $a \in \Sigma_\varepsilon$,
    \[
    \delta(q, a) =
    \begin{cases}
        \delta_1(q, a) & \text{se } q \in Q_1 \\
        \delta_2(q, a) & \text{se } q \in Q_2 \\
        \{q_1, q_2\} & \text{se } q = q_0 \text{ e } a = \varepsilon \\
        \emptyset & \text{se } q = q_0 \text{ e } a \neq \varepsilon.
    \end{cases}
    \]
\end{enumerate}

\paragraph{Teorema 1.47(orale)}
\label{teorema-1.47}
\vspace{1em}
\text{}
\newline
\begin{tcolorbox}[colback=green!10!white, colframe=green!50!black, title=Teorema 1.47 (orale)]
    La  classe dei linguaggi regolari è chiusa rispetto all'operazione di concatenazione.
\end{tcolorbox}

\textbf{IDEA:}
Abbiamo i linguaggi regolari $A_{1}$ e $A_{2}$ e vogliamo provare che $A_{1}\circ A_{2}$ è regolare.
L'idea è sempre quella di prendere due NFA e combinarli in uno solo, ma questa volta in modo diverso. Poniamo come stato iniziale di $N$ lo stato iniziale di $N_{1}$. Gli stati accettanti di $N_{1}$ hanno ulteriori $\varepsilon$-archi che non deterministicamente permettono di diramarsi in $N_{2}$ ogni volta che $N_{1}$ è in uno stato accettante, indicando che ha trovato un pezzo iniziale dell'input che costituisce una stringa in $A_{1}$. Gli stati accettanti di $N$ sono solo gli stati accettanti di $N_{2}$. Quindi, esso accetta quando l'input può essere diviso in due parti, la prima accettata da $N_{1}$ e la seconda da $N_{2}$.

\begin{figure}[H]
    \centering
    \includegraphics[width=0.5\textwidth]{Immagini/9.png}
    \caption{Esempio di concatenazione di due linguaggi regolari}
    \label{fig:concatenation_example1}
\end{figure}
\vspace{1em}

\textbf{DIMOSTRAZIONE:}
Sia $N_1 = (Q_1, \Sigma, \delta_1, q_1, F_1)$ che riconosce $A_1$ ed
$N_2 = (Q_2, \Sigma, \delta_2, q_2, F_2)$ che riconosce $A_2$.
Costruiamo $N = (Q, \Sigma, \delta, q_0, F)$ per riconoscere $A_1 \cdot A_2$.
\begin{enumerate}
    \item $Q = Q_1 \cup Q_2.$
        \begin{itemize}
            \item Gli stati di $N$ sono tutti gli stati di $N_1$ e $N_2$.
        \end{itemize}
    \item Lo stato $q_1$ è uguale allo stato iniziale di $N_1$.
    \item $F = F_2.$
        \begin{itemize}
            \item Gli stati accettanti di $N$ sono gli stati accettanti di $N_2$.
        \end{itemize}
    \item Definiamo $\delta$ in modo che per ogni $q \in Q$ e ogni $a \in \Sigma_\varepsilon$,
    \[
    \delta(q, a) =
    \begin{cases}
        \delta_1(q, a) & \text{se } q \in Q_1 \text{ e } q \notin F_1 \\
        \delta_1(q, a) & \text{se } q \in F_1 \text{ e } a \neq \varepsilon \\
        \delta_1(q, a) \cup \{q_2\} & \text{se } q \in F_1 \text{ e } a = \varepsilon \\
        \delta_2(q, a) & \text{se } q \in Q_2.
    \end{cases}
    \]
\end{enumerate}

\newpage
\paragraph{Teorema 1.49(orale)}
\text{  }
\begin{tcolorbox}[colback=orange!10!white, colframe=orange!50!black, title=Teorema 1.49 (orale)]
La classe dei linguaggi regolari è chiusa rispetto all'operazione star.
\end{tcolorbox}

\textbf{IDEA:}
Abbiamo un linguaggio regolare $A_{1}$ e vogliamo provare che anche $A^{*}_{1}$ è regolare. Prendiamo un NFA $N_{1}$ per $A_{1}$ e lo modifichiamo per riconoscere $A^{*}_{1}$. L'NFA $N$ risultante accetterà il suo input quando esso può essere diviso in varie parti ed $N_{1}$ accetta ogni parte.
Possiamo costruire $N$ come $N_{1}$ con $\varepsilon$-archi supplementari che dagli stati accettanti ritornano allo stato iniziale. In questo modo, quando l'elaborazione giunge alla fine di una parte che $N_{1}$ accetta, la macchina $N$ ha la scelta di tornare indietro allo stato iniziale per provare a leggere un'altra parte che $N_{1}$ accetta. Inoltre,  dobbiamo modificare $N$ in modo che accetti $\varepsilon$, che è sempre un elemento di $A^{*}_{1}$. L'idea è di aggiungere un nuovo stato iniziale, che è anche uno stato accettante, e che ha un $\varepsilon$-arco entrante nel vecchio stato iniziale.
\vspace{1em}

\begin{figure}[H]
    \centering
    \includegraphics[width=0.5\textwidth]{Immagini/10.png}
    \caption{Esempio di operazione star su un linguaggio regolare}
    \label{fig:star_example1}
\end{figure}

\textbf{DIMOSTRAZIONE:}
Sia $N_{1} = (Q_{1},\Sigma,\delta_{1},q_{1},F_{1})$ che riconosce $A_{1}$.
Costruiamo $N = (Q,\Sigma,\delta,q_{0},F)$ per riconoscere $A^{*}_{1}$.
\begin{enumerate}
    \item $Q = \{q_{0}\}\cup Q_{1}.$
        \begin{itemize}
            \item Gli stati di $N$ sono gli stati di $N_{1}$ più un nuovo stato iniziale.
        \end{itemize}
    \item Lo stato $q_{0}$ è il nuovo stato iniziale.
    \item $F = \{q_{0}\} \cup F_{1}.$
        \begin{itemize}
            \item Gli stati accettanti sono i vecchi stati accettanti più il nuovo stato iniziale.
        \end{itemize}
    \item Definiamo $\delta$ in modo che per ogni $q \in Q$ e ogni $a \in \Sigma_\varepsilon$,
    \[
    \delta(q,a) =
    \begin{cases}
        \delta_{1}(q,a) & \text{se } q \in Q_{1} \text{ e } q \notin F_{1} \\
        \delta_{1}(q,a) & \text{se } q \in F_{1} \text{ e } a \neq \varepsilon \\
        \delta_{1}(q,a) \cup \{q_{1}\} & \text{se } q \in F_{1} \text{ e } a = \varepsilon \\
        \{q_{1}\} & \text{se } q = q_{0} \text{ e } a = \varepsilon \\
        \emptyset & \text{se } q = q_{0} \text{ e } a \neq \varepsilon
    \end{cases}
    \]
\end{enumerate}
\vspace{4em}

\subsection{Espressioni regolari}

In aritmetica, possiamo usare le operazioni $+,\times$ per costruire espressioni come
$$(5+3)\times4.$$
Analogamente, possiamo usare le operazioni regolari per costruire espressioni che descrivono linguaggi, che sono chiamate \textbf{espressioni regolari}.
Un esempio è:
$$(0\cup 1)0^{*}.$$
Il valore dell'espressione aritmetica è il numero 32, mentre il valore di un'espressione regolare è un linguaggio. In questo caso, il valore è il linguaggio che consiste di tutte le stringhe che iniziano con uno 0 o un 1 seguito da un qualsiasi numero di simboli uguali a 0.
Le espressioni regolari hanno un ruolo importante nelle applicazioni dell'informatica. Nelle applicazioni che coinvolgono testo, gli utenti possono voler cercare stringhe che soddisfano alcuni schemi.
\vspace{3em}

\textbf{ESEMPIO 1.51}

Un altro esempio di espressione regolare è $$(0 \cup 1)^{*}.
$$ Essa inizia con il linguaggio $(0 \cup 1)$ a cui applica l'operatore $*$. Il valore di questa espressione è il linguaggio che consiste di tutte le possibili stringhe di simboli 0 e 1. Se $\Sigma = \{0,1\},$ possiamo usare $\Sigma$ come abbreviazione per l'espressione regolare $(0\cup 1)$. Più in generale, se $\Sigma$ è un qualsiasi alfabeto, l'espressione regolare $\Sigma$ descrive il linguaggio che consiste di tutte le stringhe di lunghezza 1 su questo alfabeto e $\Sigma^*$ descrive il linguaggio che consiste di tutte le stringhe su quell'alfabeto. Analogamente, $\Sigma^{*}1$ è il linguaggio che contiene tutte le stringhe che termiinano con 1. Il linguaggio $(0\Sigma^{*}) \cup (\Sigma^{*}1)$ consiste di tutte le stringhe che iniziano con uno 0 o terminano con un 1.

\begin{tcolorbox}[colback=red!10!white, colframe=red!50!black, title=IMPORTANTE!]
    Nelle espressioni regolari, l'operazione star è eseguita per prima,
    seguita dalla concatenazione e infine dall'unione, a meno che delle parentesi
    non cambino l'ordine usuale.
\end{tcolorbox}

\subsubsection{Definizione formale di espressioni regolari}

\begin{tcolorbox}[colback=blue!10!white, colframe=blue!50!black, title=Definizione]
    Diciamo che $R$ è un'\textcolor{myred}{espressione regolare} se $R$ è
    \begin{enumerate}
        \item $a$ per qualche $a$ nell'alfabeto $\Sigma$,
        \item $\varepsilon,$
        \item $\emptyset,$
        \item $(R_{1}\cup R_{2}),$ dove $R_{1}$ ed $R_{2}$ sono espressioni regolari,
        \item $(R_{1}\circ R_{2}),$ dove $R_{1}$ ed $R_{2}$ sono espressioni regolari, o
        \item $(R^*_{1}),$ dove $R_{1}$ è un'espressione regolare.
    \end{enumerate}

    Nei punti 1 e 2, le espressioni regolari $a$ e $\varepsilon$ rappresentano i linguaggi $\{a\}$ e $\{\varepsilon\},$ rispettivamente. Nel punto 3, l'espressione regolare $\emptyset$ rappresenta il linguaggio vuoto. Nei punti 4, 5, e 6, le espressioni rappresentano i linguaggi ottenuti prendendo l'unione o la concatenazione dei linguaggi $R_{1}$ ed $R_{2}$, o lo star del linguaggio $R_1$, rispettivamente.
\end{tcolorbox}

\begin{tcolorbox}[colback=red!10!white, colframe=red!50!black, title=ATTENZIONE!]
    Non confondere le espressioni regolari $\varepsilon \text{ e } \emptyset.$ L'espressione $\varepsilon$ rappresenta il linguaggio che contiene una sola stringa - ossia, la stringa vuota - mentre $\emptyset$ rappresenta il linguaggio che non contiene alcuna stringa.
\end{tcolorbox}

Apparentemente, sembra che stiamo definendo la nozione di espressione regolare in termini di sé stessa. Se fosse vero, avremmo una definizione circolare, che non sarebbe valida. Comunque, $R_{1}$, ed $R_{2}$ sono sempre più piccole di $R$. Quindi in realtà stiamo definendo le espressioni regolari in termini di espressioni regolari più piccole, evitando in tal modo la circolarità. Una definizione di questo tipo è chiamata una \textcolor{myred}{definizione induttiva}.
\vspace{1em}
\text{}
\newline
Le parentesi in un'espressione possono essere omesse. Se lo sono, la valutazione è fatta nell'ordine della precedenza: star, poi concatenazione, poi unione.
Per comodità, usiamo $R^{+}$ come abbreviazione per $RR^{*}$. In altre parole, mentre $R^{*}$ contiene tutte le stringhe che sono concatenazione di 0 o più stringhe di $R$, il linguaggio $R^{+}$ contiene tutte le stringhe che sono concatenazione di 1 o più stringhe di $R$. Quindi $R^{+}\cup \varepsilon = R^{*}$. Inoltre, denotiamo con $R^{k}$ l'abbreviazione della concatenazione di $k$ copie di $R$ insieme.

Quando vogliamo distinguere tra un'espressione regolare $R$ e il linguaggio
che descrive, denotiamo con $L(R)$ il linguaggio di $R$.

\begin{figure}[H]
    \centering
    \includegraphics[width=0.4\textwidth]{Immagini/11.png}
    \caption{Esempio di espressione regolare}
    \label{fig:regular_expression_example1}
\end{figure}
\vspace{1em}
\text{}
\newline
$R \cup \emptyset = R.$
Aggiungere il linguaggio vuoto a un qualsiasi altro linguaggio non lo cambia.
\newline
$R\circ \varepsilon = R$.
Concatenare la stringa vuota a una qualsiasi stringa non la cambia.

Tuttavia, scambiare $\emptyset$ e $\varepsilon$ nelle precedenti identità può far si che le uguaglianze non siano vere.
\begin{itemize}
    \item $R \cup \varepsilon$ può non essere uguale a $R$.
    \item $\quad$ Per esempio, se $R = 0,$ allora $L(R) = \{0\}$, ma $L(R \cup \varepsilon) = \{0, \varepsilon\}$.
    \item $R \circ \emptyset$ può non essere uguale a $R$.
    \item $\quad$ Per esempio, se $R = 0$, allora $L(R) = \{0\}$ ma $L(R \circ \emptyset) = \emptyset$.
\end{itemize}

Gli oggetti elementari in un linguaggio di programmazione, chiamati $token$, come i nomi delle variabili e le costanti, possono essere descritti con espressioni regolari. Per esempio, una costante numerica che può avere una parte decimale e/o un segno può essere descritta come un elemento del linguaggio

$$(+ \cup - \cup \varepsilon)(D^{+}\cup D^{+}.D^{*}\cup D^{*}.D^{+})$$

dove $D = \{0,1,2,3,4,5,6,7,8,9\}$ è l'alfabeto delle cifre decimali. Esempi di stringhe generate sono: $72,\text{ }3.14159,\text{ }+7.,$ e $-.01.$

\subsubsection{Equivalenza con gli automi finiti}

Le espressioni regolari e gli automi finiti sono equivalenti rispetto alla loro potenza descrittiva. Ogni espressione regolare può essere trasformata in un automa finito che riconosce il linguaggio che essa descrive, e viceversa. Ricorda che in linguaggio regolare è un linguaggio che è riconosciuto da qualche automa finito.
\vspace{4em}

\paragraph{Teorema 1.54}
\label{teorema-1.54}
\text{  }
\begin{tcolorbox}[colback=green!10!white, colframe=green!50!black, title=Teorema 1.54]
    Un linguaggio è regolare se e solo se qualche espressione regolare lo descrive.
\end{tcolorbox}

Questo teorema deve essere dimostrato in entrambe le direzioni. Noi enunciamo e proviamo ciascuna direzione in un lemma separato.

\paragraph{Lemma 1.55(orale)}
\label{lemma-1.55}
\text{  }
\begin{tcolorbox}[colback=orange!10!white, colframe=orange!50!black, title=Lemma 1.55 (orale)]
    Se un linguaggio è descritto da un'espressione regolare, allora esso è regolare.
\end{tcolorbox}

\textbf{IDEA:}
Supponiamo di avere un'espressione regolare $R$ che descrive un linguaggio $A$. Mostriamo come trasformare $R$ in un NFA che riconosce $A$.
Per il \hyperref[corollario-1.40]{\textcolor{blue}{Corollario 1.40}}, se un NFA riconosce $A$ allora $A$ è regolare.
\vspace{1em}

\textbf{DIMOSTRAZIONE:}
Trasformiamo $R$ in un NFA $N$. Consideriamo i sei casi nella definizione di espressione regolare.
\begin{enumerate}
    \item $R = a \text{ per qualche } a \in \Sigma.$ Allora $L(R) = \{a\}$ e il seguente NFA riconosce $L(R)$. 
        \begin{figure}[H]
            \centering
            \includegraphics[width=0.3\textwidth]{Immagini/12.png}
            \label{fig:your_image1}
        \end{figure}
        Nota che questa macchina soddisfa la definizione di NFA ma non quella di DFA perché ha qualche stato con nessun arco uscente per ogni possibile simbolo di input. Naturalmente, qui avremmo potuto presentare un DFA equivalente; ma un NFA è tutto quello di cui abbiamo bisogno per ora, ed è più facile da descrivere.
        
        Formalmente, $N = (\{q_{1},q_{2}\},\Sigma,\delta,q_{1},\{q_{2}\})$, dove descriviamo $\delta$ dicendo che $\delta(q_{1},a) = \{q_2\}$ e $\delta(r,b) = \emptyset \text{ per }r\neq q_{1}$ o $b \neq a.$
    \item $R = \varepsilon.$ Allora $L(R) = \{\varepsilon\}$ e il seguente NFA riconosce $L(R)$.
    \begin{figure}[H]
        \centering
        \includegraphics[width=0.3\textwidth]{Immagini/13.png}
        \label{fig:your_image1}
    \end{figure}
    Formalmente, $N = (\{q_{1}\},\Sigma,\delta,q_{1},\{q_{1}\})$ dove $\delta(r,b) = \emptyset$ per ogni $r$ e $b$.
    \item $R = \emptyset.$ Allora $L(R) = \emptyset,$ e il seguente NFA riconosce $L(R)$.
    \begin{figure}[H]
        \centering
        \includegraphics[width=0.3\textwidth]{Immagini/14.png}
        \label{fig:your_image}
    \end{figure}
    Formalmente, $N = (\{q\},\sigma,\delta,q,0)$, dove $\delta(r,b) = \emptyset \text{ per ogni } r \text{ e } b.$
    \item $R = R_{1}\cup R_{2}.$
    \item $R = R_{1}\circ R_{2}.$
    \item $R = R^{*}_{1}.$
    
    Per gli ultimi tre casi, usiamo le costruzioni date nelle prove che la classe dei linguaggi regolari è chiusa rispetto alle operazioni regolari. In altre parole, costruiamo l'NFA per $R$ dagli NFA per $R_{1}$ ed $R_{2}$ (o solo $R_{1}$ nel caso 6) e mediante l'appropriata costruzione della chiusura.
\end{enumerate}

\begin{figure}[H]
    \centering
    \includegraphics[width=0.4\textwidth]{Immagini/15.png}
    \includegraphics[width=0.4\textwidth]{Immagini/16.png}
    \label{fig:your_image}
\end{figure}

\paragraph{Lemma 1.60(orale 24-30)}
\label{lemma-1.60}
\text{  }
\begin{tcolorbox}[title=Lemma 1.60 (orale 24-30)]
    Se un linguaggio è regolare, allora è descritto da un'espressione regolare.
\end{tcolorbox}

\textbf{IDEA.}

Dobbiamo mostrare che se un linguaggio $A$ è regolare, allora un'espressione regolare lo descrive. 
Poiché $A$ è regolare, esso è accettato da un DFA. 
Descriviamo una procedura per trasformare i DFA in espressioni regolari equivalenti. 
Dividiamo questa procedura in due parti, usando un nuovo tipo di automa finito chiamato \textbf{automa finito non deterministico generalizzato}, GNFA.
Prima mostiamo come trasformare un DFA in un GNFA e poi come trasformare un GNFA in un'espressione regolare.

Gli automi finiti non deterministici generalizzati sono semplicemente automi finiti non deterministici nei quali gli archi delle transizioni possono
avere espressioni regolari come etichette, invece che solo elementi dell'alfabeto o $\varepsilon$. 
Il GNFA legge blocchi di simboli dall'input, non necessariamente solo un simbolo alla volta come in un comune NFA. 
Il GNFA si muove lungo un arco di transizione che collega due stati leggendo un blocco di simboli dall'input che formano una stringa descritta dall'espressione regolare su quell'arco. 
Un GNFA è non deterministico e quindi può avere diversi modi di elaborare la stessa stringa di input.
Esso accetta il suo input se la sua elaborazione può far sì che il GNFA sia in uno stato accettante alla fine dell'input. 
La figura seguente presenta un esempio di un GNFA.

\begin{figure}[H]
    \centering
    \includegraphics[width=0.3\textwidth]{Immagini/17.png}
    \caption{Esempio di GNFA}
    \label{fig:your_image}
\end{figure}
Per comodità, richiediamo che i GNFA abbiano sempre una forma speciale
che soddisfi le seguenti condizioni.
\begin{itemize}
    \item Lo stato iniziale ha archi di transizione uscenti verso un qualsiasi altro stato ma nessun arco entrante proveniente da un qualsiasi altro stato.
    \item Esiste un solo stato accettante, ed esso ha archi entranti provenienti da un qualsiasi altro stato ma nessun arco uscente verso un qualsiasi altro stato. Inoltre, lo stato accettante non è uguale allo stato iniziale.
    \item Eccetto che per lo stato iniziale e lo stato accettante, un arco va da ogni stato ad ogni altro stato e anche da ogni stato in se stesso.
\end{itemize}

Possiamo facilmente trasformare un DFA in un GNFA nella forma speciale.
Aggiungiamo semplicemente un nuovo stato iniziale con un $\varepsilon$-arco che entra nel vecchio stato iniziale e un nuovo stato accettante con $\varepsilon$-archi entranti, provenienti dai vecchi stati accettanti. 
Se alcuni archi hanno più etichette (o se ci sono più archi che collegano gli stessi due stati nella stessa direzione), sostituiamo ognuno di essi con un solo arco la cui etichetta è l'unione delle precedenti etichette. 
Infine, aggiungiamo archi con etichetta $\emptyset$ tra stati che non hanno archi. 
Questo ultimo passo non cambierebbe il linguaggio riconosciuto perché una transizione etichettata con $\emptyset$ non può mai essere usata.
Ora mostriamo come trasformare un GNFA in un'espressione regolare.
Supponiamo che il GNFA abbia $k$ stati. Allora, poichè un GNFA deve avere uno stato iniziale e uno accettante ed essi devono essere diversi tra loro, sappiamo che $k \geq 2$. Se $k > 2$, costruiamo un GNFA equivalente con $k-1$ stati. Questo passo può essere ripetuto sul nuovo GNFA fino a quando esso è ridotto a 2 stati. Se $k = 2$ il GNFA ha un solo arco che va dallo stato iniziale allo stato accettante. L'etichetta di questo arco è l'espressione regolare equivalente.
Per esempio, le fasi per trasformare un DFA con 3 stati in un'espressione regolare equivalente sono mostrati nella figura seguente.

\begin{figure}[H]
    \centering
    \includegraphics[width=0.4\textwidth]{Immagini/29.png}
    \caption{Trasformazione di un GNFA in un'espressione regolare}
    \label{fig:your_image}
\end{figure}

Il passo cruciale è costruite un GNFA equivalente con uno stato in meno quando $k \geq 2$.
Lo facciamo scegliendo uno stato, estraendolo dalla macchina, e modificando il resto in modo che sia ancora riconosciuto lo stesso linguaggio.
Ogni stato può essere scelto, purchè non sia lo stato iniziale o lo stato accettante. Siamo certi che un tale stato esiste perchè $k > 2$.
Chiamiamo $q_{rip}$ lo stato rimosso.

Dopo aver rimosso $q_{rip}$ modifichiamo la macchina cambiando le espressioni regolari che etichettano ciascuno dei restanti archi.
Le nuove etichette controbilanciano l'assenza di $q_{rip}$ reintegrando le computazioni perse.
La nuova etichetta che va da uno stato $q_i$ a uno stato $q_j$ è un'espressione regolare che descrive tutte le stringhe che porterebbero la macchina da $q_i$ a $q_j$ o direttamente o tramite $q_{rip}$.
Illustriamo questa strategia nella seguente figura.

\begin{figure}[H]
    \centering
    \includegraphics[width=0.5\textwidth]{Immagini/44.png}
    \caption{Esempio di trasformazione di un GNFA}
    \label{fig:gnfa_example}
\end{figure}

Nella vecchia macchina, se

\begin{enumerate}
    \item $q_{i}$ va a $q_{rip}$ con un arco etichettato $R_1$,
    \item $q_{rip}$ va in se stesso con un arco etichettato $R_2$,
    \item $q_{rip}$ va a $q_j$ con un arco etichettato $R_3$,
    \item $q_i$ va a $q_j$ con un arco etichettato $R_4$,
\end{enumerate}
allora nella nuova macchina, l'arco da $q_i$ a $q_j$ riceve l'etichetta
$$(R_1)(R_2)^* (R_3)\cup (R_4).$$
Facciamo questa modifica per ogni arco che va da un qualsiasi stato $q_i$ a un qualsiasi stato $q_j$, includendo il caso un cui $q_i = q_j$.
La nuova macchina riconosce il linguaggio di partenza.
\vspace{1em}
\text{}
\newline
\textbf{DIMOSTRAZIONE.}
Formalizziamo questa idea.
In primo luogo, per rendere più facile la prova, definiamo formalmente il nuovo tipo di automa introdotto.
Un GNFA è simile a un automa finito non deterministico tranne che per la funzione di transizione, che ha la forma
$$ \delta : (Q - \{q_{accept} \}) \times (Q- \{q_{start} \}) \rightarrow R.$$

Il simbolo $R$ è la collezione di tutte le espressioni regolari sull'alfabeto $\Sigma$, e $q_{start}$ e $q_{accept}$ sono gli stati iniziale e accettante.
Se $\delta(q_i,q_j) = R$, l'arco dallo stato $q_i$ allo stato $q_j$ ha come etichetta l'espressione regolare $R$.
Il dominio della funzione di transizione è $(Q - \{q_{accept} \}) \times (Q- \{q_{start} \})$ perchè un arco collega ogni stato a un qualsiasi altro stato, tranne che nessun arco è uscente da $q_{accept}$ o è entrante in $q_{start}$.

\paragraph{Definizione 1.64(orale 24-30)}
\text{ }

\begin{tcolorbox}[colback=yellow!10!white, colframe=yellow!50!black, title=Definizione 1.64 (orale 24-30)]
    Un automa finito non deterministico generalizzato è una quintupla, $(Q,\Sigma,\delta,q_{start},q_{accept})$, dove:
    \begin{enumerate}
        \item $Q$ è l'insieme finito degli stati,
        \item $\Sigma$ è l'alfabeto di input,
        \item $\delta: (Q - \{q_{accept}\}) \times (Q - \{q_{start}\}) \rightarrow R$ è la funzione di transizione,
        \item $q_{start}$ è lo stato iniziale e
        \item $q_{accept}$ è lo stato accettante.
    \end{enumerate}
\end{tcolorbox}

Un GNFA accetta una stringa $w$ in $\Sigma^{*}$ se $w = w_{1}w_{2}...w_{k},$ dove ogni $w_{i}$ è in $\Sigma^{*}$ ed esiste una sequenza di stati $q_{0},q_{1},...,q_{k}$ tale che:
\begin{enumerate}
    \item $q_{0}= q_{start}$ è lo stato iniziale,
    \item $q_{k}= q_{accept}$ è lo stato accettante, e
    \item per ogni $i$, risulta $w_{i}\in L(R_{i}),$ dove $R_{i}= \delta(q_{i-1},q_{i});$ in altre parole, $R_{i}$ è l'espressione sull'arco da $q_{i-1}$ a $q_{i}$.
\end{enumerate}

Tornando alla prova del \hyperref[lemma-1.60]{\textcolor{blue}{Lemma 1.60}}, sia $M$ il DFA per il linguaggio A.
Allora trasformiamo $M$ in un GNFA $G$ aggiungendo un nuovo stato iniziale e un nuovo stato accettante e i necessari archi di transizione supplementari.
Usiamo la procedura $CONVERT(G)$, che prende un GNFA e restituisce un'espressione regolare equivalente. 
Questa procedura usa la \textbf{ricorsione}, che significa che essa chiama se stessa. 
Evitiamo un ciclo infinito perchè la procedura chiama se stessa solo per elaborare un GNFA che ha uno stato in meno.
Il caso un cui il GNFA ha due stati è trattato senza ricorsione.
\vspace{1em}

$CONVERT(G):$
\begin{enumerate}
    \item Sia $k$ il numero di stati di $G$.
    \item Se $k = 2$, allora $G$ deve consistere di uno stato iniziale, uno stato accettante, e un signolo arco che li collega ed è etichettato con un'espressione regolare $R$. Restituisce l'espressione $R$.
    \item Se $k > 2$, scegliamo un qualsiasi stato $q_{rip} \in Q$ diverso da $q_{start}$ e $q_{accept}$ e sia $G^{'}$ il GNFA $(Q^{'},\Sigma,\delta^{'},q_{start},q_{accept})$, dove $$Q^{'} = Q - \{q_{rip}\},$$ e per ogni $q_i \in Q^{'} - \{q_{accept}\}$ e ogni $q_j \in Q^{'} - \{q_{start}\}$, poniamo $$ \delta^{'}(q_i,q_j) = (R_1)(R_2)^{*}(R_3) \cup (R_4),$$ dove $R_1 = \delta(q_i,q_{rip})$, $R_2 = \delta(q_{rip},q_{rip},)$ $R_3 = \delta(q_{rip},q_j)$ ed $R_4 = \delta (q_i,q_j)$.
    \item Calcola $CONVERT(G^{'})$ e restituisce questo valore. 
\end{enumerate}

Di seguito proviamo che $CONVERT$ restituisce un valore corretto.

\begin{figure}[H]
    \centering
    \includegraphics[width=0.5\textwidth]{Immagini/45.png}
    \label{fig:nuova_immagine}
\end{figure}

\begin{tcolorbox}[colback=yellow!10!white, colframe=yellow!50!black, title=Esempio 1.66]
    In questo esempio, usiamo il precedente algoritmo per trasformare un DFA in un'espressione regolare. Iniziamo con il DFA con due stati nella Figura 1.67(a).
Nella Figura 1.67(b), creiamo un GNFA con quattro stati aggiungendo un nuovo stato iniziale e un nuovo stato accettante, chiamati $s$ e $a$ invece di $q_{start}$ e $q_{accept}$ in modo da poterli disegnare in modo conveniente. Per evitare di ingombrare la figura, non disegniamo gli archi etichettati $0$, sebbene essi vi siano. Nota che sostituiamo le etichette $a,b$ sul ciclo nello stato 2 del DFA con l'etichetta $a\cup b$ nel corrispondente punto del GNFA. Facciamo in questo modo perché le etichette del DFA rappresentano due transizioni, una per $a$ e l'altra per $b$, mentre il GNFA può avere solo una singola transizione che va da 2 a sé stesso.
Nella Figura 1.67(c), eliminiamo lo stato 2 e aggiorniamo le etichette degli archi restanti. In questo caso, la sola etichetta che cambia è quella da $1$ ad $a$. Nella parte (b) era $0$, ma nella parte (c) essa è $b(a\cup b)^{*}$. Otteniamo questo risultato seguendo il passo 3 della procedura $CONVERT$. Lo stato $q_{i}$ è lo stato 1, lo stato $q_{j}$ è $a$, e $q_{rip}$ è 2, quindi $R_{1} = b, R_{2} = a\cup b, R_{3} = \varepsilon \text{ ed } R_{4} = 0$. Pertanto, la nuova etichetta sull'arco da $1$ ad $a$ è $(b)(a\cup b)^{*}(\varepsilon) \cup 0$. Semplifichiamo quest'espressione regolare in $b(a \cup b)^{*}$. Nella Figura 1.67(d), eliminiamo lo stato 1 dalla parte (c) e seguiamo la stessa procedura. Poiché restano solo lo stato iniziale e lo stato accettante, l'etichetta sull'arco che li collega è l'espressione regolare equivalente al DFA iniziale.
\end{tcolorbox}

\begin{figure}[H]
    \centering
    \includegraphics[width=0.4\textwidth]{Immagini/18.png}
    \includegraphics[width=0.4\textwidth]{Immagini/19.png}
    \caption{Trasformazione di un DFA in un'espressione regolare}
    \label{fig:your_image}
\end{figure}

\subsubsection{Grammatiche Regolari}
Una grammatica $G = (V,\Sigma,R,S)$ è detta
\begin{itemize}
    \item \textbf{\textit{Lineare sinistra}} se, per ogni ($\alpha,\beta$) $\in R$, si ha che $\alpha \in V$ e $\beta \in (\Sigma^* \cup V\Sigma^*)$. 
    \item \textbf{\textit{Lineare destra}} se, per ogni ($\alpha,\beta$) $\in R$, si ha che $\alpha \in V$ e $\beta \in (\Sigma^* \cup \Sigma^* V)$.
    \item \textbf{\textit{Regolare}} se è lineare sinistra o lineare destra.
\end{itemize}

\textbf{ESEMPIO\_1:} La seguente grammatica per numeri naturali è regolare (più precisamente, lineare destra):
\begin{align*}
    &NUMB ::= 0 \mid 1 DIGITSEQ \mid ... \mid 9 DIGITSEQ \\
    &DIGITSEQ ::= \varepsilon \mid 0 DIGITSEQ \mid ... \mid 9 DIGITSEQ
\end{align*}

\textbf{ESEMPIO\_2:} Il linguaggio (regolare) $0(10)^*$ è generato dalla seguente grammatica lineare destra:
\begin{align*}
    &S ::= 0A \\
    &A ::= \varepsilon \mid 10A
\end{align*}
o, equivalentemente dalla grammatica lineare sinistra:
\begin{align*}
    &S ::= 0 \mid S10
\end{align*}
Con la prima grammatica, possiamo derivare $0101010$ come segue:
\begin{align*}
    &S \Rightarrow 0A \Rightarrow 010A \Rightarrow 01010A \Rightarrow 0101010A \Rightarrow 0101010
\end{align*}

Con la seconda grammatica, possiamo derivare $0101010$ come segue:
\begin{align*}
    &S \Rightarrow S10 \Rightarrow S1010 \Rightarrow S101010 \Rightarrow 0101010
\end{align*}

\subsection{Linguaggi regolari vs Grammatiche lineari destre}

\paragraph{Teorema 9.1(orale)}
\label{teorema-9.1}
\text{}
\newline
Se $L$ è regolare, allora è generato da una grammatica lineare destra.

\begin{tcolorbox}[colback=yellow!10!white, colframe=yellow!50!black, title=Dimostrazione]
    Sia $L = L(M)$ per un DFA $M = (Q,\Sigma,\delta,q_0,F)$.

    Supponiamo per ora che $q_0$ non sia finale.

    Allora, $L = L(G)$, dove $G$ è la grammatica lineare destra ($Q,\Sigma,R,q_0$), dove $R$ include
    \begin{itemize}
        \item $q ::= aq'$, se $\delta(q,a) = q'$,
        \item $q ::= a$, se $\delta(q,a) \in F$,
    \end{itemize}
    Per induzione su $|w|$, si può provare che $\delta(q,w) = q'$ se e solo se $q \Rightarrow^* wq'$.

    Se $q_0$ è finale, consideriamo la grammatica $G' = (Q \cup \{S\}, \Sigma, R', S)$ con $S \notin Q$ e $R' = R \cup \{ S ::= \varepsilon | q_0 \}$ dove $R$ è costruito sopra.
\end{tcolorbox}

\textbf{ESEMPIO:} Si consideri il seguente DFA per il linguaggio $0(10)^*$:

\begin{figure}[H]
    \centering
    \includegraphics[width=0.3\textwidth]{Immagini/114.png}
    \label{fig:dfa_example} 
\end{figure}

La grammatica lineare destra per questo DFA è
\begin{align*}
    &A ::= 0B \mid 1D \mid 0 \\
    &B ::= 0D \mid 1C \\
    &C ::= 0B \mid 1D \mid 0 \\
    &D ::= 0D \mid 1D \\
\end{align*}
Si noti che la variabile $D$ non è realmente necessaria (nell'automa abbiamo bisogno dello stato <<pozzo>> $D$ perchè l'automa è deterministico, e quindi dobbiamo gestire tutti i possibili input, per es. le stringhe che iniziano con un 1)

Quindi, una grammatica più compatta (ed equivalente alla precedente) è 
\begin{align*}
    &A ::= 0B \mid 0 \\
    &B ::= 1C \\
    &C ::= 0B \mid 0 \\
\end{align*}

\paragraph{Corollario1(orale)}
\label{corollario-9.1}
\text{}
\newline
\begin{tcolorbox}[colback=blue!10!white, colframe=blue!50!black, title=Dimostrazione]
    Siccome $L$ è regolare, anche $L^R$ è regolare (per una proprietà di chiusura dei linguaggi regolari).

    Sia $L^R = L(M)$ per un DFA $M$.
    
    Per il \hyperref[teorema-9.1]{Teorema 9.1}, esiste una grammatica lineare destra $G$ t.c. $L^R = L(G)$.
    
    Ora si considerei $G^R$, la grammatica otenuta da $G$ invertendo le parti destre di tutte le sue produzioni.

    $G^R$ è una grammatica lineare sinistra e $L(G^R) = (L(G))^R = (L^R)^R = L$.
\end{tcolorbox}

\paragraph{Teorema 9.2(orale)}
\label{teorema-9.2}
\text{}
\newline
Se $L$ è generato da una grammatica lineare destra, allora $L$ è regolare.

\begin{tcolorbox}[colback=yellow!10!white, colframe=yellow!50!black, title=Dimostrazione]
    Sia $L = L(G)$, per una qualche grammatica lineare destra $G = (V,\Sigma,R,S)$.

    Costruiamo un NFA con $\varepsilon$-mosse $M = (Q,\Sigma,\delta,q_0,F)$ che simuli derivazioni in $G$:
    \begin{itemize}
        \item $Q$ è formato da tutti gli [$\alpha$] tali che $\alpha = S$ o $\alpha$ è un suffisso (non necessariamente proprio) di una qualche parte destra di una regola di $R$
        \item $q_0 = [S]$
        \item $F = \{[\varepsilon]\}$
        \item $\delta$ è definito come segue:
        \begin{itemize}
            \item Se $A \in V$, allora $\delta([A],\varepsilon) = \{[\alpha] \mid A ::= \alpha \in R\}$.
            \item Se $a \in \Sigma$ ed $\alpha \in (\Sigma^* \cup \Sigma^*V)$, allora $\delta([a\alpha],a) = \{[\alpha]\}$.
        \end{itemize}
    \end{itemize}
    Per induzione su $|w|$, si può provare che $[\alpha] \in \delta([S],w)$ se e solo se $S \Rightarrow^* w\alpha$.

    Siccome $[\varepsilon]$ è l'unico stato finale, $M$ accetta $w$ (cioè, $[\varepsilon] \in \delta([S],w)$) sse $S \Rightarrow^* w$.
\end{tcolorbox}

\textbf{ESEMPIO:} Consideriamo la grammatica lineare destra:
\begin{align*}
    &S ::= 0A \\
    &A ::= \varepsilon \mid 10A
\end{align*}
L'automa ad essa associato è:

\begin{figure}[H]
    \centering
    \includegraphics[width=0.3\textwidth]{Immagini/115.png}
    \label{fig:dfa_example}
\end{figure}
E' facile convincersi che il linguaggio accettato è $0(10)^*$.

\newpage
\subsection{Grammatiche lineari sinistre vs Linguaggi regolari}
\paragraph{Corollario2(orale)}
\label{corollario-9.2}
\text{}
Se $L$ è generato da una grammatica lineare sinistra, allora $L$ è regolare.

\begin{tcolorbox}[colback=blue!10!white, colframe=blue!50!black, title=Dimostrazione]
    Sia $L = L(G)$, per una qualche grammatica lineare sinistra $G$.

    Si consideri $G^R$, la grammatica lineare destra ottenuta da $G$ invertendo le parti destre di tutte le sue regole.
    
    Chiaramente, $L(G^R) = (L(G))^R$.

    Per il \hyperref[teorema-9.2]{Teorema 9.2}, sicoome $G^R$ è lineare destra, $L(G^R)$ è regolare.
    
    Siccome i linguaggi regolari sono chiusi rispetto all'operazione di complemento, $L = (L(G^R))^R$ è regolare.

    Ma $(L(G^R))^R = ((L(G))^R)^R = L(G)$, e quindi $L = L(G)$ è regolare.
\end{tcolorbox}

\subsection{Left- vs Right-linear Grammars}
\paragraph{Corollario(orale)}
\label{corollario-9.3}
\text{}
$L$ ammette una grammatica lineare destra se e solo se ammette una grammatica lineare sinistra
\begin{tcolorbox}[colback=blue!10!white, colframe=blue!50!black, title=Dimostrazione]
    $L = L(G)$, per qualche grammatica lineare destra
    \begin{itemize}
        \item SSE $L$ è regolare (\hyperref[teorema-9.1]{Teorema 9.1} + \hyperref[teorema-9.2]{Teorema 9.2})
        \item SSE $L = L(G),$ per qualche grammatica lineare sinistra (\hyperref[corollario-9.1]{Corollario1} + \hyperref[corollario-9.2]{Corollario2})
    \end{itemize}
\end{tcolorbox}

\begin{tcolorbox}[colback=red!10!white, colframe=red!50!black, title=\textbf{OSS}]
    Questo corollario dà anche un algoritmo per passare da una grammatica lineare sinistra a una lineare destra, e viceversa (tramite automi).
    
    (esistono algoritmi che effettuano la trasformazione direttamanete e in maniera più efficiente)    
\end{tcolorbox}

\subsection*{Accettori di Linguaggi}
In corrispondenza di ogni tipo di grammatica, esiste una particolare macchina astratta in grado di accettare tutti e soli i linguaggi generati da quel tipo di grammatiche:

\begin{figure}[H]
    \centering
    \includegraphics[width=0.3\textwidth]{Immagini/116.png}
    \label{fig:your_image}
\end{figure}

\subsubsection{Linguaggi non regolari}
In questa sezione, mostriamo come provare che alcuni linguaggi non possono essere riconosciuti da alcun automa finito.

Consideriamo il linguaggio $B = \{0^{n}1^{n}|n \geq 0\} = \{\varepsilon,01,0011,000111,...\}$. Intuitivamente un DFA che riconosce $B$ dovrebbe ricordare quanti $0$ ha visto fin quando legge l'input, ma non è possibile con un numero finito di stati.
Di seguito presentiamo un metodo per provare che linguaggi come $B$ non sono regolari.
Solo perchè il linguaggio sembra richiedere memoria non limitata, non significa che non sia regolare.
Per esempio, consideriamo due linguaggi sull'alfabeto $\Sigma = \{0,1\}$:
\begin{itemize}
    \item $C = \{w \mid w \text{ ha lo stesso numero di simboli uguali a 0 e simboli uguali a 1}\}$ non è regolare;
    \item $D = \{w \mid w \text{ ha un numero uguale di occorrenze di 01 e 10 come sottostringhe}\}$.
\end{itemize}

Sorprendentemente, $D$ è regolare ed è accettato dal seguente DFA:

\begin{figure}[H]
    \centering
    \includegraphics[width=0.4\textwidth]{Immagini/20.png}
    \label{fig:your_image}
\end{figure}

\paragraph{Il pumping lemma}
\vspace{1em}
\text{}
\newline
Questo teorema afferma che tutti i linguaggi regolari hanno una proprietà speciale. Se noi possiamo mostrare che un linguaggio non ha questa proprietà, siamo sicuri che esso non è regolare. La proprietà afferma che tutte le stringhe nel linguaggio possono essere "replicate" se la loro lunghezza raggiunge almeno uno specifico valore speciale, chiamato la \textbf{lunghezza del pumping}. Questo significa che ogni tale stringa contiene una parte che può essere ripetuta un numero qualsiasi di volte ottenendo una stringa che appartiene ancora al linguaggio.

\paragraph{Teorema 1.70(orale)}
\label{teorema-1.70}
\text{}
\begin{tcolorbox}[colback=green!10!white, colframe=green!50!black, title=Teorema 1.70]
    Se $A$ è un linguaggio regolare, allora esiste un numero $p$ (la lunghezza del pumping) tale che se $s$ è una qualsiasi stringa in $A$ di lunghezza almeno $p$, allora $s$ può essere divisa in tre parti, $s = xyz$, soddisfacenti le seguenti condizioni:
    \begin{enumerate}
        \item per ogni $i \geq 0, xy^{i}z \in A$,
        \item $|y| > 0$, e
        \item $|xy| \leq p$.
    \end{enumerate}
\end{tcolorbox}

\begin{tcolorbox}[colback=red!10!white, colframe=red!50!black, title=IMPORTANTE!]
    Ricordiamo che $|s|$ rappresenta la lunghezza della stringa $s$, $y^{i}$ indica $i$ copie di $y$ concatenate insieme, e $y^{0}$ è uguale a $\varepsilon$.
\end{tcolorbox}

\begin{itemize}
    \item Quando $s$ è divisa in $xyz$, $x$ o $z$ potrebbe essere $\varepsilon$, ma la condizione 2 dice che $y \neq \varepsilon$ (senza la condizione 2 il teorema sarebbe banalmente vero).
    \item La condizione 3 afferma che le parti $x$ e $y$ insieme hanno lunghezza al più $p$; questa è una condizione tecnica supplementare che ogni tanto troviamo utile quando dimostriamo che alcuni linguaggi non sono regolari.
    \item Scritto più precisamente il pumping lemma è:
\end{itemize}
\vspace{1em}

\href{https://chatgpt.com/share/6759c242-4ec8-8011-9b41-8664075f8a1a}{\textcolor{blue}{Link a chatgpt che lo spiega meglio.}}

\textbf{PUMPING LEMMA:}
$$
A \text{ regolare }\Rightarrow \exists\text{ }p \in \mathbb{N}\text{ }\forall s\in\ A(\text{ }|s| \geq p \Rightarrow \exists \text{ }x,y,z \quad t.c.(\text{ }s = xyz\wedge |y|>0 \wedge |xy|\leq p \wedge \forall i\in \mathbb{N}.xy^{i}z \in A))
$$
\vspace{1em}

\textbf{CONTRAPPOSTA: (il prof dice di usare questa, daje G.!)}
$$
\forall\text{ }p \in \mathbb{N}\text{ }\exists s\in A(\text{ }|s| \geq p \wedge \forall \text{ }x,y,z (s \neq xyz \vee |y| = 0 \vee |xy| > p \vee \exists i\in \mathbb{N}.xy^{i}z \notin A)) \Rightarrow A \text{ non regolare}
$$
\vspace{1em}

\textbf{EQUIVALENTEMENTE:}
$$
\forall\text{ }p \in \mathbb{N}\text{ }\exists s\in A(\text{ }|s| \geq p \wedge \forall \text{ }x,y,z ((s = xyz \wedge |y| > 0 \wedge |xy| \leq p))) \Rightarrow A \text{ non regolare}
$$
\vspace{1em}

Utilizzo pratico (provare che $A$ non è regolare):
\begin{enumerate}
    \item Scegliere $p$ arbitrario.
    \item Scegliere $s \in A$ lunga al massimo $p$ e decomponila in tutti i possibili $xyz$, con $|y| > 0$ e $|xy| \leq p$.
    \item Per ogni decomposizione mostrare che $\forall i\in \mathbb{N}.xy^{i}z \notin A$.
    \item Concludere che $A$ non è regolare.
\end{enumerate}
\vspace{1em}

\textbf{IDEA:}
Sia $M = (Q,\Sigma,\delta,q_{1},F)$ un DFA che riconosce $A$. Assegnamo alla lunghezza del pumping $p$ il numero degli stati di $M$. Mostriamo che ogni stringa $s$ in $A$ di lunghezza almeno $p$ può essere divisa nelle tre parti $xyz$, soddisfacenti le nostre tre condizioni. Cosa accade se nessuna stringa in $A$ è di lunghezza almeno $p$? Allora il nostro compito è perfino più facile perchè il teorema diventa \textbf{banalmente} vero: ovviamente le tre condizioni valgono per tutte le stringhe di lunghezza almeno $p$ se non vi è alcuna di tali stringhe.
Se $s$ in $A$ ha lunghezza almeno $p$, consideriamo la sequenza di stati che $M$ attraversa nella computazione con input $s$. Inizia con lo stato iniziale $q_{1}$, poi va in $q_{3}$, poi per esempio in $q_{13}$. Se $s$ è in $A$, sappiamo che $M$ accetta $s$, quindi $q_{13}$ è uno stato accettante.
Se supponiamo che $n$ sia la lunghezza di $s$, la sequenza di stati $q_{1},q_{3},q_{20},...,q_{13}$ ha lunghezza $n+1$. Poichè $n$ è almeno $p$ , sappiamo che $n+1$ è più grande di $p$, il numero di stati di $M$. Quindi, la sequenza deve contenere uno stato che si ripete (\textbf{principio della piccionaia}).
La figura seguente mostra la stringa $s$ e la sequenza di stati che $M$ attraversa quando elabora $s$. Lo stato $q_{9}$ è quello che si ripete.

\begin{figure}[H]
    \centering
    \includegraphics[width=0.4\textwidth]{Immagini/21.png}
    \label{fig:your_image}
\end{figure}

Ora dividiamo $s$ nelle tre componenti $x,y$ e $z$. La componente $x$ è la
parte di $s$ che compare prima di $q_{9}$, la componente $y$ è la parte tra le due
occorrenze di $q_{9}$ e la componente $z$ è la parte restante di $s$, che viene dopo
la seconda occorrenza di $q_{9}$. Quindi $x$ porta $M$ dallo stato $q_{1}$ a $q_{9}$, $y$ riporta
$M$ da $q_{9}$ a $q_{9}$ e $z$ porta $M$ da $q_{9}$ allo stato accettante $q_{13}$, come mostrato
nella figura seguente.

\begin{figure}[H]
    \centering
    \includegraphics[width=0.4\textwidth]{Immagini/22.png}
    \label{fig:your_image}
\end{figure}

Vediamo perchè questa divisione di $s$ soddisfa le tre condizioni. Supponiamo di eseguire $M$ sull'input $xyyz$. Sappiamo che $x$ porta $q_{1}$ a $q_{9}$, e poi il primo $y$ lo riporta da $q_{9}$ a $q_{9}$, come fa il secondo $y$ e poi $z$ lo porta in $q_{13}$. Quindi essendo $q_{13}$ uno stato accettante, $M$ accetta l'input $xyyz$. Il discorso è analogo per $xy^{i}z$ per ogni $i>0$. Nel caso $i = 0$, $xy^{i}z=xz$, anch'essa accettata. Questo prova la condizione 1.
Per verificare la condizione 2, vediamo che $|y| > 0$, poichè era la parte di $s$ tra due diverse occorrenze dello stato $q_{9}$.
Per ottenere la condizione 3, ci assicuriamo che $q_{9}$ sia la prima ripetizione nella sequenza. Per il principio della piccionaia, i primi $p+1$ stati nella sequenza devono contenere una ripetizione. Quindi $|xy| \leq p$.
\vspace{1em}

\textbf{DIMOSTRAZIONE:}
Sia $M = (Q,\Sigma,\delta,q_{1},F)$ un DFA che riconosce $A$ e sia $p$ il numero di stati di $M$.
Sia $s = s_{1}s_{2}...s_{n}$ una stringa in $A$ di lunghezza $n$, dove $n \geq p$. Sia $r_{1},...r_{n+1}$ la sequenza di stati attraversati da $M$ mentre elabora $s$, quindi $r_{i+1} = \delta(r_{i},s_{i})$ per $1 \leq i \leq n$. Questa sequenza ha lunghezza $n+1$, che è almeno $p+1$. Due tra i primi $p+1$ stati devono essere lo stesso stato, per il principio della piccionaia. Chiamiamo il primo di questi $r_{j}$ e il secondo $r_{l}$. Poichè $r_{l}$ si presenta tra le prime $p+1$ posizioni in una sequenza che inizia in $r_{1}$, abbiamo $l \leq p+1$. Ora sia $x = s_{1}...s_{j-1},\text{ }y = s_{j}...s_{l-1}$ e $z = s_{l}...s_{n}$. Poichè $x$ porta $M$ da $r_{1} \text{ a }r_{j}$, $y$ porta $M$ da $r_{j}$ a $r_{j}$ e $z$ porta $M$ da $r_{j}$ a $r_{n+1}$, che è uno stato accettante, $M$ deve accettare $xy^{i}z \text{ per }i \geq 0$. Sappiamo che $j\neq l$, perciò $|y| > 0$; e $l \leq p+1,$ perciò $|xy| \leq p$. Quindi tutte le condizioni del pumping lemma sono rispettate.
\vspace{3em}

Per usare il pumping lemma per provare che un linguaggio $B$ non è regolare, in primo luogo si assuma che $B$ sia regolare per ottenere una contraddizione. Poi si usi il pumping lemma per assicurare l'esistenza di una lunghezza del pumping $p$ tale che tutte le stringhe di lunghezza maggiore o uguale a $p$ in $B$ possano essere iterate. In seguito, si trovi una stringa $s$ in $B$ che ha lunghezza maggiore o uguale a $p$, ma che non può essere iterata. Infine, si dimostri che $s$ non può essere iterata considerando tutti i modi di dividere $s$ in $x$, $y$ e $z$ (prendendo in considerazione la condizione 3 del pumping lemma
se è utile) e, per ogni tale divisione, trovando un valore $i$ tale che $xy^{i}z \notin B$. Questo passo finale spesso comporta il dover raggruppare i vari modi di dividere $s$ in diversi casi e l'analizzarli individualmente. L'esistenza di $s$ contraddirebbe il pumping lemma se $B$ fosse regolare. Quindi $B$ non può essere regolare. Trovare $s$ a volte richiede un po' di ragionamento creativo. Potresti dover cercare tra diversi candidati per s prima di scoprirne uno che funzioni. Prova con elementi di $B$ che sembrano esibire l'essenza della non regolarità di $B$.

\begin{figure}[H]
    \centering
    \includegraphics[width=0.3\textwidth]{Immagini/24.png}
    \includegraphics[width=0.3\textwidth]{Immagini/26.png}
    \includegraphics[width=0.3\textwidth]{Immagini/27.png}
    \includegraphics[width=0.3\textwidth]{Immagini/28.png}
    \caption{Esempi di applicazione del pumping lemma}
    \label{fig:your_image}
\end{figure}

\section{Linguaggi context-free}

In questo capitolo presentiamo le \textbf{grammatiche context-free}, un metedo più potente per descrivere linguaggi. Queste grammatiche possono descrivere alcuni aspetti che hanno una struttura ricorsiva.
I linguaggi associati alle grammatiche context-free sono chiamati \textbf{linguaggi context-free}.
Introduciamo anche gli \textbf{automi a pila} (pushdown automata), una classe di macchine che riconoscono i linguaggi context-free.

\subsection{Grammatiche context-free}

Un esempio di grammatica context-free, che chiamiamo $G_1$, è il seguente

\begin{align*}
    A &\rightarrow 0A1 \\
    A &\rightarrow B \\
    B &\rightarrow \text{\#}
\end{align*}

Una grammatica consiste di un insieme di \textbf{regole di sostituzione}, anche chiamate \textbf{produzioni}. Ogni regola appare come una linea nella grammatica, costituita da un simbolo e una stringa separati da una freccia. Il simbolo è chiamato \textbf{variabile}. La stringa consiste di variabili e altri simboli chiamati \textbf{terminali}
. I terminali sono analoghi ai simboli dell'alfabeto di input e sono spesso reappresentati da lettere minuscole, numeri o simboli speciali. Una delle variabili è chiamata \textbf{variabile iniziale}. Essa generalmente si trova sul lato sinistro della regola più in alto.
Per esempio, la grammatica $G_1$ ha tre regole. Le variabili di $G_1$ sono $A$ e $B$, e $A$ la variabile iniziale. I suoi terminali sono $0,1$ e $\#$.
\vspace{2em}

\paragraph{Definizione formale di grammatica context-free (CFG):}
\text{  }

\begin{tcolorbox}[colback=yellow!10!white, colframe=yellow!50!black, title=Definizione 2.2]
    Una grammatica context-free è una quadrupla $G = (V,\Sigma,R,S)$ dove:
    \begin{enumerate}
        \item $V$ è un insieme finito di \textbf{variabili},
        \item $\Sigma$ è un insieme finito di \textbf{terminali} disgiunto da $V$,
        \item $R$ è un insieme finito di \textbf{regole}, o produzioni, ciascuna delle quali è della forma $A \rightarrow \alpha$, dove $A$ è una variabile e $\alpha$ è una stringa di variabili e terminali (cioè $\alpha \in (V \cup \Sigma)^{*}$), e
        \item $S$ è la variabile iniziale.
    \end{enumerate}
\end{tcolorbox}

Se $u,v,w$ sono stringhe di variabili e terminali e $A \rightarrow w$ è una regola della grammatica, diciamo che $uAv$ \textbf{produce} $uwv$, e lo denotiamo con $uAv \Rightarrow uwv$. Diciamo che $u \textbf{ deriva } v$, e lo denotiamo con $u \Rightarrow^* v$, se $u = v$ o se esiste una sequenza $u_1,u_2,...,u_k, \text{ con } k \geq 0$ e 
$$
u \Rightarrow u_1 \Rightarrow u_2 \Rightarrow ... \Rightarrow u_k \Rightarrow v.
$$
Il \textbf{linguaggio della grammatica} è $\{w \in \Sigma^{*} \mid S \Rightarrow^* w\}$, cioè l'insieme di tutte le stringhe che possono essere derivate dalla variabile iniziale.

\textbf{Esempi di grammatiche context-free}

\begin{figure}[H]
    \centering
    \includegraphics[width=0.5\textwidth]{Immagini/30.png}
    \includegraphics[width=0.5\textwidth]{Immagini/31.png}
    \includegraphics[width=0.5\textwidth]{Immagini/32.png}
    \caption{Esempio di grammatica context-free}
    \label{fig:grammar_example1}
\end{figure}

\textbf{Progettare grammatiche context-free}
\vspace{1em}

Le tecniche seguenti sono utili, singolarmente o combinate, quando affrontiamo il problema di costruire una CFG.
Innanzitutto, molti CFL (context-free language) sono l'unione di CFL più semplici. Se devi costruire una CFG per un CFL che puoi dividere in componenti più semplici, fallo e poi costruisci grammatiche separate per ciascuna componente. Queste singole grammatiche possono facilmente essere fuse in una grammatica per il linguaggio iniziale unendo le loro regole e poi aggiungendo la nuova regola $S \rightarrow S_1 \mid S_2 \mid ... \mid S_k$, dove le variabili $S_i$ sono le variabili iniziali per le grammatiche individuali.
Per esempio, per ottenere una grammatica per il linguaggio $\{0^n 1^n \mid n \geq 0\} \cup \{1^n 0^n \mid n \geq 0\}$, costruiamo prima la grammatica
$$
S_1 \rightarrow 0S_1 1 \mid \varepsilon
$$
per il linguaggio $\{0^n 1^n \mid n \geq 0\}$ e poi la grammatica
$$
S_2 \rightarrow 1S_2 0 \mid \varepsilon
$$
per il linguaggio $\{1^n 0^n \mid n \geq 0\}$ e poi aggiungiamo la regola $S \rightarrow S_1 \mid S_2$ ottenendo la grammatica
\begin{align*}
    S \rightarrow S_1 \mid S_2 \\
    S_1 \rightarrow 0S_1 1 \mid \varepsilon \\
    S_2 \rightarrow 1S_2 0 \mid \varepsilon.
\end{align*}

In secondo luogo, costruire una CFG per un linguaggio che sia regolare è facile se si può prima costruire un DFA per quel linguaggio. 
Si può trasformare un DFA in una CFG equivalente nel modo seguente. Si introduca una variabile $R_i$ per ogni stato $q_i$ del DFA. 
Si aggiunga la regola $R_i \rightarrow aR_j$ alla CFG se $\delta(q_i,a) = q_j$ è una transizione nel DFA. 
Si aggiunga la regola $R_i \rightarrow \varepsilon$ se $q_i$ è uno stato accettante del DFA. Si assuma $R_0$ come variabile iniziale della grammatica, dove $q_0$ è lo stato iniziale della macchina. 
Si può verificare che la CFG risultante genera lo stesso linguaggio di quello riconosciuto dal DFA.
\vspace{2em}

\textbf{Ambiguità}

Qualche volta una grammatica può generare la stessa stringa in più modi diversi. Una tale stringa avrà diversi alberi sintattici e quindi diversi significati.
Se una grammatica genera la stessa stringa in più modi diversi, diciamo che la stringa è derivata \textbf{ambiguamente} in quella grammatica. Se una grammatica genera alcune stringhe ambiguamente, diciamo che la grammatica è \textbf{ambigua}.

\begin{figure}[H]
    \centering
    \includegraphics[width=0.5\textwidth]{Immagini/33.png}
    \caption{Esempio di grammatica context-free ambigua}
    \label{fig:ambiguous_grammar_example1}
\end{figure}
Questa grammatica non tiene conto delle usuali regole di precendenza e quindi può raggruppare $+$ prima di $\times$ o viceversa. Quindi la grammatica è ambigua.
Quando diciamo che una grammatica genera ambiguamente una stringa, intendiamo che la stringa ha due diversi alberi sintattici, non due differenti derivazioni.
Due derivazioni possono differire sono nell'ordine in cui esse sostituiscono le variabili ma non nell aloro struttura complessiva.
Una derivazione di una stringa $w$ in una grammatica $G$ è una \textbf{derivazione a sinistra} se a ogni passo la variabile sostituita è quella che si trova più a sinistra.
\vspace{1em}

\begin{tcolorbox}[colback=blue!10!white, colframe=blue!50!black, title=Definizione 2.7]
    Una stringa $w$ è derivata \textbf{ambiguamente} in una grammatica context-free $G$ se essa ha due o più diverse derivazioni a sinistra. Una grammatica $G$ è \textbf{ambigua} se essa genera qualche stringa ambiguamente.
\end{tcolorbox}

\newpage
\paragraph{Forma normale di Chomsky}
\label{Forma normale di Chomsky}
\text{ }

Quando si lavora con le grammatiche context-free, è spesso conveniente averle in forma semplificata. La forma di Chomsky è una delle più semplici e utili e viene utilizzata per dare algoritmi che lavorano con grammatiche context-free.

\begin{tcolorbox}[colback=blue!10!white, colframe=blue!50!black, title=Forma Normale di Chomsky]
    Una grammatica context-free è in \textbf{forma normale di Chomsky} se ogni regola è della forma:
    \begin{enumerate}
        \item $A \rightarrow BC$, dove $A, B, C$ sono variabili e né $B$ né $C$ sono la variabile iniziale, oppure
        \item $A \rightarrow a$, dove $A$ è una variabile e $a$ è un terminale, oppure
        \item $S \rightarrow \varepsilon$, dove $S$ è la variabile iniziale.
    \end{enumerate}
\end{tcolorbox}

\paragraph{Teorema 2.9(orale 24-30)} 
\text{(se non hai voglia di leggerti la dimostrazione, leggi le regole nell'esempio).}
\begin{tcolorbox}[colback=yellow!10!white, colframe=yellow!50!black, title=Teorema 2.9]
    Ogni linguaggio context-free è generato da una grammatica context-free in forma normale di Chomsky.
\end{tcolorbox}

\textbf{IDEA.}
Possiamo trasformare ogni grammatica $G$ in forma normale di Chomsky. La trasformazione ha diversi passi nei quali le regole che violano le condizioni sono rimpiazzate con regole equivalenti. Innanzitutto, aggiungiamo una nuova variabile iniziale. Poi eliminiamo tutte le \textbf{$\varepsilon-regole$} della forma $A \rightarrow \varepsilon$. Eliminiamo anche tutte le \textbf{regole unitarie} della forma $A \rightarrow B$. In entrambi i casi, modifichiamo la grammatica in modo da essere sicuri che generi ancora lo stesso linguaggio. Infine, trasformiamo le restanti regole nella forma adeguata.
\vspace{1em}

\textbf{DIMOSTRAZIONE.}
In primo luogo, aggiungiamo una nuova variabile iniziale $S_0$ e la regola $S_0 \rightarrow S$, dove $S$ era la variabile iniziale di partenza.
Questo cambiamento garantisce che la variabile iniziale non compare sul lato destro di una regola.
In secondo luogo, ci occupiamo di tutte le $\varepsilon-regole$. Eliminiamo una $\varepsilon-regola$, dove $A$ non è la variabile iniziale. Poi, per ogni occorenza di $A$ sul lato destro di una regola, aggiungiamo una nuova regola con quell'occorrenza cancellata. In altre parole, se $R \rightarrow uAv$ è una regola in cui $u$ e $v$ sono stringhe di variabili e terminali, aggiungiamo la regola $R \rightarrow uv$. Facciamo così per ogni occorrenza di $A$, in modo che la regola $R \rightarrow uAvAw$ faccia si che vengano aggiunte $R\rightarrow uvAw$, $R \rightarrow uAvw$, ed $R \rightarrow uvw$. Se abbiamo la regola $R \rightarrow A$, aggiungiamo $R \rightarrow \varepsilon$. Ripetiamo questi passi fino a eliminare tutte le $\varepsilon-regole$ che non coinvolgono la variabile iniziale.
Inoltre, ci occupiamo delle regole unitarie. Eliminiamo una regola unitaria $A \rightarrow B$. Poi, per ogni regola $B \rightarrow u$, aggiungiamo la regola $A \rightarrow u$ a meno che questa non sia una regola unitaria precendentemente cancellata.
Come prima, $u$ è una stringa di variabili e terminali. Ripetiamo questi passi fino a eliminare tutte le regole unitarie.
Infine, trasformiamo tutte le restanti regole nella forma appropriata.
Rimpiazziamo ogni regola $A \rightarrow u_1u_2...uk$ dove $k \geq 3$ e ciascun $u_i$ è una variabile o un simbolo terminale, con le regole $A \rightarrow u_1A_1,\text{ }A_1 \rightarrow u_2A_2,\text{ }A_2 \rightarrow u_3A_3,...\text{and } A_{k-2} \rightarrow u_{k-1}u_k$. Le $A_i$ sono nuove variabili.
Rimpiazziamo ogni terminale $u_i$ nelle precedenti regole con la nuova variabile $U_i$ e aggiungiamo la regola $U_i \rightarrow u_i$.
\vspace{3em}

\href{https://chatgpt.com/share/6759c2bc-a61c-8011-af18-0db9d52bdd29}{\textcolor{blue}{Link a chatgpt che lo spiega meglio.}}

\begin{figure}[H]
    \centering
    \includegraphics[width=0.3\textwidth]{Immagini/34.png}
    \includegraphics[width=0.3\textwidth]{Immagini/35.png}
    \includegraphics[width=0.3\textwidth]{Immagini/36.png}
    \caption{Esempio regole Teorema 2.9}
    \label{fig:your_image}
\end{figure}

\subsection{Automi a pila}
In questa sezione introduciamo un nuovo tipo di modello computazionale chiamato \textbf{automa a pila} (pushdown automata). Questi automi sono come gli automi finiti non deterministici ma hanno una componente in più chiamata \textbf{pila} stack. La pila fornisce memoria aggiuntiva oltre alla quantità finita di essa disponibile nel controllo. La pila consente a tali automi di riconoscere alcuni linguaggi non regolari.
La figura seguente è una rappresentazione schematica di un automa finito. Il controllo rappresenta gli stati e la funzione di transizione, il nastro contiene la stringa di input, e la freccia rappresenta la testina sull'input, che indica il successivo simbolo di input da leggere.
\begin{figure}[H]
    \centering
    \includegraphics[width=0.3\textwidth]{Immagini/37.png}
    \caption{Rappresentazione schematica di un automa a pila}
    \label{fig:pushdown_automaton1}
\end{figure}
Un automa a pila (PDA) può scrivere simboli nella pila e rileggerli in seguito. In qualunque momento il simbolo sulla cima (top) della pila può essere letto e rimosso.
L'operazione di scrivere un simbolo sulla pila è spesso chiamata \textbf{push}, quella di eliminare un simbolo dalla pila è spesso chiamata \textbf{pop}.
In altre parole, una pila è in dispositivo di memoria "last in, first out".

Una pila è molto importante perchè può mantenere una quantità non limitata di informazioni. Ricordiamo che un automa finito non può riconoscere il linguaggio $\{0^n 1^n \mid n \geq 0\}$ poichè non può memorizzare numeri molto grandi nella sua memoria finita. Un PDA è in grado di riconoscere questo linguaggio poichè può usare la sua pila per memorizzare il numero di simboli uguali a 0 che ha visto.

La seguente descrizione informale mostra come opera l'automa per questo linguaggio.
\vspace{1em}

Legge i simboli di input. Scrive ciascuno 0 letto sulla pila. Non appena vede simboli uguali a 1, cancella uno 0 dalla pila per ogni 1 letto. Se la lettura dell'input termina esattamente quando la pila diventa priva di simboli uguali a 0, accetta l'input. Se la pila si svuota ma restano simboli uguali a 1 o se i simboli uguali a 1 sono finiti ma la pila contiene ancora simboli uguali a 0 o se qualche 0 appare nell'input dopo i simboli uguali a 1, rifiuta l'input.
\vspace{2em}

Gli automi a pila possono essere non deterministici. 
Automi a pila deterministici e non deterministici non sono computazionalmente equivalenti. 
Gli automi a pila non deterministici riconoscono alcuni linguaggi che nessun automa a pila deterministico può riconoscere. Ricordiamo che gli automi finiti deterministici e non deterministici riconoscono la stessa classe di linguaggi, quindi la situazione per gli automi a pila è diversa. Ci concentreremo sugli automi a pila non deterministici perchè questi automi sono computazionalmente equivalenti alle grammatiche context-free.

\paragraph{Definizione formale di automa a pila}
\text{ }

La definizione di automa a pila è simile a quella di un automa finito, tranne che per la pila. La pila è un dispositivo che contiene simboli presi da un qualche alfabeto. La macchina può usare differenti alfabeti per il suo input e la sua pila, quindi ora specifichiamo sia un alfabeto di simboli di input $\Sigma$ sia un alfabeto di pila $\Gamma$.

Nel cuore di ogni definizione di automa c'è la funzione di transizione, che descrive il suo comportamento. Ricordiamo che $\Sigma_\varepsilon = \Sigma \cup \{\varepsilon \}$ e $\Gamma_\varepsilon = \Gamma \cup \{\varepsilon \}$. il dominio della funzione di transizione è $Q \times \Sigma_{\varepsilon} \times \Gamma_\varepsilon$.
Quindi lo stato corrente, il prossimo simbolo di input letto, e il simbolo sulla cima della pila determinano la mossa seguente di un automa a pila.
L'uno o l'altro simbolo può essere $\varepsilon$, il che determina che la macchina si muova senza leggere un simbolo di input o senza leggere un simbolo dalla pila.

Per quanto riguarda il codominio della funzione di transizione, dobbiamo considerare cosa può fare l'automa quando è in una specifica situazione. Esso può trovarsi in un nuovo stato ed eventualmente scrivere un simbolo sulla cima della pila. La funzione $\delta$ può indicare quest'azione restituendo un elemento di $Q$ insieme a un elemento di $\Gamma_\varepsilon$, cioè un elemento di $Q \times \Gamma_\varepsilon$. Poichè permettiamo il non determinismo in questo modello, una situazione può avere diverse mosse successive lecite. La funzione di transizione ingloba il non determinismo nel modo usuale, restituendo un insieme di elementi di $Q \times \Gamma_{\varepsilon}$, cioè un elemento di $P(Q \times \Gamma_{\varepsilon})$.
Mettendo tutto questo insieme, la nostra funzione di transizione $\delta$ assume la forma
$\delta : Q \times \Sigma_{\varepsilon} \times \Gamma_{\varepsilon} \rightarrow P(Q \times \Gamma_{\varepsilon})$.

\begin{tcolorbox}[colback=blue!10!white, colframe=blue!50!black, title=Automi a Pila]
Un automa a pila (PDA) è una sestupla $(Q, \Sigma, \Gamma, \delta, q_0, F)$, dove $Q, \Sigma, \Gamma, F$ sono tutti insiemi finiti, e
\begin{enumerate}
    \item $Q$ è l'insieme finito di stati,
    \item $\Sigma$ è l'alfabeto di input,
    \item $\Gamma$ è l'alfabeto della pila,
    \item $\delta: Q \times \Sigma_\varepsilon \times \Gamma_\varepsilon \rightarrow P(Q \times \Gamma_\varepsilon)$ è la funzione di transizione,
    \item $q_0 \in Q$ è lo stato iniziale,
    \item $F \subseteq Q$ è l'insieme degli stati accettanti.
\end{enumerate}
\end{tcolorbox}

Un automa a pila $M = (Q, \Sigma, \Gamma, \delta, q_0, F)$ computa come segue. Accetta un input $w$ se $w$ può essere scritto come $w = w_1w_2...w_m$, dove ciascun $w_i \in \Sigma_\varepsilon$
ed esistono sequenze di stati $r_0,r_1,...,r_m \in Q$ e di stringhe $s_0,s_1,...s_m \in \Gamma^*$ che soddisfano le tre condizioni seguenti. 
Le stringhe $s_i$ rappresentano la sequenza del contenuto della pila che $M$ ha su un ramo accettante della computazione.

\begin{enumerate}
    \item $r_0 = q_0$ e $s_0 = \varepsilon$. Questa condizione significa che $M$ inizia correttamente, nello stato iniziale e con una pila vuota.
    \item Per $i = 0,...,m-1$, abbiamo $(r_{i+1}, b) \in \delta(r_i, w_{i+1}, a)$, dove $s_i = at$ e $s_{i+1} = bt$ per qualche $a, b \in \Gamma_\varepsilon$ e $t \in \Gamma^*$. Questa condizione afferma che $M$ si muove correttamente in base allo stato, al simbolo di pila, e al prossimo simbolo in input.
    \item $r_m \in F$. Questa condizione afferma che alla fine dell'input $M$ si trova in uno stato accettante.
\end{enumerate}

\begin{figure}[H]
    \centering
    \includegraphics[width=0.5\textwidth]{Immagini/38.png}
    \includegraphics[width=0.5\textwidth]{Immagini/39.png}
    \caption{Esempio di automa a pila}
    \label{fig:pushdown_automaton_example1}
\end{figure}

La definizione di un PDA non contiene alcun meccanismo esplicito per permettere al PDA di controllare se la pila è vuota. Questo PDA è in grado di ottenere lo stesso effetto inserendo inizialmente un simbolo speciale \textdollar \text{ }nella pila. Allora nel caso in cui veda il \textdollar \text{ }di nuovo, sa che la pila è di fatto vuota.
Questo PDA è in grado di ottenere lo stesso risultato poichè lo stato accettante potrebbe essere eventualmente raggiunto solo quando la macchina è alla fine dell'input. Qindi d'ora in poi assumiamo che i PDA possono controllare la fine dell'input e sappiamo che possiamo implementarlo nello stesso modo.

\begin{figure}[H]
    \centering
    \includegraphics[width=0.5\textwidth]{Immagini/40.png}
    \includegraphics[width=0.5\textwidth]{Immagini/41.png}
    \includegraphics[width=0.5\textwidth]{Immagini/42.png}
    \caption{Esempio di automa a pila}
    \label{fig:nuova_immagine}
\end{figure}

\paragraph{Equivalenza con le grammatiche context-free}
\text{ }

In questa sezione mostriamo che grammatiche context-free e automi a pila sono computazionalmente equivalenti. Entrambi sono in grado di descrivere la classe dei linguaggi context-free.
Mostreremo come trasformare ogni grammatica context-free in automa a pila che riconosce lo stesso linguaggio e viceversa.
Ricordiamo che abbiamo definito un linguaggio context-free come un linguaggio che può essere descritto con una grammatica context-free.
Quindi il nostro obiettivo è il teorema seguente.

\paragraph{Teorema 2.20}
\label{teorema-2.20}
\text{ }

\begin{tcolorbox}[colback=yellow!10!white, colframe=yellow!50!black, title=Teorema 2.20]
    Un linguaggio è context-free se e solo se qualche automa a pila lo riconosce.
\end{tcolorbox}

\newpage
\paragraph{Lemma 2.21(orale)}
\text{ }

\begin{tcolorbox}[colback=blue!10!white, colframe=blue!50!black, title=Lemma 2.21 (orale)]
    Se un linguaggio è  context-free, allora esiste un automa a pila che lo riconosce.
\end{tcolorbox}

\textbf{IDEA.}

Sia $A$ un CFL. Dalla definizione sappiamo che esiste una CFG, $G$, che genera $A$. Mostriamo come trasformare $G$ in un PDA equivalente, che chiamiamo $P$.
Il PDA $P$ che ora descriviamo, opererà accettando il suo input $w$, se $G$ genera tale input, determinando se esiste una derivazione per $w$. Ricordiamo che una derivazione è semplicemente la sequenza di sostituzioni fatte nel processo di generazione di una stringa mediante una grammatica.
Ogni passo di derivazione produce una \textbf{stringa intermedia} di variabili e terminali. Noi progettiamo $P$ in modo che possa stabilire se una serie di sostituzioni che usano le regole di $G$ possa condurre dalla variabile iniziale a $w$.

Una delle difficoltà nell'esaminare se esiste una derivazione per $w$ consiste nel capire quali sostituzioni fare. Il non determinismo di un PDA gli consente di ipotizzare la sequenza di sostituzioni giuste. In ogni passo della derivazione, una delle regole per una particolare variabile è scelta non deterministicamente e usata per sostituire quella variabile.
Il PDA $P$ inizia scrivendo la variabile iniziale sulla sua pila. 
Esso passa attraverso una serie di stringhe intermedie, facendo una sostituzione dietro l'altra. 
Infine può giungere a una stringa che contiene solo simboli terminali, il che significa che ha usato la grammatica per derivare una stringa. 
Allora $P$ accetta se questa stringa è identica alla stringa che ha ricevuto in input.

Implementare questa strategia su un PDA richede un'idea supplementare.
Dobbiamo capire come il PDA immagazzina le stringhe intermedie quando passa dall'una all'altra. 
Usare semplicemente la pila per immagazzinare ciascuna stringa intermedia è allettante. 
Tuttavia non funziona del tutto, poichè il PDA ha bisogno di trovare le variabili nelle stringhe intermedie e fare sostituzioni. 
Il PDA può accedere solo al simbolo sulla cima della pila e questo potrebbe essere un simbolo terminale e non una variabile. 
Il modo per aggirare questo problema è mantenere solo parte della stringa intermedia sulla pila: i simboli che iniziano con la prima variabile nella stringa intermedia. 
Tutti i simboli terminali che compaiono prima della prima variabile sono subito abbinati con i simboli nella stringa di input.

La figura seguente mostra il PDA $P$.

\begin{figure}[H]
    \centering
    \includegraphics[width=0.7\textwidth]{Immagini/43.png}
    \label{fig:pda_example1}
\end{figure}

Quello che segue è una descrizione informale di $P$.
\begin{enumerate}
    \item Inserisce il simbolo marcatore \textdollar \text{ }e la variabile iniziale sulla pila.
    \item Ripete i seguenti passi finché la pila non è vuota:
    \begin{itemize}
        \item Se sulla cima della pila c'è il simbolo di una variabile $A$, sceglie non deterministicamente una delle regole per $A$ e sostituisce $A$ con la stringa sul lato destro della regola.
        \item Se sulla cima della pila c'è un simbolo terminale $a$, legge il simbolo seguente dall'input e lo confronta con $a$. Se essi sono uguali, ripete. Se essi non sono uguali, rifiuta su questo ramo del non determinismo.
        \item Se sulla cima della pila c'è il simbolo \textdollar \text{ }, entra nello stato accettante. In questo modo accetta l'input se esso è stato completamente letto.
    \end{itemize}
\end{enumerate}
\vspace{1em}
 
\textbf{DIMOSTRAZIONE.}

Ora diamo i dettagli formali della costruzione dell'automa a pila $P = (Q,\Sigma,\Gamma,\delta,q_{start},F)$. 
Per rendere più chiara la costruzione, usiamo una notazione abbreviata per la funzione di transizione.
Questa notazione fornisce un modo per scrivere un'intera stringa sulla pila in un passo della macchina. 
Possiamo simulare quest'azione introducendo stati aggiuntivi per scrivere la stringa un simbolo alla volta, come realizzato nella seguente costruzione formale.

Siano $q$ ed $r$ stati del PDA e siano $a$ in $\Sigma_\varepsilon$ ed $s$ in $\Gamma_\varepsilon$.
Supponiamo di volere che il PDA vada da $q$ a $r$ quando legge $a$ ed elimina $s$.
Inoltre, vogliamo che esso inserisca l'intera stringa $u = u_1...u_l$ sulla pila simultaneamente.
Possiamo eseguire questa azione introducendo nuovi stati $q_1,...,q_{l-1}$ e definendo la funzione di transizione come segue:
\[
\begin{aligned}
\delta(q,a,s) & \text{ contiene } (q_1,u_l), \\
\delta(q_1,\varepsilon,\varepsilon) & = \{(q_2,u_{l-1})\}, \\
\delta(q_2,\varepsilon,\varepsilon) & = \{(q_3,u_{l-2})\}, \\
& \vdots \\
\delta(q_{l-1},\varepsilon,\varepsilon) & = \{(r,u_1)\}.
\end{aligned}
\]

Useremo la notazione $(r,u) \in \delta(q,a,s)$ per denotare che quando $q$ è lo stato in cui si trova l'automa, $a$ è il prossimo simbolo di input ed $s$ è il simbolo sulla cima della pila, il PDA può leggere $a$ ed eliminare $s$, poi inserire la stringa $u$ nella pila e passare nello stato $r$. La figura seguente ne mosta la realizzazione.

\begin{figure}[H]
    \centering
    \includegraphics[width=0.5\textwidth]{Immagini/46.png}
    \label{fig:notation_example1}
\end{figure}

Gli stati di $P$ sono $Q = \{q_{start},q_{loop},q_{accept}\} \cup E$ dove $E$ è l'insieme degli stati necessari per realizzare l'abbreviazione appena descritta.
Lo stato iniziale è $q_{start}$. Lo stato accettante è $q_{accept}$.

La funzione di transizione è definita come segue. 
Cominciamo inizializzando la pila inserendo i simboli \textdollar e $S$, realizzando così il passo 1 nella descrizione informale:
$\delta(q_{start},\varepsilon,\varepsilon) = \{(q_{loop},\textdollar S)\}$.
Poi aggiungiamo le transizioni per il ciclo principale del passo 2.

In primo luogo, trattiamo il caso (a) in cui la cima della pila contiene una variabile.
Poniamo $\delta(q_{loop},\varepsilon,A) = \{(q_{loop},w) \mid \text{dove } A \rightarrow w \text{ è una regola in }R\}$.

In secondo luogo, trattiamo il caso (b) in cui la cima della pila contiene un terminale.
Poniamo $\delta(q_{loop},a,a) = \{(q_{loop},\varepsilon)\}$.

Infine, trattiamo il caso (c) in cui il marcatore scelto per indicare la pila vuota \textdollar \text{ }è sulla cima della pila.
Poniamo $\delta(q_{loop},\varepsilon,\textdollar) = \{(q_{accept},\varepsilon)\}$.

Il diagramma di stato è mostrato nella figura seguente.
\begin{figure}[H]
    \centering
    \includegraphics[width=0.4\textwidth]{Immagini/47.png}
    \label{fig:state_diagram1}
\end{figure}
Questo completa la prova del Lemma 2.21.

\begin{figure}[H]
    \centering
    \includegraphics[width=0.5\textwidth]{Immagini/48.png}
    \label{fig:your_imag1}
\end{figure}

Ora dimostriamo la direzione inversa del \hyperref[teorema-2.20]{\textcolor{blue}{Teorema 2.20}}.
Vogliamo trasformare un PDA in una CFG
Progettiamo la grammatica per simulare l'automa.
Questo compito è impegnativo perchè "programmare" un automa è più facile che "programmare" una grammatica.

\paragraph{Lemma 2.27(orale 24-30)}
\label{lemma-2.27}
\text{ }

\begin{tcolorbox}[colback=blue!10!white, colframe=blue!50!black, title=Lemma 2.27]
    Se un linguaggio è riconosciuto da un automa a pila, allora esso è context-free.
\end{tcolorbox}

\textbf{IDEA.}

Abbiamo un PDA $P$ e vogliamo costruire una CFG $G$ che genera tutte le stringhe che $P$ accetta. 
In altre parole, $G$ dovrebbe generare una stringa se quella stringa fa andare il PDA dal suo stato iniziale a uno stato accettante.

Per raggiungere questo risultato, progettiamo una grammatica che fa un po' di più.
Per ciascuna coppia di stati $p$ e $q$ in $P$, la grammatica avrà una variabile $A_{pq}$.
Questa variabile genera tutte le stringhe che possano portare $P$ da $p$ con pila vuota a $q$ con pila vuota.
Si osservi che tali stringhe possono anche condurre $P$ da $p$ a $q$, indipendentemente dal contenuto della pila in $p$, lasciando la pila in $q$ nella stessa condizione in cui era in $p$.

Innanzitutto, semplifichiamo il nostro compito modificando leggermente $P$ per munirlo delle seguenti tre casistiche.

\begin{enumerate}
    \item Ha un unico stato accettabile $q_{accept}$.
    \item Svuota la sua pila prima di accettare.
    \item Ciascuna transizione inserisce un simbolo sulla pila (effetua push) o ne elimina uno dalla pila (effettua pop), ma non fa entrambe le azioni contemporaneamente.
\end{enumerate}

Dare a $P$ le caratteristiche 1 e 2 è facile.
Per munirlo della caratteristica 3, sostituiamo ciascuna transizione che contemporaneamente elimina ed inserisce simboli con una sequenza di due transizioni che attraversa un nuovo stato, e sostituiamo ogni transizione che non elimina ne inserisce simboli con una sequenza di due transizioni che inserisce e poi elimina un simbolo arbitrario della pila.

Per progettare $G$ in modo che $A_{pq}$ generi tutte le stringhe che portano $P$ da $p$ con pila vuota a $q$ con pila vuota, dobbiamo capire come $P$ agisce su queste stringhe.
Per ognuna di tali stringhe $x$, la prima mossa di $P$ su $x$ deve essere un push, poichè ogni mossa è un push o un pop e $P$ non può eliminare da una pila vuota.
Analogamente, l'ultima mossa su $x$ deve essere un pop perchè alla fine la pila deve essere vuota.

Durante la computazione di $P$ su $x$, si presentano due eventualità.
O il simbolo eliminato alla fine è il simbolo che era stato inserito all'inizio, oppure no.
Nel primo caso, la pila potrebbe essere vuota solo all'inizio e alla fine della computazione di $P$ su $x$.
Altrimenti, il simbolo inizialmente inserito deve essere stato eliminato in qualche punto prima della fine di $x$ e quindi la pila si svuota in questo punto.
Simuliamo la prima possibilità con la regola $A_{pq} \rightarrow aA_{rs}b$ dove $a$ è l'input letto nella prima mossa, $b$ è l'input letto nell'ultima mossa, $r$ è lo stato che segue $p$ ed $s$ è lo stato che precede $q$.
Simuliamo la seconda possibilità con la regola $A_{pq} \rightarrow A_{pr}A_{rq}$, dove $r$ è lo stato in cui la pila diventa vuota.
\vspace{1em}

\textbf{DIMOSTRAZIONE.}

Sia $P = (Q,\Sigma,\Gamma,\delta,q_{0},q_{accept})$ e costruiamo $G$.
L'insieme delle variabili di $G$ è $\{A_{pq} \mid p,q \in Q\}$.
La variabile iniziale è $A_{q_{0},q_{accept}}$.
Ora descriviamo le regole di $G$ nei seguenti tre punti.
\begin{enumerate}
    \item Per ogni $p,q,r,s \in Q,\text{ } u \in \Gamma \text{ e } a,b \in \Sigma_\varepsilon, \text{ se } \delta(p,a,\varepsilon)\text{ contiene } (r,u) \text{ e } \delta(s,b,u) \text{ contiene } (q,\varepsilon)$, poni la regola $A_{pq} \rightarrow aA_{rs}b$ in $G$.
    \item Per ogni $p,q,r \in Q$, poni la regola $A_{pq} \rightarrow A_{pr}A_{rq}$ in $G$.
    \item Infine, per ogni $p \in Q$, poni la regola $A_{pp} \rightarrow \varepsilon$ in $G$.
\end{enumerate}

Si può intuire questa costruzione dalle figure seguenti.

\begin{figure}[H]
    \centering
    \includegraphics[width=0.4\textwidth]{Immagini/49.png}
    \caption{Esempio di grammatica context-free}
    \label{fig:your_image}
\end{figure}

Ora proviamo che questa costruzione funziona dimostrando che $A_{pq}$ genera $x$ se e solo se $x$ porta $P$ da $p$ con la pila vuota a $q$ con la pila vuota.
Consideriamo ogni direzione del "se e solo se" come un enuciato separato.

\begin{tcolorbox}[colback=yellow!10!white, colframe=yellow!50!black, title=Fatto 2.30]
    Se $A_{pq}$ genera $x$, allora $x$ può portare $P$ da $p$ con la pila vuota a $q$ con la pila vuota.
\end{tcolorbox}

Dimostriamo questo enunciato per induzione sul numero dei passi nella derivazione di $x$ da $A_{pq}$.
\vspace{1em}

\textbf{Base.}
La derivazione è in un solo passo.

Una derivazione in un solo passo deve usare una regola il cui lato destro non contenga variabili.
Le uniche regole in $G$ con nessuna variabile sul lato destro sono $A_{pp} \rightarrow \varepsilon$.
Ovviamente, l'input $\varepsilon$ porta $P$ da $p$ con la pila vuota a $p$ con la pila vuota quindi la base è dimostrata.
\vspace{1em}

\textbf{Passo induttivo.}
Assumiamo che l'enunciato sia vero per le derivazioni di lunghezza al più $k$, dove $k \geq 1$, e proviamo che esso è vero per le derivazioni di lunghezza $k+1$.

Supponiamo che $A_{pq}$ $\Rightarrow^{*} x$ in $k+1$ passi.
Il primo passo in questa derivazione è $A_{pq} \Rightarrow aA_{rs}b$ oppure $A_{pq} \rightarrow A_{pr}A_{rq}$.
Trattiamo i due casi separatamente.

Nel primo caso, consideriamo la parte $y$ di $x$ che $A_{rs}$ genera, quindi $x = ayb$.
Poichè $A_{rs} \Rightarrow^{*} y$, in $k$ passi, l'ipotesi induttiva ci dice che $P$ può andare da $r$ con la pila vuota a $s$ con la pila vuota.
Poichè $A_{pq} \rightarrow aA_{rs}b$ è una regola di $G, \delta(p,a,\varepsilon)$ contiene $(r,u)$ e $\delta(s,b,u)$ contiene $(q,\varepsilon)$, per qualche simbolo di pila $u$.
Quindi, se $P$ inizia in $p$ con la pila vuota, dopo aver letto $a$ può andare nello stato $r$ e inserire $u$ in cima alla pila.
Pertanto, $x$ può portare $P$ da $p$ con la pila vuota a $q$ con la pila vuota.

Nel secondo caso, consideriamo le parti $y$ e $z$ di $x$ che $A_{pr}$ e $A_{rq}$ generano, rispettivamente, quindi $x = yz$.
Poichè $A_{pr} \Rightarrow^{*} y$ e $A_{rq} \Rightarrow^{*} z$, in al più $k$ passi, l'ipotesi induttiva ci dice che $y$ può portare $P$ da $p$ a $r$ e $z$ può portare $P$ da $r$ a $q$, con la pila vuota all'inizio e alla fine della computazione.
Quindi $x$ può portare $P$ da $p$ con la pila vuota a $q$ con la pila vuota.
Questo completa la prova del passo induttivo.

\begin{tcolorbox}[colback=yellow!10!white, colframe=yellow!50!black, title=Fatto 2.31]
    Se $x$ può portare $P$ da $p$ con la pila vuota a $q$ con la pila vuota, allora $A_{pq}$ genera $x$.
\end{tcolorbox}

Dimostriamo questo enunciato per induzione sul numero di passi che $P$ impiega per andare da $p$ con la pila vuota a $q$ con la pila vuota sull'input $x$.
\vspace{1em}

\textbf{Base.}
La computazione è in 0 passi.
Se una computazione è in 0 passi, essa inizia e termina nello stesso stato diciamo, $p$. 
Quindi dobbiamo mostrare che $A_{pp} \Rightarrow^{*} x$. 
$P$ non può leggere alcun carattere in 0 passi, quindi $x = \varepsilon$. 
Per costruzione, $G$ ha la regola $A_{pp} \rightarrow \varepsilon$, perciò la base è dimostrata.

\textbf{Passo induttivo.}
Assumiamo l'enunciato vero per le computazioni di lunghezza al più $k$, dove $k \geq 0$, e dimostriamo che è vero per le computazioni di lunghezza $k + 1$.
Supponiamo che $P$ abbia una computazione dove $x$ porta da $p$ a $q$ con pila vuota in $k + 1$ passi. 
O la pila è vuota solo all'inizio e alla fine di questa computazione oppure essa si svuota anche altrove.
Nel primo caso, il simbolo che è stato inserito nella prima mossa deve essere lo stesso simbolo che è stato rimosso nell'ultima mossa. 
Chiamiamo $u$ questo simbolo. 

Sia $a$ il simbolo di input letto nella prima mossa, $b$ il
simbolo di input letto nell'ultima mossa, $r$ lo stato dopo la prima mossa
ed $s$ lo stato prima dell'ultima mossa. Allora $\delta(p,a,\varepsilon)$ contiene $(r, u)$ e
$\delta(s,b,u)$ contiene $(q,\varepsilon)$ e quindi la regola $A_{pq} \rightarrow aA_{rs}b$ è in $G$.
Sia $y$ la parte di $x$ senza $a$ e $b$, quindi $x = ayb$. 
L'input $y$ può portare $P$ da $r$ a $s$ senza toccare il simbolo $u$ che è sulla cima della pila e quindi $P$ può
andare da $r$ con una pila vuota a $s$ con una pila vuota sull'input $y$. 
Abbiamo eliminato il primo e l'ultimo passo dei $k +1$ passi nella computazione iniziale
su $x$, perciò la computazione su $y$ ha $(k + 1) - 2 = k - 1$ passi.
Quindi l'ipotesi induttiva ci dice che $A_{rs} \Rightarrow^{*} y$. 
Allora $A_{pq} \Rightarrow^{*} x$.

Nel secondo caso, sia $r$ uno stato in cui la pila si svuota oltre che all'inizio o alla fine della computazione su $x$. 
Allora le parti della computazione da $p$ a $r$ e da $r$ a $q$ contengono al più $k$ passi. 
Sia $y$ l'input letto nella prima parte e sia $z$ l'input letto nella seconda parte. 
L'ipotesi induttiva ci dice che $A_{pr} \Rightarrow^{*} y$ e $A_{rq} \Rightarrow^{*} z$. 
Poiché la regola $A_{pq} \rightarrow A_{pr}A_{rq}$ è in $G$, $A_{pq} \Rightarrow^{*} x$ e
la dimostrazione è completa.

Questo completa la prova del Lemma 2.27 e del Teorema 2.20.
\vspace{1em}

Abbiamo appena dimostrato che gli automi a pila riconoscono la classe dei lunguaggi context-free.
Questa dimostrazione ci consente di stabilire una relazione tra i linguaggi regolari e i linguaggi context-free.
Poichè ogni linguaggio regolare è riconosciuto da un automa finito e ogni automa finito è automaticamente un automa a pila che semplicemente ignora la sua pila, ora sappiamo che ogni linguaggio regolare è anche context-free. 

\begin{tcolorbox}[colback=blue!10!white, colframe=blue!50!black, title=Corollario 2.32]
    Ogni linguaggio regolare è anche context-free.
\end{tcolorbox}

\begin{figure}[H]
    \centering
    \includegraphics[width=0.4\textwidth]{Immagini/50.png}
    \label{fig:your_image}
\end{figure}

\subsection{Linguaggi non context-free}

In questa sezione presentiamo una tecnica per dimostrare che alcuni linguaggi non sono context-free.
Presentiamo un pumping lemma per i linguaggi context-free.
Esso stabilisce che per ogni linguaggio context-free esiste un particolare valore chiamato la \textbf{lunghezza del pumping} tale che tutte le stringhe più lunghe nel linguaggio possono essere "iterate".
Questa volta il significato di iterazione è un po' più complesso.
Significa che la stringa può essere divisa in cinque parti in modo tale che la seconda e la quarta parte possono insieme essere replicate un numero qualsiasi di volte ottenendo una stringa che appartiene ancora al linguaggio.

\paragraph{Il pumping lemma per i linguaggi context-free Teorema 2.34 orale(24-30)}
\text{ }

\begin{tcolorbox}[colback=yellow!10!white, colframe=yellow!50!black, title=Il pumping lemma per i linguaggi context-free]
    Se $A$ è un linguaggio context-free, allora esiste un numero $p$ (lunghezza del pumping) tale che, se $s$ è una qualsiasi stringa in $A$ di lunghezza almeno $p$, allora $s$ può essere divisa in cinque parti $s = uvxyz$ che soddisfano le condizioni
    \begin{enumerate}
        \item per ogni $i \geq 0$, $uv^ixy^iz \in A$,
        \item $|vy| > 0$ e
        \item $|vxy| \leq p$.
    \end{enumerate}
    Quando $s$ è divisa in $uvxyz$, la condizione 2 afferma che $v$ oppure $y$ non è la stringa vuota.
    Altrimenti il teorema sarebbe banalmente vero.
    La condizione 3 afferma che le parti $v,x,y$ insieme hanno lunghezza al più $p$.
    Questa condizione tecnica a volte è utile per dimostrare che alcuni linguaggi non sono context-free.
\end{tcolorbox}

\textbf{IDEA.}
Sia $A$ un CFL e sia $G$ una CFG che lo genera.
Dobbiamo mostrare che ogni stringa sufficientemente lunga $s$ in $A$ può essere iterata e restare in $A$.
L'idea dietro questa strategia è semplice.

Sia $s$ una stringa molto più lunga in $A$.
(Chiariremo in seguito cosa intendiamo con "molto lunga").
Poichè $s$ è in $A$, essa è derivabile da $G$ e quindi ha un albero sintattico.
L'albero sintattico per $s$ deve essere molto alto perchè $s$ è molto lunga.
Cioè l'albero sintattico deve contenere un cammino lungo dalla variabile alla radice dell'albero a uno dei simboli terminali su una foglia.
Per il principio della piccionaia, qualche simbolo di variabile $R$ si deve ripetere in questo cammino lungo.
Come mostra la figura seguente, questa ripetizione ci permette di sostituire il sottoalbero sotto la seconda occorrenza di $R$ con il sottoalbero sotto la prima occorrenza di $R$ e ottenere ancora un albero sintattico consentito.
Pertanto, possiamo dividere $s$ in cinque parti $uvxyz$, come indicato nella figura, e possiamo replicare il secondo e quarto pezzo e ottenere una stringa ancora nel linguaggio.
In altre parola, $uv^ixy^iz$ è ancora in $A$ per ogni $i \geq 0$.

\begin{figure}[H]
    \centering
    \includegraphics[width=0.3\textwidth]{Immagini/51.png}
    \label{fig:pumping_lemma_example}
\end{figure}

\textbf{DIMOSTRAZIONE.}
Sia $G$ una CFG per il CFL $A$.
Sia $b$ il massimo numero di simboli nel lato destro di una regola (assumiamo che sia almeno 2).
Sappiamo che, in ogni albero sintattico costruito usando questa grammatica, un nodo non può avere più di $b$ figli.
Quindi, se l'altezza dell'albero sintattico è al più $h$, la lunghezza della stringa generata è al più $b^h$.
Viceversa, se una stringa generata ha lunghezza maggiore o uguale a $b^{h+1}$, ciascuno dei suoi alberi sintattici deve avere un'altezza maggiore o uguale a $h+1$.

Sia $|V|$ il numero delle variabili in $G$. Poniamo $p$, la lunghezza del pumping, uguale a $b^{|V|+1}$.
Ora se $s$ è una stringa in $A$ e la sua lunghezza è maggiore o uguale a $p$, il suo albero sintattico deve avere altezza maggiore o uguale a $|V|+1$, poichè $b^{|V|+1} \geq b^{|V|}+1$.

Per vedere come iterare una tale stringa $s$, sia $\tau$ uno dei suoi alberi sintattici.
Se $s$ ha diversi alberi sintattici, scegliamo un albero sintattico $\tau$ che abbia il più piccolo numero di nodi.
Sappiamo che $\tau$ deve avere altezza maggiore o uguale a $|V|+1$, quindi il suo cammino più lungo dalla radice a una foglia ha lunghezza almeno $|V|+1$.
Questo cammino ha almeno $|V|+2$ nodi; uno etichettato da un terminale, gli altri etichettati da variabili. Quindi questo cammino ha almeno $|V| +1$ variabili.
Poiche $G$ ha solo $|V|$ variabili, qualche variabile $R$ è presente più di una volta su questo cammino.
Per un utilizzo successivo, scegliamo $R$ in modo che sia una variabile che si ripete tra le $|V|+1$ variabili più in basso su questo cammino.

Dividiamo $s$ in $uvxyz$.
Ogni occorrenza di $R$ ha un sottoalbero sotto essa che genera una parte della stringa $s$.
L'occorrenza più in alto di $R$ ha un sottoalbero più grande e genera $vxy$, mentre l'occorrenza più in basso genera solo $x$ con un sottoalbero più piccolo.
Entrambi questi sottoalberi sono generati dalla stessa variabile, quindi possiamo sostituire l'uno con l'altro e ottenere ancora un albero sintattico corretto.
Sostituire ripetutamente il più piccolo con il più grande fornisce gli alberi sintattici per le stringhe $uv^ixy^iz$ per ogni $i > 1$.
Sostituire il più grande con il più piccolo genera la stringa $uxz$.
Questo dimostra la condizione 1 del lemma.

Ora veniamo alle condizioni 2 e 3.

Per ottenere la condizione 2, dobbiamo essere sicuri che $v$ e $y$ non sono entrambe $\varepsilon$.
Se lo fossero, l'albero sintattico ottenuto sostituendo il più piccolo sottoalbero al più grande avrebbe meno nodi di $\tau$ e genererebbe ancora $s$.
Questo non è possibile perchè abbiamo scelto $\tau$ in modo che sia un albero sintattico per $s$ con il più piccolo numero di nodi.
Questa è la ragione per aver selezionato così $\tau$.

Per ottenere la condizione 3, dobbiamo essere sicuri che $vxy$ ha lunghezza al più $p$.
Nell'albero sintattico per $s$ l'occorrenza più in alto di $R$ genera $vxy$.
Abbiamo scelto $R$ in modo che entrambe le occorrenze di essa cadano nelle $|V|+1$ variabili più in basso del cammino e abbiamo scelto il più lungo cammino nell'albero sintattico, in modo che il sottoalbero in cui $R$ genera $vxy$ sia alto al più $|V|+1$.
Un albero con questa altezza può generare una stringa di lunghezza al più $b^{|V|+1} = p$.
\vspace{2em}

\href{https://chatgpt.com/share/6759d8be-13c4-8011-9aad-d7e3e0a8a898}{\textcolor{blue}{Link a chatgpt che lo spiega meglio.}}
\vspace{1em}

Scritto più precisamente:
\vspace{1em}

\textbf{PUMPING LEMMA:}
$$
A\text{ }C.F \Rightarrow \exists p \in \mathbb{N} \text{ } \forall s \in A(|s| \geq p \Rightarrow \exists u,v,x,y,z \text{ } t.c(s=uvxyz\text{ } \wedge \text{ }|vy| > 0\text{ } \wedge \text{ }|vxy| \leq p \wedge \forall i \in \mathbb{N}.\text{ }uv^ixy^iz \in A))
$$
\vspace{1em}

\textbf{CONTRAPPOSTA:}
$$
\forall p \in \mathbb{N} \text{ } \exists s \in A(|s| \geq p\text{ } \wedge \forall u,v,x,y,z (s\neq uvxyz\text{ } \vee \text{ }|vy| = 0\text{ } \vee \text{ }|vxy| \geq p\text{ } \vee \text{ } \exists \text{ } i \in \mathbb{N}.\text{ }uv^ixy^iz \notin A)) \Rightarrow A \text{ is not C.F.}.
$$
\vspace{1em}

\textbf{EQUIVALENTEMENTE:}
$$
\forall p \in \mathbb{N} \text{ } \exists s \in A(|s| \geq p\text{ } \wedge \forall u,v,x,y,z ((s = uvxyz\text{ } \wedge \text{ }|vy| > 0\text{ } \vee \text{ }|vxy| \leq p) \Rightarrow \exists i \in \mathbb{N}. \text{ } uv^ixy^iz \notin A)) \Rightarrow A \text{ is not C.F.}.
$$
\vspace{1em}

\begin{figure}[H]
    \centering
    \includegraphics[width=0.4\textwidth]{Immagini/52.png}
    \includegraphics[width=0.4\textwidth]{Immagini/53.png}
    \includegraphics[width=0.4\textwidth]{Immagini/54.png}
    \label{fig:pump_example}
\end{figure}
\newpage

\subsection{Grammatica generativa}

\begin{tcolorbox}[colback=blue!10!white, colframe=blue!50!black, title=Definizione 2.35]
    Una grammatica è una quadrupla $G = (V,\Sigma,R,S)$ dove
    \begin{enumerate}
        \item $V$ è un insieme finito di variabili.
        \item $\Sigma$ è un insieme finito di simboli terminali disgiunto da $V$.
        \item $R$ è un insieme finito di regole di produzione, con $R \subseteq (V \cup \Sigma)^+ \times (V \cup \Sigma)^*$.
        \item $S \in V$ è la variabile iniziale.
    \end{enumerate}
\end{tcolorbox}
Notazionalmente, regole con la stessa parte sinistra sono raggruppate, cioè se ($\alpha,\beta_1$),...,($\alpha,\beta_n$) sono regole in $R$ con prima componente $\alpha$, scriveremo $\alpha \rightarrow \beta_1 \mid ... \mid \beta_n$.

Se $u,v,\alpha,\beta$ sono stringhe di variabili e terminali, ed ($\alpha,\beta$) è una regola della grammatica, diremo che $u\alpha v$  \textit{\textbf{produce}} $u\beta v$ in un passo, e scriveremo $u\alpha v \Rightarrow u\beta v$.

Diremo che $u$ \textit{\textbf{deriva}} $v$, scritto $u \Rightarrow^* v$, se $u = v$ oppure se esiste una sequenza $u_1, u_2, ..., u_k$ con ($k \geq 0$) tale che $u \Rightarrow u_1 \Rightarrow ... \Rightarrow u_k \Rightarrow v$.

Il \textit{\textbf{linguaggio generato dalla grammatica}} $G$ è $L(G) = \{w \in \Sigma^* \mid S \Rightarrow^* w\}$.

\subsubsection{Gerarchia di Chomsky}
In base alla forma delle regole della grammatica, possiamo distinguere i seguenti tipi:

\begin{table}[H]
    \centering
    \renewcommand{\arraystretch}{1.5}
    \begin{tabular}{|c|c|c|}
        \hline
        \textbf{\textit{Tipo della grammatica}} & \textbf{\textit{Forma delle Regole}} & \textbf{\textit{Linguaggio prodotto}} \\
        \hline
        Tipo 0 & Nessun vincolo & Ricorsivamente enumerabile \\
        \hline
        Tipo 1 & \makecell{$\alpha ::= \beta$ \\ per $0 < \mid \alpha \mid \leq \mid \beta \mid$} & Context-sensitive \\
        \hline
        Tipo 2 & \makecell{$A ::= \beta$ \\ per $A \in V, \beta \in (\Sigma \cup V)^*$} & Context-free \\
        \hline
        Tipo 3 & \makecell{$A ::= \beta$ \\ per $A \in V, \beta \in (\Sigma^* \cup \Sigma^* V \bigcup V\Sigma^*)$} & Regolare \\
        \hline
    \end{tabular}
    \label{tab:example_table}
\end{table}

\begin{tcolorbox}[colback=red!10!white, colframe=red!50!black, title=\textbf{NB!}]
    Per i linguaggi che non contengono $\varepsilon$, si può vedere facilmente che ogni grammatica di tipo $i$ è anche la grammatica di tipo $j$, per ogni $j < i$.
\end{tcolorbox}

\begin{tcolorbox}[colback=red!10!white, colframe=red!50!black, title=\textbf{OSS}]
    Il nome <<context-sensitive>> deriva dal fatto che ogni grammatica di tipo 1 può essere trasformata in una grammatica equivalente in cui tutte le regole hanno la forma
    $$
    (\alpha_1 A \alpha_2, \alpha_1 \beta \alpha_2) \text{ per } A \in V, \alpha_1, \alpha_2 \in (\Sigma \cup V)^* \text{ e } \beta \in (\Sigma \cup V)^+.
    $$
    I una CFG, $\alpha_1 = \alpha_2 = \varepsilon$ e quindi $A$ può essere sempre sostituita con $\beta$;
    al contrario, in una CSG questo è possibile sono nel contesto $\alpha_1[-] \alpha_2$.
\end{tcolorbox}

\newpage
\section{La Tesi di Church-Turing}
\subsection{Macchine di Turing}
Introduciamo ora un modello molto più potente, proposto inizialmente da Alan Turing nel 1936, chamato \textbf{macchina di Turing}.
Simile ad un automa finito, ma con una memoria illimitata e senza restrizioni, una macchina di Turing è un modello molto più preciso di un computer.
Una macchina di Turing utilizza un nastro infinito come propria memoria illimitata.
Ha una testina che è in grado di leggere e scrivere simboli ed è libera di muoversi lungo il nastro.
Inizialmente il nastro contiene solo la stringa di input e tutto il resto è vuoto.
Se la macchina dece memorizzare un'informazione, la può scrivere sul nastro.
Gli output "accetta" e "rifiuta" sono ottenuti occupando appositi stati di accettazione e rifiuto.
Se non raggiunge uno stato di accettazione o di rifiuto, la macchina andrà avanti per sempre, senza mai fermarsi.

\begin{figure}[H]
    \centering
    \includegraphics[width=0.5\textwidth]{Immagini/56.png}
    \caption{Schema di una macchina di Turing}
    \label{fig:turing_machine_example}
\end{figure}

L'elenco seguente riassume le differenze tra automi finiti e macchine di Turing.

\begin{enumerate}
    \item Una macchina di Turing può sia scrivere che leggere sul nastro.
    \item La testina di lettura-scrittura può muoversi sia verso sinistra che verso destra.
    \item Il nastro è infinito.
    \item Gli stati speciali di accettazione e rifiuto hanno effetto immediato.
\end{enumerate}

Introduciamo una macchina di Turing $M_1$ per testare l'appartenenza del linguaggio 

$B = \{w\#w\mid w\in \{0,1\}^* \}$.
Vogliamo che $M_1$ accetti se l'input è un elemento di $B$ e che rifiuti altrimenti.
Il vostro obiettivo è quello di determinare se l'input comprende due stringhe identiche separate da un simbolo $\#$.
L'input è troppo lungo per voi per ricordarlo tutto, ma vi è permesso di muovervi avanti e indietro lungo l'input e porre dei contrassegni.
La strategia ovvia è quella di muoversi a zig-zag in maniera simmetrica attorno al simbolo $\#$ e stabilire se gli elementi di ugial indice su i due lati corrispondono.
Per tenere traccia dei simboli già controllati, $M_1$ barra ogni simbolo già esaminato.
Se ha barrato tutti i simboli, significa che tutti i simboli corrispondevano ed $M_1$ va in uno stato di accettazione.
Se si accorge di una mancata corrispondenza, $M_1$ entra in uno stato di rifiuto.
In pratica l0algoritmo di $M_1$ è il seguente:

$M_1 = $ "Sulla stringa di input $w$:
\begin{enumerate}
    \item Si muove a zig-zag lungo il nastro, raggiungendo posizioni corrispondenti su i due lati del simbolo $\#$, per controllare se queste posizioni contengono lo stesso simbolo. In caso negativo, o nel caso in cui non si trovi il simbolo $\#$, rifiuta. Barra gli elementi già controllati.
    \item Quando tutti gli elementi a sinistra del simbolo $\#$ sono stati barrati, verifica la presenza di eventuali simboli rimanenti a destra di $\#$. Se rimane qualche elemento, allora rifiuta, altrimenti accetta.
\end{enumerate}

La figura seguente contiene varie istantanee non consecutive del nastro di $M_1$ una volta avviata sull'input $011000\#011000$.

\begin{figure}[H]
    \centering
    \includegraphics[width=0.3\textwidth]{Immagini/57.png}
    \label{fig:turing_machine_example}
\end{figure}

\paragraph{Definizione formale di macchina di Turing}
\label{def:turing_machine}
\text{ }

Il cuore della definizione di una macchina di Turing è la funzione di transizione $\delta$ perchè essa ci dice come la macchina effettua un passo.

Per una macchina di Turing prende la forma $\delta: Q \times \Gamma \rightarrow Q \times \Gamma \times \{L,R\}$.
Cioè, quando la macchina occupa un certo stato $q$ e la testina punta alla casella del nastro contenete un simbolo $a$, e se $\delta(q,a) = (r,b,L)$, la macchina scrive il simbolo $b$ al posto di $a$, e passa nello stato $r$.
La terza componente è $L$ oppure $R$ e indica se la testina si muove a sinistra o a destra dopo la scrittura.

\begin{tcolorbox}[colback=green!10!white, colframe=green!50!black, title=Definizione 3.3]
    Una \textbf{macchina di Turing} è una 7-upla, $M = (Q, \Sigma, \Gamma, \delta, q_0, q_{accept}, q_{reject})$, dove $Q, \Sigma, \Gamma$ sono insiemi finiti e
    \begin{enumerate}
        \item $Q$ è l'insieme finito di stati,
        \item $\Sigma$ è l'alfabeto di input, non contiene il simbolo di \textbf{blank} $\sqcup$,
        \item $\Gamma$ è l'alfabeto della nastro, contiene $\Sigma$ e il simbolo di blank $\sqcup$,
        \item $\delta: Q \times \Gamma \rightarrow Q \times \Gamma \times \{L,R\}$ è la funzione di transizione,
        \item $q_0 \in Q$ è lo stato iniziale,
        \item $q_{accept} \in Q$ è lo stato di accettazione,
        \item $q_{reject} \in Q$ è lo stato di rifiuto, $q_{reject} \neq q_{accept}$.
    \end{enumerate}
\end{tcolorbox}

Una macchina di Turing computa nel seguente modo.
Inizialmente riceve il suo input $w = w_1w_2...w_n \in \Sigma^*$ sulle $n$ celle più a sinistra del nastro, mentre il resto del nastro è vuoto (simboli blank).
La testina parte dalla posizione più a sinistra del nastro.
Si noti che $\Sigma$ non contiene il simbolo di blank $\sqcup$, in tal modo il primo simbolo blank che compare sul nastro segna la fine dell'input.
Se $M$ tenta di spostare la testina a sinistra quando si trova all'estremità sinistra del nastro, allora la testina rimane ferma.
La computazione continua fino a quando la macchina raggiunge uno stato di accettazione o di rifiuto.
Se nessuno dei due stati è raggiunto, la macchina continua a computare all'infinito.
Durante la computazione di una macchina di Turing, si verificano cambiamenti dello stato corrente, del contenuto corrente del nastro, e della posizione corrente della testina.
Un'impostazione di questi tre elementi è chiamata \textbf{configurazione} della macchina di Turing.
Una configurazione è rappresentata come $uqv$, dove $u$ è la stringa di simboli a sinistra della testina, $q$ è lo stato corrente, e $v$ è la stringa di simboli a destra della testina.
La configurazione iniziale è $q_0w$, dove $w$ è l'input.
Dopo l'ultimo simbolo di $w$, il nastro contiene solo simboli blank.

Per esempio, $1011q_701111$ indica che la testina è in stato $q_7$, il nastro contiene $1011$ a sinistra della testina, e $1111$ a destra della testina.

\begin{figure}[H]
    \centering
    \includegraphics[width=0.5\textwidth]{Immagini/58.png}
    \caption{Esempio di macchina di Turing}
    \label{fig:turing_machine_example2}
\end{figure}

Supponiamo di avere $a,b,c \in \Gamma$ così come $u,v$ in $\Gamma^*$ e gli stati $q_i$ e $q_j$.
In tal caso $uaq_ibv$ e $uq_jacv$ sono due configurazioni. diciamo che 
$$
uaq_ibv \text{ produce } uq_jacv
$$
se $\delta(q_i,b) = (q_j,c,L)$.

Casi particolari si verificano quando la testina è alla fine del nastro.
Per l'estremità sinistra, la configurazione $q_ibv$ produce $q_jcv$ se $\delta(q_i,b) = (q_j,c,L)$.

Una macchina di Turing $M$ \textbf{accetta} l'input $w$ se esiste una sequenza di configurazioni $C_1,C_2,...,C_k$ tale che:
\begin{enumerate}
    \item $C_1$ è la configurazione iniziale di $M$ su $w$,
    \item per ogni $i$, $M$ va dalla configurazione $C_i$ alla configurazione $C_{i+1}$,
    \item $C_k$ è una configurazione accettante.
\end{enumerate}

L'insieme di stringhe che $M$ accetta rappresenta il \textbf{linguaggio di M}, o il \textbf{linguaggio riconosciuto da M}, denotato con $L(M)$.

\begin{tcolorbox}[colback=yellow!10!white, colframe=yellow!50!black, title=Definizione 3.5]
    Un linguaggio è \textbf{Turing-riconoscibile} se esiste una macchina di Turing che lo riconosce.
\end{tcolorbox}

Quando attiviamo una macchina di Turing su un input, essa può accettare, rifiutare o andare avanti all'infinito.
Una macchina di Turing può non accettare sia perchè si ferma in uno stato di rifiuto, sia perchè non si ferma affatto.
A volte distinguere una macchina che è entrata in un ciclo infinito da una che sta ancora computando è difficile.
Per questo motivo preferiamo macchine di Turing che si fermano su ogni input; tali macchine non ciclano mai.
Una macchina è detta \textbf{decisore} perchè prende in ogni caso una decisione.
Diciamo che un decisore \textbf{decide} un certo linguaggio se riconosce tale linguaggio.

\begin{tcolorbox}[colback=green!10!white, colframe=green!50!black, title=Definizione 3.6]
    Un linguaggio è \textbf{Turing-decidibile} se esiste una macchina di Turing che lo decide.
\end{tcolorbox}

\begin{figure}[H]
    \centering
    \includegraphics[width=0.3\textwidth]{Immagini/59.png}
    \includegraphics[width=0.3\textwidth]{Immagini/60.png}
    \includegraphics[width=0.3\textwidth]{Immagini/61.png}
    \label{fig:turing_machine_example2}
\end{figure}


\begin{figure}[H]
    \centering
    \includegraphics[width=0.35\textwidth]{Immagini/62.png}
    \includegraphics[width=0.35\textwidth]{Immagini/63.png}
    \includegraphics[width=0.35\textwidth]{Immagini/64.png}
    \label{fig:turing_machine_example2}
\end{figure}

\subsection{Varianti delle Macchine di Turing}
\paragraph{Macchine di Turing multinastro}
\label{def:multi_turing_machine}
\text{ }

Una macchina di Turing multinastro è una macchina di Turing con più nastri, ognuno con la propria testina.
Ogni testina può muoversi indipendentemente dalle altre.
Inizialmente l'input su trova sul nastro 1, mentre gli altri nastri sono vuoti.
La funzione di transizione per una macchina di Turing multinastro è simile a quella di una macchina di Turing, ma con una piccola differenza.
La funzione di transizione per una macchina di Turing multinastro è della forma 
$$
\delta: Q \times \Gamma^k \rightarrow Q \times \Gamma^k \times \{L,R,S\}^k
$$,
dove $k$ è il numero di nastri. L'espressione
$$
\delta(q_i,a_1a_2...a_k) = (q_j,b_1b_2...b_k,L,R,...,L)
$$
significa che, se la macchina si trova nello stato $q_i$ e le testine da $1$ a $k$ leggono i simboli da $a_1$ a $a_k$, allora la macchina va nello stato $q_j$, scrive i simboli da $b_1$ a $b_k$ sui nastri, e muove le testine a sinistra o a destra o le lascia ferme.
Le macchine di Turing multinastro sembrano più potenti delle macchine di Turing ordinarie, ma possiamo dimostrare che sono equivalenti in potenza di calcolo.
Ricordiamo che due macchine sono equivalenti se riconoscono lo stesso linguaggio.

\newpage
\paragraph{Teorema 3.13 orale(24-30)}
\label{teorema-3.13}
\text{ }

\begin{tcolorbox}[colback=blue!10!white, colframe=blue!50!black, title=Teorema 3.13 (orale 24-30)]
    Per ogni macchina di Turing multinastro esiste una macchina di Turing a nastro singolo equivalente.
\end{tcolorbox}

\textbf{DIMOSTRAZIONE.}
Mostriamo come convertire una macchina di Turing multinastro $M$ in una TM $S$ equivalente a nastro singolo.
L'idea chiave è quella di mostrare come simulare $M$ con $S$.
Supponiamo che $M$ abbia $k$ nastri.
Allora $S$ simula l'effetto di $k$ nastri memorizzando le loro informazioni sul singolo nastro.
Essa utilizza il nuovo simbolo $\#$ per separare le informazioni dei nastri.
Oltre al contenuto dei nastri, $S$ deve tenere traccia delle posizioni delle testine.
Lo fa scrivendo un simbolo con un punto sopra per contrassegnare la posizione in cui si troverebbe la testina di un nastro.
Pensate a questi come nastri e testine "virtuali".
Come in precedenza, i simboli del nastro "puntati" sono semplicemente simboli nuovi che sono stati aggiunti all'alfabeto del nastro.

\begin{figure}[H]
    \centering
    \includegraphics[width=0.4\textwidth]{Immagini/65.png}
    \label{fig:turing_machine_example2}
\end{figure}

$S = $ "Sulla stringa di input $w$:
\begin{enumerate}
    \item Inizialmente $S$ mette il suo nastro nel formato che rappresenta tutti i $k$ nastri di $M$. Il nastro formattato contiene $$\#\dot{w_{1}}w_2...w_n\#\dot{\sqcup}\#\dot{\sqcup}\#...\#.$$
    \item Per simulare ogni singola mossa, $S$ scansiona il suo nastro dal primo $\#$, che segna l'estremità sinistra, al $(k+1)$mo $\#$, che segna l'estremità destra, per determinare i simboli puntati dalle testine virtuali. Successivamente $S$ fa un secondo passggio per aggiornare i nastri in accordo alla funzione di transizione $M$.
    \item Se in qualsiasi momento $S$ sposta una delle testine virtuali a destra su un $\#$, questa azione significa che $M$ ha spostato la testina corrispondente sulla parte di nastro vuota non letta in precendenza. Quindi $S$ scrive un simbolo blank in questa cella del nastro e sposta il contenuto del nastro, da questa cella fino al simbolo $\#$ più a destra, di una unità a destra. Poi prosegue la simulazione come prima.
\end{enumerate}

\paragraph{Corollario 3.15}
\label{cor:da MdT multitape a single tape}
\text{ }

\begin{tcolorbox}[colback=blue!10!white, colframe=blue!50!black, title=Corollario 3.15]
    Un linguaggio è Turing-riconoscibile se e solo se è riconosciuto da una macchina di Turing multinastro.
\end{tcolorbox}

\paragraph{Macchine di Turing non deterministiche}
\label{def:non_deterministic_turing_machine}
\text{ }

Una macchina di Turing non deterministica è una macchina di Turing che può avere più di una mossa lecita in una certa situazione.
La funzione di transizione di una macchina di Turing non deterministica è della forma
$$
\delta: Q \times \Gamma \rightarrow P(Q \times \Gamma \times \{L,R\}).
$$
La computazione di una macchina di Turing non deterministica è un albero i cui rami corrispondono alle diverse scelte possibili previste per la macchina.
Se qualche ramo della computazione porta allo stato di accettazione, allora la macchina accetta l'input.
Ora mostriamo che il non determinismo non influisce sulla potenza di calcolo del modello macchina di Turing.

\paragraph{Teorema 3.16 orale(24-30)}
\label{teorema-3.16}
\text{ }

\begin{tcolorbox}[colback=blue!10!white, colframe=blue!50!black, title=Teorema 3.16 orale(24-30)]
    Per ogni macchina di Turing non deterministica esiste una macchina di Turing deterministica equivalente.
\end{tcolorbox}

\textbf{IDEA.}
Possiamo simulare qualsiasi TM non deterministica $N$ con una TM deterministica $D$.
L'idea di base consiste nel provare tutte le possivili scelte che può fare $N$ durante la computazione non deterministica.
Se $D$ trova lo stato di accettazione su uno qualsiasi di questi rami, allora $D$ accetta.
Altrimenti la simulazione di $D$ non terminerà.
Guardiamo alla computazione di $N$ su un input $w$ come ad un albero.
Ogni ramo dell'albero rappresenta una scelta non deterministica.
Ogni nodo dell'albero è una configurazione di $N$.
La radice dell'albero è la configurazione iniziale di $N$ su $w$.
La TM $D$ esplora questo albero alla ricerca di una configurazione di accettazione.
Eseguire accuratamente questa ricerca è cruciale perchè $D$ potrebbe non visitare l'intero albero.
Un'idea è quella di esplorare l'albero con una DFS(depth-first search).
Se $D$ esplorasse in questo modo, essa potrebbe rimanere per sempre ad esplorare un eventuale cammino infinito e non trovare una configurazione accettante in qualche altro cammino.
Quindi progettiamo $D$ in modo da esplorare l'albero utilizzando una BFS(breadth-first search).
Questa strategia esplora tutti i cammini che terminano alla stessa profondità prima di andare ad esplorare ogni cammino che termina alla profondità successiva.
Questo metodo garantisce che $D$ visita ogni node dell'albero finchè non incontra una cofigurazione di accettazione.
\vspace{1em}

\textbf{DIMOSTRAZIONE.}
La TM $D$ deterministica che simula ha tre nastri.
Dal \hyperref[th:da MdT multitape a single tape]{\textcolor{blue}{Teorema 3.13}}, sappiamo che possiamo simulare una TM multinastro con una TM a nastro singolo.
La macchina $D$ utilizza i suoi tre nastri in maniera particolare, come illustrato nella figura seguente.
Il nastro 1 contiene sempre la striqnga di input e non viene mai modificato.
Il nastro 2 mantiene una copia del nastro di $N$ corrispondente a qualche diramazione della sua computazione non deterministica.
Il nastro 3 tiene traccia della posizione di $D$ nell'albero delle computazioni di $N$.

\begin{figure}[H]
    \centering
    \includegraphics[width=0.5\textwidth]{Immagini/66.png}
    \label{fig:turing_machine_example2}
\end{figure}

Consideriamo in primo luogo la rappresentazione dei dati sul nastro 3.
Ogni nodo dell'albero può avere al massimo $b$ figli, dove $b$ è la dimensione del più grande insieme di scelte possibili date dalla funzione di transizione di $N$.
Ad ogni nodo della struttura assegniamo un indirizzo che è una stringa sull'alfabeto $\Gamma_b = \{1,2,...,b\}$.
Assegniamo l'indirizzo 231 al nodo a cui si arriva partendo dalla radice, spostandosi al suo secondo figlio, spostandosi ancora da tale nodo al suo terzo fuglio ed infine spostandosi al primo figlio di quest'ultimo nodo.
Ogni simbolo della stringa ci dice quale deve essere la scelta successiva, durante la simulazione di un passo in una ramificazione di una computazione non deterministica di $N$.
A volte il simbolo può non corrispondere ad una scelta se sono disponibili troppe poche scelte per configurazione.
In tal caso l'indirizzo non è valio e non corrisponde ad alcun nodo.
Il nastro 3 continene una stringa su $\Gamma_b$.
Essa rappresenta la ramificazione della computazione di $N$ dalla radice al nodo indirizzato da tale stringa, a meno che l'indirizzo non sia valido.
La stringa vuota è l'indirizzo della radice dell'albero.
Siamo ora pronti a descrivere $D$.
\begin{enumerate}
    \item Inizialmente il nastro 1 contiene l'input $w$ e i nastri 2 e 3 sono vuoti.
    \item Copia il nastro 1 sul nastro 2 ed inizializza la stringa sul nastro 3 a $\varepsilon$.
    \item Utilizza il nastro 2 per simulare $N$ con input $w$ su una ramificazione della sua computazione non deterministica. Prima di ogni passo di $N$, consulta il simbolo successivo sul nastro 3 per determinare quale scelta fare tra quelle consentite dalla funzione di transizione di $n$. Se non rimangono più simboli sul nastro 3 o se questa scelta deterministica non è valida, interromper questo cammino andando alla fase 4. Va alla fase 4 anche quando si verifica una configurazione di rifiuto. Se incontra una configurazione di accettazione, allora accetta l'input.
    \item Sostituisce la stringa sul nastro 3 con la stringa successiva rispetto all'ordine sulle stringhe. Simula la ramificazione successiva della computazione di $N$ andando al passo 2.
\end{enumerate}

\begin{figure}[H]
    \centering
    \includegraphics[width=0.5\textwidth]{Immagini/78.png}
    \caption{Esempio di computazione}
    \label{fig:nuova_immagine}
\end{figure}

\paragraph{Corollario 3.18}
\label{cor:da MdT deterministica a non deterministica}
\text{ }

\begin{tcolorbox}[colback=blue!10!white, colframe=blue!50!black, title=Corollario 3.18]
    Un linguaggio è Turing-riconoscibile se e solo se esiste una macchina di Turing non deterministica che lo riconosce.
\end{tcolorbox}

Possiamo modificare la dimostrazione del teorema precedente in modo che se $N$ si ferma sempre su tutte le ramificazioni durante la computazione, $D$ si ferma sempre.
Chiameremo una macchina non deterministica \textbf{decisore} se tutte le ramificazioni si fermano su ogni input.

\paragraph{Corollario}
\label{cor:da MdT deterministica a non deterministica}
\text{ }

\begin{tcolorbox}[colback=blue!10!white, colframe=blue!50!black, title=Corollario]
    Un linguaggio è decidibile se e solo se esiste una macchina di Turing non deterministiche che lo decide.
\end{tcolorbox}

\paragraph{Enumeratori}
\label{def:enumerator}
\text{ }

Come accennato in precedenza, alcune persone usano il termine \textit{linguaggio ricorsivamente enumerabile} per indicare un linguaggio Turing-riconoscibile.
Questo termine deriva da una variante di macchina di Turing chiamata \textbf{\textit{enumeratore}}.
Definito in modo informale, un enumeratore è una macchina di Turing con una stampante collegata.
Ogni volta che la macchina di Turing vuole aggiungere una stringa alla lista, invia la stringa alla stampante.
La figura seguente fornisce uno schema di questo modello.

\begin{figure}[H]
    \centering
    \includegraphics[width=0.3\textwidth]{Immagini/67.png}
    \caption{Schema di un enumeratore}
    \label{fig:enumerator_example}
\end{figure}

Un enumeratore $E$ inizia con un nastro di input vuoto. Se l'enumeratore non si ferma, esso può stampare un elenco infinito di stringhe.
Il linguaggio enumerato da $E$ è la collezione di tutte le stringhe che esso stampa.
Inoltre, $E$ potrebbe generare le stringhe del linguaggio in qualsiasi ordine, anche con ripetizioni.
\newpage
\paragraph{Teorema 3.21 (orale)}
\label{th:da MdT a enumerator}
\text{ }

\begin{tcolorbox}[colback=red!10!white, colframe=red!50!black, title=Teorema 3.21 orale]
    Un linguaggio è Turing-riconoscibile se e solo se esiste un enumeratore che lo enumera.
\end{tcolorbox}

\textbf{DIMOSTRAZIONE.}
Come prima cosa, dimostriamo che se abbiamo un enumeratore $E$ che enumera un linguaggio $A$, allora esiste una TM $M$ che riconosce $A$.
La TM $M$ funziona come segue.
\vspace{1em}

$M = $ "Sulla stringa di input $w$:
\begin{enumerate}
    \item Esegue $E$. Ogni volta che $E$ genera una stringa, la confronta con $w$.
    \item Se $w$ appare nell'output di $E$, accetta".
\end{enumerate}

Chiaramente, $M$ accetta quelle stringhe che compaiono sulla lista di $E'$.

Ora occupiamoci dell'altra direzione.
Se la TM $M$ riconosce un linguaggio $A$, possiamo costruire il seguente enumeratore $E$ per $A$.
Sia $s_1,s_2,s_3,...$ l'elenco di tutte possibili stringhe in $\Sigma^*$.
\vspace{1em}

$E = $ "Ignora l'input.
\begin{enumerate}
    \item Ripete i seguenti passi per $i = 1,2,3,...$ .
    \item Esegue $M$ per $i$ passi su ogni input, $s_1,s_2,...,s_i$.
    \item Se qualche computazione accetta, stampa la corrispondente $s_j$."
\end{enumerate}

Se $M$ accetta una particolare stringa $s$, alla fine $s$ apparirà nella lista generata da $E$.
In realtà essa apparirà nella lista un numero infinito di volte, perchè $M$ ritorna all'inizio su ogni stringa per ogni ripetizione del passo 1.
Questa procedura fa si che $M$ lavori in parallelo su tutte le possibili stringhe di input.


\subsection{La definizione di algoritmo}

Parlando in maniera informale, un algoritmo è un insieme di istruzioni semplici per l'esecuzione di un certo compito.
Anche se gli algoritmi hanno avuto una lunga storia nel campo della matematica, la nozione di algoritmo non è stata formalizzata con precisione fino al ventesimo secolo.
Prima di allora, i matematici avevano una nozione intuitiva di algoritmo, e invocavano questo concetto quando li usavano o li descrivevano.

La seguente storia racconta come una definizione precisa di algoritmo sia stata cruciale nel caso di un importante problema matematico.

\paragraph{I problemi di Hilbert}
\label{def:hilbert_problems}
\text{ }

Nel 1900, il matematico David Hilbert presentò una lista di 23 problemi matematici che avrebbero dovuto guidare la ricerca matematica del ventesimo secolo.
Il decimo problema sulla sua lista riguardava gli algoritmi.

Prima di descrivere il problema, introduciamo brevemente i polinomi.
Un \textbf{polinomio} è una somma di termini, dove ogni \textbf{termine} è il prodotto di alcune variabili ad una costante, chiamata \textbf{coefficiente}.
Per esempio
$$
6 \cdot x \cdot x \cdot x \cdot y \cdot z \cdot z = 6x^3yz^2
$$
è un termine con coefficiente 6, e 
$$
6x^3yz^2 + 3xy^2 - x^3 - 10
$$
è un polinomio con quattro termini sulle variabili $x,y,z$.

In questa discussione, consideriamo come coefficienti solo numeri interi.
Una \textbf{radice} di un polinomio è un'assegnazione di valori alle sue variabili per cui il valore del polinomio è 0.
Questo polinomio ha radice $x = 5$, $y = 3$, e $z = 0$.
Questa è una \textbf{radice intera}, perchè a tutte le variabili sono assegnati valori interi.
\newline

Il decimo problema di Hilbert consiste nell'ideare un algoritmo per verificare se un polinomio abbia o meno una radice intera.
Hilbert non usò il termine "algoritmo", piuttosto "un processo in base al quale esso può essere determinato da un numero finito di operazioni".
Nel modo in cui Hilbert ha formulato questo problema, ha chiesto esplicitamente che fosse "progettato" un algoritmo. 
In tal modo, egli diede per scontato l'esistenza di un tale algoritmo.
Come ora sappiamo, non esiste un algoritmo per questo problema; si tratta di un problema non risolvibile algoritmicamente.
Per poter provare che un algoritmo non esiste è necessario avere una definizione di algoritmo chiara.
Per poter avere progressi sul decimo problema si è dovuto attendere una tale definizione.
La definizione è arrivata nel 1936 da parte di Alonzo Church ed Alan Turing.

Church usò un sistema di notazione detto $\lambda$-calcolo per definire gli algoritmi.
Turing lo fece con le sue "macchine".
E' stato dimostrato che queste due definizioni sono equivalenti.
Questa connessione tra la nozione informale di algoritmo e la relativa definizione precisa è ora chiamata: la \textbf{\textit{Tesi di Church-Turing}}.

La Tesi di Church-Turing fornisce la definizione di algoritmo necessaria per risolvere il decimo problema di Hilbert.
Nel 1970 Yuri Matijasevic, un matematico sovietico, dimostrò che il decimo problema di Hilbert è irrisolvibile.

\begin{figure}[H]
    \centering
    \includegraphics[width=0.5\textwidth]{Immagini/68.png}
    \label{fig:example_image}
\end{figure}

Riscriviamo il decimo problema di Hilbert nella nostra terminologia.
In tal modo introduciamo alcuni temi che analizzeremo nei prossimi capitoli. Sia
$$
D = \{p \mid p \text{ è un polinomio con una radice intera}\}.
$$
Il decimo problema di Hilbert chiede in sostanza se l'insieme $D$ è decidibile.
La risposta è negativa.
Di contro possiamo dimostrare che $D$ è Turing-riconoscibile.
Prima di fare ciò, consideriamo un problema più semplice.
Si tratta di un analogo del decimo problema di Hilbert per i polinomi che hanno un'unica variabile, come per esempio $4x^3 - 2x?2 + x - 7$.
Sia 
$$
D_1 = \{p \mid p \text{ è un polinomio su x avente una radice intera}\}.
$$

Ecco una TM $M_1$ che riconosce $D_1$:
\newline

$M_1 = $ "Sulla stringa di input $\langle p \rangle$: dove $p$ è un polinomio sulla variabile $x$.
\begin{enumerate}
    \item Valuta $p$ con $x$ posta successivamente ai valori $0,1,-1,2,-2,3,-3,...$ . Se in qualsiasi momento la valutazione del polinomio è 0, accetta."
\end{enumerate}

Se $p$ ha una radice intera, $M_1$ ad un certo punto la trova e accetta. 
Se $p$ non ha una radice intera, $M_1$ sarà in esecuzione per sempre. 
Nel caso di più variabili, possiamo presentare una TM $M$ simile che riconosce $D$. 
Ora $M$ considera tutte le possibili impostazioni delle variabili con valori interi.
Sia $M_1$ che $M$ sono riconoscitori, ma non decisori. 
Possiamo convertire $M_1$ in un decisore per $D_1$, perché possiamo calcolare dei limiti all'intervallo di valori in cui possono trovarsi le radici di un polinomio a singola variabile e limitare la ricerca a questo intervallo di valori. 

\paragraph{Terminologia per la descrizione di macchine di Tuning}
\label{def:terminology_turing_machine}
\text{ }

Continuiamo a parlare di macchine di Turing, ma il centro reale del nostro interesse saranno gli algoritmi. 
Cioè, le macchine di Turing servono soltanto come modello preciso per la definizione di algoritmo. 
Stabiliamo ora un formato e la notazione che utilizzeremo per descrivere le macchine di Turing. 
L'input di una macchina di Turing è sempre una stringa. 
Se volessimo fornire in input un oggetto diverso da una stringa, dovremmo prima rappresentare tale oggetto come una stringa.
Le stringhe possono rappresentare facilmente polinomi, grafi, grammatiche, automi e qualsiasi combinazione di questi oggetti.
Una macchina di Turing può essere programmata per decodificarno la rappresentazione in modo che possa essere interpretata nel modo corretto.
La nostra notazione per la codifica di un oggetto $O$ nella sua rappresentazione sotto forma di stringa è $\langle O \rangle$.
Se abbiamo vari oggetti $O_1,O_2,...,O_k$, la loro codifica congiunta è $\langle O_1,O_2,...,O_k \rangle$.

Nel nostro formato, descriviamo gli algoritmi delle macchine di Turing con un segmento di testo tra parentesi. 
Dividiamo l'algoritmo in fasi, ciascuna fase di solito consiste di più passi di calcolo della macchina di Turing.
Indichiamo la struttura a blocchi dell'algoritmo con un'indentazione ulteriore. 
La prima riga dell'algoritmo descrive l'input alla macchina. 
Se la descrizione dell'input è semplicemente $w$, l'input è considerato una stringa. 
Se la descrizione dell'input è la codifica di un oggetto del tipo $\langle A \rangle$, la macchina di Turing dapprima implicitamente controlla se l'input codifica correttamente un oggetto della forma desiderata, rifiutando in caso contrario.

\begin{figure}[H]
    \centering
    \includegraphics[width=0.4\textwidth]{Immagini/69.png}
    \label{fig:example_image1}
\end{figure}

\begin{figure}[H]
    \centering
    \includegraphics[width=0.4\textwidth]{Immagini/70.png}
    \includegraphics[width=0.4\textwidth]{Immagini/71.png}
    \label{fig:example_image2}
\end{figure}
\newpage

\section{Decidibilità}
In questo capitolo iniziamo ad investigare la potenza degli algoritmi nella risoluzione di problemi.
Dimostreremo che alcuni problemi possono essere risolti in ,amiera algoritmica ed altri no.

\subsection{Linguaggi Decidibili}

In questa sezione ci concentriamo su linguaggi che riguardano degli automi delle grammatiche.
Ad esempio, presentiamo un algoritmo che verifica se una stringa è o meno un elemento di un linguaggio context-free(CFL).
Questi linguaggi sono interessanti per vari motivi.
Il primo è che alcuni problemi di questo tipo sono correlati ad applicazioni.
Questo problema di verificare se una CFG genera una stringa è correlato al problema del riconoscimento e compilazione dei programmi in un linguaggio di programmazione.
Il secondo motivo è che altri problemi relativi ad automi e grammatiche non sono decidibili mediante algoritmi.

\subsubsection{Problemi decidibili relativi a linguaggi regolari}
Iniziamo con alcuni problemi computazionali relativi ad automi finiti.
Forniamo alcuni algoritmi per testare se un automa finito accetta o meno una stringa, se il linguaggio di un automa finito è vuoto e se due automi finiti sono equivalenti.
Si noti che abbiamo scelto di rappresentare i problemi computazionali mediante linguaggi.
Ad esempio, il \textbf{problema dell'accettazione} per DFA consiste nel testare se un particolare automa finito deterministico accetta una data stringa, può essere espresso come un linguaggio, $A_{DFA}$.
Questo linguaggio contiene le codifiche di tutti i DFA con le stringhe che essi accettano.
Definiamo 
$$
A_{DFA} = \{\langle B,w \rangle \mid B \text{ è un DFA che accetta la stringa di input } w \}.
$$
Il problema di verificare se un DFA $B$ accetta l'input $w$ coincide con il problema di verificare se $\langle B,w \rangle$ è un elemento del linguaggio $A_{DFA}$.
Mostrare che il linguaggio è decidibile equivale a mostrare che il problema computazionale è decidibile.
Nel seguente teorema si mostra che $A_{DFA}$ è decidibile.

\paragraph{Teorema 4.1(orale)}
\label{teorema-4.1}
$A_{DFA}$ è un linguaggio decidibile.
\vspace{1em}
\text{}
\newline
\hbox{\textbf{IDEA.}}
Abbiamo semplicemente bisogno di presentare una TM $M$ che decide $A_{DFA}$.
\vspace{1em}
\text{}
\newline
$M = "$Su input $\langle B,w \rangle$, dove $B$ è un DFA e $w$ è una stringa:
\begin{enumerate}
    \item Simula $B$ su input $w$.
    \item Se la simulazione termina in uno stato di accettazione, accetta. Se non termina in uno stato di accettazione rifiuta.$"$
\end{enumerate}
\vspace{1em}
\text{}
\newline
\hbox{\textbf{DIMOSTRAZIONE.}} 
In primo luogo, esaminiamo l'input $\langle B,w \rangle$.
Si tratta di una rappresentazione di un DFA $B$ e di una stringa $w$.
Una rappresentazione ragionevole di $B$ è semplicemente una lista delle sue cinque componenti: $Q,\Sigma,\delta,q_0,F$.
Quando $M$ riceve il suo input, per prima cosa verifica se esso rappresenta correttamente un DFA $B$ ed una stringa $w$.
In caso contrario, $M$ rifiuta.
Poi $M$ effettua direttamente la simulazione.
Tiene traccia dello stato corrente di $B$ e della posizione corrente di $B$ nell'input $w$ scrivendo queste informazioni sul suo nastro.
Inizialmente lo stato corrente di $B$ è $q_0$ e la posizione corrente di $B$ nell'input $w$ è la prima posizione più a sinistra di $w$.
Gli stati e le posizioni vengono aggiornati in base alla funzione di transizione specificata $\delta$.
Quando $M$ termina l'elaborazione dell'ultimo simbolo di $w$, $M$ accetta l'input se $B$ è in uno stato di accettazione; $M$ rifiuta l'input se $B$ non è in uno stato di accettazione.
\vspace{1em}
\text{}
\newline
Possiamo dimostrare un teorema simile per automi a stati finiti non deterministici. Dato 
$$
A_{NFA} = \{\langle B,w \rangle \mid B \text{ è un NFA che accetta la stringa di input } w \}.
$$
Mostriamo che $A_{NFA}$ è decidibile.
\newpage
\paragraph{Teorema 4.2(orale)}
\label{teorema-4.2}
$A_{NFA}$ è un linguaggio decidibile.
\vspace{1em}
\text{}
\newline
\hbox{\textbf{DIMOSTRAZIONE.}}
Presentiamo una TM $N$ che decide $A_{NFA}$.
Potremmo progettare $N$ in modo che operi come $M$, simulando un NFA invece di un DFA.
Invece, procederemo in modo diverso per illustrare una nuova idea: $N$ usa $M$ come una sottoprocedura.
Poichè $M$ è progettata per funzionare con un DFA, $N$ prima converte l'NFA che riceve come input in un DFA e poi lo passa ad $M$.
\vspace{1em}
\text{}
\newline
$N = $ "Sulla stringa di input $\langle B,w \rangle$, dove $B$ è un NFA e $w$ è una stringa:
\begin{enumerate}
    \item Converte l'NFA $B$ in un DFA equivalente $C$, usando la procedura di conversione data nel \hyperref[teorema-1.39]{\textcolor{blue}{Teorema 1.39}}.
    \item Esegue la TM $M$ del \hyperref[teorema-4.1]{\textcolor{blue}{Teorema 4.1}} su input $\langle C,w \rangle$.
    \item Se $M$ accetta, accetta; altrimenti rifiuta."
\end{enumerate}

L'esecuzione della TM $M$ nella fase 2 significa incorporare $M$ nella progettazione di $M$ come sottoprocedura.
\vspace{1em}

In modo analogo, possiamo determinare se un'espressione regolare genera una data stringa. 
\newline
Consideriamo
$$
A_{REX} = \{\langle R,w \rangle \mid R \text{ è un'espressione regolare che genera la stringa di input } w \}.
$$

\paragraph{Teorema 4.3(orale)}
\label{teorema-4.3}
$A_{REX}$ è un linguaggio decidibile.
\vspace{1em}
\text{}
\newline
\hbox{\textbf{DIMOSTRAZIONE.}}
La seguente TM $P$ decide $A_{REX}$.
\vspace{1em}
\text{}
\newline
$P = $ "Sulla stringa di input $\langle R,w \rangle$, dove $R$ è un'espressione regolare e $w$ è una stringa:
\begin{enumerate}
    \item Converte l'espressione regolare $R$ in un NFA $A$ equivalente, mediante la procedura di conversione data nel \hyperref[teorema-1.54]{\textcolor{blue}{Teorema 1.54}}.
    \item Esegue la TM $N$ su input $\langle A,w \rangle$.
    \item Se $N$ accetta, accetta; altrimenti rifiuta."
\end{enumerate}
\vspace{1em}
\text{}
\newline
I Teoremi 4.1, 4.2 e 4.3 ci dicono che, ai fini della decidibilità, è equivalente presentare alla macchina di Turing un DFA, un NFA, oppure un espressione regolare, perchè la macchina è in grado di convertire una codifica nell'altra.

Ora affrontiamo un diverso tipo di problema concernente gli automi a stati finiti: il \textbf{test del vuoto} per il linguaggio di un automa finito.
Nei precedenti tre teoremi dovevamo decidere se un automa finito accettasse una particolare stringa.
Nella prossima dimostrazine, dobbiamo determinare se un automa finito accetta una qualche stringa.
Consideriamo
$$
E_{DFA} = \{\langle A \rangle \mid A \text{ è un DFA e } L(A) = \emptyset \}.
$$

\paragraph{Teorema 4.4(orale)}
\label{teorema-4.4}
$E_{DFA}$ è un linguaggio decidibile.
\vspace{1em}
\text{}
\newline
\hbox{\textbf{DIMOSTRAZIONE.}}
Il DFA accetta almeno una stringa se e solo se dallo stato iniziale può raggiungere uno stato di accettazione percorrendo il verso delle frecce del DFA.
Per verificare questa condizione possiamo progettare un TM $T$ che utilizza un algoritmo di marcatura analogo a quello utilizzato nell'\hyperref[fig:example_image1]{\textcolor{blue}{Esempio 3.23}}.
\vspace{1em}
\text{}
\newline
$T = $ "Su input $\langle A \rangle$, dove $A$ è un DFA:
\begin{enumerate}
    \item Marca lo stato iniziale di $A$.
    \item Ripete finchè nessuno stato nuovo viene marcato:
    \item Marca qualsiasi stato che ha una transizione proveniente da uno stato già marcato.
    \item Se nessuno stato di accettazione è stato marcato, accetta; altrimenti rifiuta."
\end{enumerate}
\vspace{1em}

Il prossimo teorema afferma che determinare se due DFA riconoscono lo stesso linguaggio è decidibile. Sia
$$
EQ_{DFA} = \{\langle A,B \rangle \mid A \text{ e } B \text{ sono DFA e } L(A) = L(B) \}.
$$

\paragraph{Teorema 4.5(orale)}
\label{teorema-4.5}
$EQ_{DFA}$ è un linguaggio decidibile.
\vspace{1em}
\text{}
\newline
\hbox{\textbf{DIMOSTRAZIONE.}}
Per dimostrare questo teorema usiamo il \hyperref[teorema-4.4]{\textcolor{blue}{Teorema 4.4}}.
Costruiamo un nuovo DFA $C$ a partire da $A$ e $B$ , dove $C$ accetta solo quelle stringhe che sono accettate da $A$ o da $B$, ma non da entrambi.
In particolare, se $A$ e $B$ riconoscono lo stesso linguaggio, $C$ non accetterà nulla.
Il linguaggio di $C$ è
$$
L(C) = (L(A) \cap \overline{L(B)}) \cup (\overline{L(A)} \cap L(B)).
$$
Questa espressione è a volte denotata come \textbf{differenza simmetrica} di $L(A)$ e $L(B)$ ed è illustrata nella figura seguente.
Qui $\overline{L(A)}$ denota il complemento di $L(A)$.
La differenza simmetrica è utile perchè $L(C) = \emptyset$ se e solo se $L(A) = L(B)$.
Siamo in grado di costruire $C$ da $A$ e $B$ con le costruzioni fatte per dimostrare che la classe dei linguaggi regolari risulta chiusa rispetto alle operazioni di complemento, unione e intersezione.
Queste costruzioni sono algoritmi che possono essere eseguiti da macchine di Turing.
Una volta che abbiamo costruito $C$ possiamo usare il \hyperref[teorema-4.4]{\textcolor{blue}{Teorema 4.4}} per determinare se $L(C) = \emptyset$ o meno.
Se è vuoto, $L(A)$ e $L(B)$ devono essere uguali.
\vspace{1em}
\text{}
\newline
$F = $"Su input $\langle A,B \rangle$, dove $A$ e $B$ sono DFA:
\begin{enumerate}
    \item Costruisce il DFA $C$ come descritto.
    \item Esegue la TM $T$ del \hyperref[teorema-4.4]{\textcolor{blue}{Teorema 4.4}} su input $\langle C \rangle$.
    \item Se $T$ accetta, accetta; altrimenti rifiuta."
\end{enumerate}

\begin{figure}[H]
    \centering
    \includegraphics[width=0.4\textwidth]{Immagini/72.png}
    \caption{Differenza simmetrica di due linguaggi}
    \label{figura-4.6}
\end{figure}

\subsubsection{Problemi decidibili relativi a linguaggi context-free}
Descriviamo ora algoritmi per determinare se una CFG genera una particolare stringa e per determinare se il linguaggio di una CGF è vuoto.
\newline
Consideriamo
$$
A_{CFG} = \{\langle G,w \rangle \mid G \text{ è una CFG che genera la stringa di input } w \}.
$$

\paragraph{Teorema 4.7(orale)}
\label{teorema-4.7}
$A_{CFG}$ è un linguaggio decidibile.
\newline

\hbox{\textbf{IDEA.}}
Per la CFG $G$ e la stringa $w$ vogliamo determinare se $G$ genera $w$.
Un'idea è quella di utilizzare $G$ per passare attraverso tutte le derivazioni per determinare se ne esiste una di $w$.
Quest'idea non funziona, poichè bisognerebbe provare infinite derivazioni.
Se $G$ non genera $w$, questo algoritmo potrebbe non fermarsi.
Tale idea dà una macchina di Turing che è un riconoscitore, non un decisore, per $A_{CFG}$.
Per rendere questa macchina di Turing un decisore occorre assicurare che l'algoritmo provi solo un numero finito di derivazioni.
Precedentemente abbiamo dimostrato che, se G fosse in forma normale di Chomsky, qualsiasi derivazione di $w$ avrebbe $2n - 1$ passi, dove $n$ è la lunghezza di $w$.
In tal caso sarebbe sufficiente controllare solo derivazioni di $2n - 1$ passi per determinare se $G$ genera $w$.
Tali derivazioni sono in numero finito.
Possiamo convertire $G$ in forma normale di Chomsky utilizzando la procedura data nel \hyperref[Forma normale di Chomsky]{\textcolor{blue}{Teorema}}.
\vspace{1em}
\text{}
\newline
\hbox{\textbf{DIMOSTRAZIONE.}}
Diamo la TM $S$ per $A_{CFG}$
\vspace{1em}
\text{}
\newline
$S = $ "Su input $\langle G,w \rangle$, dove $G$ è una CFG e $w$ è una stringa:
\begin{enumerate}
    \item Converti la CFG $G$ in forma normale di Chomsky.
    \item Lista tutte le derivazioni di $2n - 1$ passi, dove $n$ è la lunghezza di $w$; tranne se $n = 0$, in tal caso lista tutte le derivazioni di un passo.
    \item Se una di tali derivazioni genera $w$, accetta; altrimenti rifiuta."
\end{enumerate}

Il problema di determinare se una CFG genera una particolare stringa è correlato al problema della compilazione dei linguaggi di programmazione.
Ricordiamo che abbiamo dato procedure per la conversione in entrambi i sensi tra CFG e PDA nel \hyperref[teorema-2.20]{\textcolor{blue}{Teorema 2.20}}.
Quindi tutto quello che diciamo circa la decidibilità dei problemi concernenti le CFG vale anche per i PDA.
Torniamo al problema dei test del vuoto per il linguaggio di una CFG.
Come abbiamo fatto per i DFA, possiamo dimostrare che il problema di determinare se una CFG genera almeno una stringa è decidibile. Sia
$$
E_{CFG} = \{\langle G \rangle \mid G \text{ è una CFG e } L(G) = \emptyset \}.
$$

\paragraph{Teorema 4.8(orale)}
\label{teorema-4.8}
$E_{CFG}$ è un linguaggio decidibile.
\vspace{1em}
\text{}
\newline
\hbox{\textbf{IDEA.}}
Per progettare un algoritmo per questo problema, potremmo tentare di utilizzare la TM $S$ del \hyperref[teorema-4.7]{\textcolor{blue}{Teorema 4.7}}.
Tale teorema afferma che siamo in grado di verificare se una CFG genera una particolare stringa $w$.
Per determinare se $L(G) = \emptyset$, l'algoritmo potrebbe tentare di passare attraverso tutte le possibili $w$, una per una.
Però esistono infinite $w$ da testare, quindi questo metodo potrebbe far sì che non termini mai.
Abbiamo bisogno di trovare un approccio differente. Per determinare se il linguaggio di una grammatica è vuoto, abbiamo bisogno di testare se la variabile iniziale può generare una stringa di terminali.
L'algoritmo fa questo risolvendo un problema più generale.
Determina \textit{per ogni variabile} se essa è in grado di generare una stringa di terminali.
Quando l'algoritmo ha determinato che una variabile può generare qualche stringa di terminali, l'algoritmo tiene traccia di queste informazioni marcando tale variabile.
Dapprima, l'algoritmo marca tutti i simboli terminali della grammatica.
Poi, scandisce tutte le regole della grammatica.
Se trova una regola che consente a qualche variabile di essere sostituita da una stringa di simboli, che sono già tutti marcati, l'algoritmo sa che anche questa varivile può essere marcata.
L'algoritmo continua in questo modo finchè non può marcare ulteriori variabili.
La TM $R$ implementa questo algoritmo.
\vspace{1em}
\text{}
\newline
\hbox{\textbf{DIMOSTRAZIONE.}}
\vspace{1em}
\text{}
\newline
$R = $ "Su input $\langle G \rangle$, dove $G$ è una CFG:
\begin{enumerate}
    \item Marca tutti i simboli terminali di $G$.
    \item Ripeti finchè nessuna nuova variabile viene marcata:
    \item Marca una qualsiasi variabile $A$ tale che $G$ ha una regola $A \rightarrow U_{1}U_{2}...U_{k}$, dove ogni $U_{i}$ è già stato marcato.
    \item Se la variabile iniziale non è segnata, accetta; altrimenti rifiuta."
\end{enumerate}

Consideriamo ora il problema di determinare se due grammatiche context-free generano lo stesso linguaggio. Sia
$$
EQ_{CFG} = \{\langle G,H \rangle \mid G \text{ e } H \text{ sono CFG e } L(G) = L(H) \}.
$$
Il \hyperref[teorema-4.5]{\textcolor{blue}{Teorema 4.5}} fornisce un algoritmo che decide l'analogo linguaggio $EQ_{DFA}$ per i DFA.
Abbiamo utilizzato la procedura di decisione per $E_{DFA}$ per dimostrare che $EQ_{DFA}$ è decidibile.
Ma qualcosa non va con questa idea! La classe dei linguaggi context-free non è chiusa rispetto al complemento o intersezione.
Infatti, $EQ_{CFG}$ non è decidibile.
Ora mostriamo che i linguaggi context-free possono essere decisi dalle macchine di Turing.

\paragraph{Teorema 4.9}
\label{teorema-4.9}
Ogni linguaggio context-free è decidibile.
\vspace{1em}
\text{}
\newline
\hbox{\textbf{IDEA.}}
Sia $A$ un CFL. Il nostro obiettivo è mostrare che $A$ è decidibile.
Una (cattiva) idea è quella di convertire un PDA per $A$ direttamente in una TM.
Ciò non è difficile da realizzare, perchè simulare una pila con il nastro di una TM è semplice.
Il PDA per $A$ può essere non deterministico, ma ciò sembra andare bene, perchè siamo in grado di convertirlo in una TM non deterministica e sappiamo che qualsiasi TM non deterministica può essere convertita in una TM deterministica equivalente.
C'è tuttavia una difficoltà.
Alcuni rami della computazione del PDA possono andare avanti per sempre, leggendo e scrivendo la pila senza mai arrestarsi.
La TM che effettua la simulazione avrebbe quindi alcuni cammini che non terminano mai, quindi la TM non sarebbe un decisore.
\vspace{1em}
\text{}
\newline
E' necessaria una diversa soluzione.
Dimostriamo il teorema con la TM $S$ che abbiamo progettato nel \hyperref[teorema-4.7]{\textcolor{blue}{Teorema 4.7}} per decidere $A_{CFG}$.
\vspace{1em}
\text{}
\newline
\hbox{\textbf{DIMOSTRAZIONE.}}
Sia $G$ una CFG per $A$, progettiamo una TM $M_G$ che decide $A$.
Costruiamo una copia di $G$ in $M_G$.
Essa funziona come segue.

$M_G = $ "Su input $w$:
\begin{enumerate}
    \item Esegue la TM $S$ del \hyperref[teorema-4.7]{\textcolor{blue}{Teorema 4.7}} su input $\langle G,w \rangle$.
    \item Se $S$ accetta, accetta; altrimenti rifiuta."
\end{enumerate}
\vspace{1em}
\text{}
\newline
Il Teorema 4.9 fornisce il collegamento finale nelle relazioni tra le quattro principali classi di linguaggi che abbiamo descritto fin'ora: regolari, context-free, decidibili e Turing-riconoscibili.
\begin{figure}[H]
    \centering
    \includegraphics[width=0.4\textwidth]{Immagini/73.png}
    \caption{Relazioni tra le classi di linguaggi}
    \label{figura-4.10}
\end{figure}

\subsection{Indecidibilità}
In questa sezione dimostriamo uno dei teoremi più importanti della teoria della computazione:
esiste un problema specifico che è algoritmicamente irrisolvibile.
\vspace{1em}
\text{}
\newline
Ora affronteremo il nostro primo teorema che stabilisce l'indecidibilità di uno specifico linguaggio:
il problema di determinare se una macchina di Turing accetta una determinata stringa in input.
Chiamiamo tale linguaggio $A_{TM}$ per analogia con $A_{DFA}$ e $A_{CFG}$.
Tuttavia, mentre $A_{DFA}$ e $A_{CFG}$ sono decidibili, $A_{TM}$ non lo è. Sia
$$
A_{TM} = \{\langle M,w \rangle \mid M \text{ è una TM e } M \text{ accetta la stringa } w \}.
$$

\paragraph{Teorema 4.11(orale 24-30)}
\label{teorema-4.11}
\text{}
\newline
$A_{TM}$ è indecidibile.
\vspace{1em}
\text{}
\newline
Prima di vedere la dimostrazione, osserviamo che $A_{TM}$ è Turing-riconoscibile.
Quindi questo teorema mostra che i riconoscitori sono più potenti dei decisori.
Richiedere che una TM si fermi su ogni input limita le tipologie dei linguaggi che possono essere riconosciuti.
La seguente macchina di Turing $U$ riconosce $A_{TM}$.
\vspace{1em}
\text{}
\newline
$U = $ "Su input $\langle M,w \rangle$, dove $M$ è una TM e $w$ è una stringa:
\begin{enumerate}
    \item Simula $M$ su input $w$.
    \item Se durante la computazione $M$ entra nello stato di accettazione, accetta; se $M$ entra nello stato di rifiuto, rifiuta."
\end{enumerate}

Si noti che questa macchina cicla su input $\langle M,w \rangle$ se $M$ cicla su $w$, questo è il motivo per cui non decide $A_{TM}$.
Se l'algoritmo avesse modo di determinare che $M$ non si ferma su $w$, sarebbe in tal caso in grado di rifiutare $w$.
Come dimostreremo, un algoritmo non ha modo di determinarlo.

La macchina di Turing $U$ è, di per sè, interessante.
E' un esempio della prima macchina universale di Turing introdotta nel 1936.
Questa macchina è chiamata universale perchè è in grado di simulare qualsiasi altra macchina di Turing a partire dalla descrizione di tale macchina.
La macchina di Turing universale ha avuto un ruolo importante nello stimolare lo sviluppo di computer a programma memorizzato.

\subsubsection{Il metodo della diagonalizzazione}

\begin{tcolorbox}[title=Definizione 4.12]
\label{definizione-4.12}
Supponiamo di avere gli insiemi $A$ e $B$ ed una funzione $f$ da $A$ in $B$.
Diciamo che $f$ è \textbf{\textit{iniettiva}}, se essa non mappa mai due elementi diversi in uno stesso punto - cioè se $f(a) \neq f(b)$ per ogni $a,b \in A$ con $a \neq b$.
Diciamo che $f$ è \textbf{\textit{suriettiva}}, se tocca ogni elemento di $B$ - cioè se per ogni $b \in B$ esiste un $a \in A$ tale che $f(a) = b$.
Diciamo che $A$ e $B$ hanno la \textbf{\textit{stessa cardinalità}} se esiste una funzione iniettiva e suriettiva $f : A \rightarrow B$.
Una funzione che è sia iniettiva che suriettiva è detta \textbf{\textit{biettiva}}.
In una funzione biettiva ogni elemento di $A$ viene mappato in un unico elemento di $B$ e per ogni elemento di $B$ esiste un unico elemento di $A$ che viene mappato in esso.
Una biezione è un modo semplice per accoppiare elementi di $A$ con elementi di $B$.
\end{tcolorbox}

\begin{figure}[H]
    \centering
    \includegraphics[width=0.4\textwidth]{Immagini/74.png}
    \caption{Biezione tra due insiemi}
    \label{figura-4.13}
\end{figure}

\begin{tcolorbox}[title=Definizione 4.14]
    Un insieme $A$ è \textbf{\textit{numerabile}} se è finito o ha la stessa cardinalità di $\mathbb{N}$.    
\end{tcolorbox}

\begin{figure}[H]
    \centering
    \includegraphics[width=0.3\textwidth]{Immagini/75.png}
    \label{figura-4.14}
\end{figure}

Dopo aver visto la biezione da $\mathbb{N}$ a $\mathbb{Q}$ si potrebbe pensare di dimostrare che presi qualsiasi coppia di insiemi finiti, questi hanno la stessa dimensione.
Dopo tutto dovete solo mostrare una biezione, e questo esempio mostra che esistono biezioni sorprendenti.
Tuttavia, per alcuni insiemi infiniti non c'è alcuna biezione con $\mathbb{N}$.
Questi insiemi sono semplicemente troppo grandi.
\newline
Un tale insieme è detto \textbf{\textit{non numerabile}}.

L'insieme dei numeri reali è un esempio di insieme non numerabile.

\paragraph{Teorema 4.17}
\label{teorema-4.17}
\text{}
\newline
$\mathbb{R}$ è non numerabile.  
\vspace{1em}
\text{}
\newline
\hbox{\textbf{DIMOSTRAZIONE.}}
Per dimostrare che $\mathbb{R}$ è non numerabile, mostriamo che non esiste una biezione tra $\mathbb{N}$ e $\mathbb{R}$.
La dimostrazione è per assurdo.
Supponiamo quindi che esista una biezione $f$ tra $\mathbb{N}$ e $\mathbb{R}$.
Il nostro compito è quello di dimostrare che $f$ non funziona come dovrebbe.
Per essere una biezione, $f$ deve mappare ogni intero positivo in un numero reale.
Tuttavia troveremo un numeero $x$ in $\mathbb{R}$ che non è mappato da $f$, il che rappresenterà la contraddizione cercata.
Un metodo per trovare questo numero $x$ è quello di costruirlo.
Scegliamo ogni cifra di $x$ in modo da renderlo differente da ogni numero reale accoppiato con un elemento di $\mathbb{N}$.
Alla fine saremo sicuri che $x$ risulta diverso da ogni numero reale accoppiato con un elemento di $\mathbb{N}$.
Possiamo illustrare questa idea con un esempio.
Supponiamo che la biezione $f$ esista.
Prendiamo $f(1) = 3.14159...,f(2) = 55.55555...,f(3)=...,$ e così via, tanto per dare alcuni valori per $f$.
Allora $f$ accoppia il numero $1$ con $3.14159...$, il numero $2$ con $55.55555...$, e così via.
La tabella seguente mostra alcuni valori di un'ipotetica biezione tra $\mathbb{N}$ e $\mathbb{R}$.
\begin{figure}[H]
    \centering
    \includegraphics[width=0.15\textwidth]{Immagini/76.png}
    \label{figura-4.18}
\end{figure}
Costruiamo il numero $x$ desiderato, dandone la rappresentazione decimale.
Si tratta di un numero comreso tra $0$ e $1$, quindi tutte le sue cifre significative sono cifre frazionarie che seguono la virgola decimale.
Il nostro obiettivo è quello di garantire che $x \neq f(n)$ per ogni $n \in \mathbb{N}$.
Per garantire $x \neq f(1)$, scegliamo la cifra di $x$ come una qualsiasi cifra diversa dalla prima cifra decimale 1 di $f(1) = 3.14159...$.
Arbitrariamente, scegliamola pari a 4.
Per assicurarci che $x \neq f(2)$, scegliamo la seconda cifra decimale di $x$ diversa dalla seconda cifra decimale 5 di $f(2) = 55.55555...$.
Arbitrariamente, scegliamola pari a 6.
La terza cifra frazionaria di $f(3) = 0.12345...$ è 3, quindi scegliamo la terza cifra frazionaria di $x$ pari a 4.
Continuando in questo modo lungo la diagonale della tavola per $f$, otteniamo tutte le cifre di $x$, come mostrato nella tabella seguente.
Sappiamo che $x$ non è $f(n)$ per ogni $n$ perchè differisce da $f(n)$ nella n-esima cifra frazionaria.
(Un piccolo problema nasce dal fatto che alcuni numeri, come $0.1999...$ e $0.2000...$, sono uguali, anche se le loro rappresentazioni decimali sono diverse.
Evitiamo questo problema semplicemente non selezionando mai le cifre 9 o 0 per $x$).
\begin{figure}[H]
    \centering
    \includegraphics[width=0.4\textwidth]{Immagini/77.png}
    \label{figura-4.19}
\end{figure}

Il teorema precedente ha un'importante applicazione nella teoria della computazione.
Esso dimostra che alcuni linguaggi non sono decidibili e neppure Turing-riconoscibili, per la ragione che l'insieme dei linguaggi è non numerabile mentre l'insieme di tutte le macchine di Turing è numerabile.
Poichè ogni macchina di Turing è in grado di riconoscere un solo linguaggio e ci sono più linguaggi che macchine di Turing, alcuni linguaggi non sono riconosciuti da una qualche macchina di Turing.
Tali linguaggi non sono Turing-riconoscibili, come enunciato nel seguente corollario.

\paragraph{Corollario 4.18(orale 24-30)}
\label{corollario-4.18}
\text{}
Alcuni linguaggi non sono Turing-riconoscibili.
\vspace{1em}
\text{}
\newline
\hbox{\textbf{DIMOSTRAZIONE.}} 
Per dimostrare che l'insieme di tutte le macchine di Turing è numerabile dobbiamo prima osservare che l'insieme di tutte le stringhe $\Sigma^*$ è numerabile, per ogni alfabeto $\Sigma$.
Avendo solo un numero finito di stringhe di ogni lunghezza, possiamo formare una lista di $\Sigma^*$ scrivendo tutte le stringhe di lunghezza 0, lunghezza 1, lunghezza 2, e così via.
L'insieme di tutte le macchine di Turing è numerabile perchè ogni macchina di Turing $M$ può essere codificata con una stringa $\langle M \rangle$.
Se ci limitiamo a tralasciare quelle stringhe che non sono la codifica di una macchina di Turing, possiamo ottenere una lista di tutte le macchine di Turing.
Per mostrare che l'insieme di tutti i linguaggi è non numerabile, dobbiamo prima osservare che l'insieme delle sequenze binarie infinite è non-numerabile.
Una sequenza binaria infinita è una sequenza senza fine di 0 e 1.
Sia $\mathcal{B}$ l'insieme di tutte le sequenze binarie infinite.
Possiamo dimostrare che $\mathcal{B}$ è non numerabile utilizzando una dimostrazione mediante diagonalizzazione simile a quella che abbiamo utilizzato nel \hyperref[teorema-4.17]{\textcolor{blue}{Teorema 4.17}} per mostrare la non numerabilità di $\mathbb{R}$.
Sia $\mathcal{L}$ l'insieme di tutti i linguaggi sull'alfabeto $\Sigma$.
Mostriamo che $\mathcal{L}$ è non numerabile dando una corrispondenza con $\mathcal{B}$, dimostrando così che i due insiemi hanno la stessa cardinalità.
Sia $\Sigma^* = \{s_1, s_2, s_3, ...\}$.
Ogni linguaggio $A \in \mathcal{L}$ ha un'unica sequenza in $\mathcal{B}$.
Il bit $i$-esimo della sequenza è un 1 se $s_i \in A$ e 0 altrimenti; questa è chiamata la \textbf{\textit{sequenza caratteristica}} di $A$.
Per esempio, se $A$ fosse il linguaggio di tutte le stringhe che iniziano con uno 0 sull'alfabeto binario, la sia sequenza caratteristica $X_A$ sarebbe
$$
\Sigma^* = \{\varepsilon, 0, 1, 00, 01, 10, 11, 000, 001, 010, ...\}; \\
$$
$$
A = \{ 0, 00,01, 000, 001,...\}; \\
$$
$$
X_A = 0\quad 1\quad 0\quad 1\quad 1\quad 0\quad 0\quad 1\quad 1\quad ...\quad.
$$
La funzione $f: \mathcal{L} \rightarrow \mathcal{B}$, dove $f(A)$ è uguale alla sequenza caratteristica di $A$, è sia iniettiva che suriettiva, quindi è una biezione.
Pertanto, essendo $\mathcal{B}$ non numerabile, anche $\mathcal{L}$ è non numerabile.

Abbiamo così dimostrato che l'insieme di tutti i linguaggi non può essere messo in corrispondenza biunivoca con l'insieme di tutte le macchine di Turing.
Concludiamo quindi che alcuni linguaggi non sono riconosciuti da alcuna macchina di Turing.

\subsubsection{Un linguaggio indecidibile}
Siamo ora pronti a dimostrare il \hyperref[teorema-4.11]{\textcolor{blue}{Teorema 4.11}}, l'indecidibilità del linguaggio
$$
A_{TM} = \{\langle M,w \rangle \mid M \text{ è una TM e } M \text{ accetta la stringa } w \}.
$$
\vspace{1em}
\text{}
\newline
\hbox{\textbf{DIMOSTRAZIONE.}}
Assumiamo che $A_{TM}$ è decidibile, per poi ottenere una contraddizione.
Supponiamo che $H$ sia un decisore per $A_{TM}$.
Sull'input $\langle M,w \rangle$, dove $M$ è una TM e $w$ è una stringa, $H$ si ferma ed accetta se $M$ accetta $w$.
Inoltre, $H$ si ferma e rifiuta se $M$ non accetta $w$.
In altri termini, assumiamo che $H$ è una TM, dove
$$
H(\langle M,w \rangle) = 
\begin{cases}
    accetta & \text{se } M \text{ accetta } w, \\
    rifiuta & \text{se } M \text{ non accetta } w.
\end{cases}
$$
Ora costruiamo una nuova macchina di Turing $D$ avente $H$ come sottoprocedura.
Questa nuova TM chiama $H$ per determinare cosa fa $M$ quando l'input è la sua stessa descrizione $\langle M \rangle$.
Una volta che $D$ ha determinato questa informazione, essa fa il contrario.
Cioè, rifiuta se $M$ accetta ed accetta se $M$ non accetta. Diamo ora una descrizione di $D$.
\newline
$D = $ "Sull'input $\langle M \rangle$, dove $M$ è una TM:
\begin{enumerate}
    \item Esegue la TM $H$ su input $\langle M,\langle M \rangle \rangle$.
    \item Dà in output l'opposto di ciò che $H$ dà in output. Cioè, se $H$ accetta, rifiuta; se $H$ rifiuta, accetta."
\end{enumerate}

Non lasciatevi confondere dall'idea di attivare una macchina sulla sua stessa descrizione!
Ciò è simile all'esecuzione di un programma che chiama se stesso in input, qualcosa che si fa a volte nella pratica.
Ad esempio, un compilatore è un programma che traduce altri programmi.
Un compilatore per il linguaggio Python può essere esso stesso scritto in Python, quindi eseguire tale programma su se stesso avrebbe senso.
Ricapitolando,
$$
D(\langle M \rangle) =
\begin{cases}
    accetta & \text{se } M \text{ non accetta } \langle M \rangle, \\
    rifiuta & \text{se } M \text{ accetta } \langle M \rangle.
\end{cases}
$$
Cosa succede quando eseguiamo $D$ con la sua stessa descrizione $\langle D \rangle$ in input?
In tal caso, otteniamo
$$
D(\langle D \rangle) =
\begin{cases}
    accetta & \text{se } D \text{ non accetta } \langle D \rangle, \\
    rifiuta & \text{se } D \text{ accetta } \langle D \rangle.
\end{cases}
$$
Indipendentemente da ciò che $D$ fa, essa è costretta a dare il contrario, il che è ovviamente una contraddizione.
Quindi, nè la TM $D$ nè la TM $H$ possono esistere.
\vspace{1em}
\newline
Rivediamo i passi di questa prova. Supponiamo che la TM $H$ decida $A_{TM}$.
Quindi usiamo $H$ per costruire una TM $D$ che prende in input $\langle M \rangle$, tale che $D$ accetta $M$ esattamente quando $M$ non accetta il suo input $\langle M \rangle$.
Infine, eseguiamo $D$ su se stessa.
Quindi, le macchine eseguono le seguenti azioni, dove l'ultima riga fornisce la contraddizione.
\begin{itemize}
    \item $H$ accetta $\langle M,w \rangle$ esattamente quando $M$ accetta $w$.
    \item $D$ rifiuta $\langle M \rangle$ esattamente quando $M$ accetta $\langle M \rangle$.
    \item $D$ rifiuta $\langle D \rangle$ esattamente quando $D$ accetta $\langle D \rangle$.
\end{itemize}           
Dove si usa la diagonalizzazione nella dimostrazione del \hyperref[teorema-4.11]{\textcolor{blue}{Teorema 4.11}}?
Essa diviene evidente quando si esaminano le tavole del comportamenteo delle TM $H$ e $D$.
In queste tavole indicizziamo le righe con tutte le TM $M_1,M_2,M_3,...$ e le colonne con tutte le descrizioni $\langle M_1 \rangle, \langle M_2 \rangle, \langle M_3 \rangle$, ... delle TM.
Le entrate dicono se la macchina di una determinata riga accetta l'input della colonna corrispondente.
L'entrata accetta se la macchina accetta l'input, ma è vuota, se rifiuta o entra in loop su quell'input.
Nella seguente figura abbiamo creato alcune voci per illustrare l'idea.
\begin{figure}[H]
    \centering
    \includegraphics[width=0.4\textwidth]{Immagini/79.png}
    \caption{L'entrata $i,j$ accetta se $M_i$ accetta $\langle M_j \rangle$}
    \label{figura-4.20} 
\end{figure}
Nella figura seguente, abbiamo aggiunto $D$ alla \hyperref[figura-4.20]{Figura 4.20}.
Secondo la nostra assunzione, $H$ è una TM così come $D$.
Quindi deve comparire nella lista delle TM.
Si noti che $D$ calcola il contrario degli elementi sulla diagonale.
La contraddizione si ottiene in corrispondenza del punto interrogativo, dove l'entrata dovrebbe essere l'opposto di se stessa.
\begin{figure}[H]
    \centering
    \includegraphics[width=0.4\textwidth]{Immagini/80.png}
    \caption{La contraddizione nella dimostrazione del \hyperref[teorema-4.11]{\textcolor{blue}{Teorema 4.11}}}
    \label{figura-4.21}
\end{figure}

\subsubsection{Un linguaggio non Turing-riconoscibile}
Nella sezione precedente abbiamo dimostrato che un linguaggio - precisamente $A_{TM}$ - non è decidibile.
Ora mostriamo un linguaggio che non è neppure Turing-riconoscibile.
Si noti che $A_{TM}$ non basta allo scopo, perchè abbiamo dimostrato che $A_{TM}$ è Turing-riconoscibile.
Il teorema seguente mostra che, se un linguaggio ed il suo complemento sono entrambi Turing-riconoscibili, allora il linguaggio è decidibile.
Quindi, per qualsiasi linguaggio non decidibile, o esso non è Turing-riconoscibile oppure il suo complemento non è Turing-riconoscibile.
Ricordiamo che il complemento di un linguaggio è il linguaggio costituito da tutte le stringhe che non sono nel linguaggio.
Diciamo che un linguaggio è \textbf{\textit{co-Turing-riconoscibile}} se esso è il complemento di un linguaggio Turing-riconoscibile.

\paragraph{Teorema 4.22(orale)}
\label{teorema-4.22}
\text{}
\newline
Un linguaggio è decidibile se e solo se è Turing-riconoscibile e co-Turing-riconoscibile.
\newline
In altre parole, un linguaggio è decidibile esattamente quando sia esso che il suo complemento sono Turing-riconoscibili.
\vspace{1em}
\text{}
\newline
\hbox{\textbf{DIMOSTRAZIONE.}}
Abbiamo due direzioni da dimostrare.
In primo luogo, se $A$ è decidibile, possiamo facilmente vedere che sia $A$ che il suo complemento $\overline{A}$ sono Turing-riconoscibili.
Qualsiasi linguaggio decidibile è Turing-riconoscibile, e il complemento di un linguaggio decidibile è anch'esso decidibile.
Per la direzione inversa, se entrambi $A$ e $\overline{A}$ sono Turing-riconoscibili, indichiamo con $M_1$ il riconoscitore per $A$ e con $M_2$ il riconoscitore per $\overline{A}$.
La seguente macchina di Turing $M$ è un decisore per $A$.
\vspace{1em}
\text{}
\newline
$M = $ "Su input $w$:
\begin{enumerate}
    \item Esegue sia $M_1$ che $M_2$ su $w$ in parallelo.
    \item Se $M_1$ accetta, accetta; se $M_2$ accetta, rifiuta."
\end{enumerate}
L'esecuzione di due macchine in parallelo significa che $M$ ha due nastri, uno per simulare $M_1$ e l'altro per simulare $M_2$.
In questo caso $M$ alterna la simulazione di un passo di $M_1$ con un passo di $M_2$ e continua così finchè una delle due macchine accetta.
Ora dimostriamo che $M$ decide $A$.
Ogni stringa $w$ è in $A$ o in $\overline{A}$.
Pertanto una tra $M_1$ e $M_2$ accetta $w$.
Poichè $M$ si ferma ogni volta che $M_1$ accetta oppure $M_2$ accetta, allora $M$ si ferma sempre quindi è un decisore.
Inoltre, accetta tutte le stringhe in $A$ e rifiuta tutte le stringhe che non sono in $A$.
Quindi $M$ è un decisore per $A$, e pertanto $A$ è decidibile.
\newline
\paragraph*{Corollario 4.23}
\label{corollario-4.23}
\text{}
\newline
\begin{tcolorbox}[title=Corollario 4.23]
Il linguaggio $\overline{A_{TM}}$ non è Turing-riconoscibile.
\end{tcolorbox}
\vspace{1em}
\text{}
\newline
\hbox{\textbf{DIMOSTRAZIONE.}}
Sappiamo che $A_{TM}$ è Turing-riconoscibile.
Se anche il suo complemento $\overline{A_{TM}}$ fosse Turing-riconoscibile, allora $A_{TM}$ sarebbe decidibile, il che è falso.
Il \hyperref[teorema-4.11]{\textcolor{blue}{Teorema 4.11}} ci dice che $A_{TM}$ non è decidibile, quindi $\overline{A_{TM}}$ non è Turing-riconoscibile.

\section{Riducibilità}
In questo capitolo esamineremo vari altri problemi irrisolvibili.
Nel far ciò introdurremo il metodo principale per dimostrare che alcuni problemi sono computazionalmente irrisolvibili.
Tale metodo si chiama \textbf{\textit{riducibilità}}.

Una \textbf{\textit{riduzione}} è un modo di convertite un problema in un altro problema in modo tale che una soluzione al secondo problema può essere usata per risolvere il primo problema.
La riducibilità coinvolge sempre due problemi, che noi chiamiamo $A$ e $B$.
Se $A$ si riduce a $B$, possiamo usare una soluzione per $B$ per risolvere $A$.
Notate che la riducibilità non dice nulla circa la soluzione dei problemi $A$ o $B$ individualmente, ma soltanto qualcosa circa la risolubilità di $A$ quando abbiamo una soluzione per $B$.
\newline
La riducibilità svolge un ruolo importante nella classificazione dei problemi in base alla decidibilità e, successivamente, anche nella teoria della complessità.

Quando $A$ è riducibile a $B$, trovare la soluzione di $A$ non può essere più difficile di risolvere $B$ perchè una soluzione per $B$ offre una soluzione ad $A$.
In termini di teoria della computabilità, se $A$ è riducibile a $B$ e $B$ è decidibile, allora $A$ è decidibile.
Equivalentemente, se $A$ è indecidibile e riducibile a $B$, $B$ è indecidibile.
Questa ultima parte è la chiave per dimostrare che alcuni problemi sono indecidibili.

In breve, il nostro metodo per dimostrare che un problema è indecidibile sarà quello di mostrare che qualche altro problema già noto per essere indecidibile si riduce ad esso.

\subsection{Problemi indecidibili dalla teoria dei linguaggi}
Abbiamo già stabilito l'indecidibilità di $A_{TM}$, il problema di determinare se una macchina di Turing accetta un dato input.
Consideriamo ora un problema affine, $HALT_{TM}$, il problema di determinare se una macchina di Turing si ferma (accettando o rifiutando) su un dato input.
Questo problema è generalmente noto come il \textbf{\textit{problema della fermata}}.
Usiamo l'indecidibilità di $A_{TM}$ per dimostrare l'indecidibilità del problema della fermata riducendo $A_{TM}$ a $HALT_{TM}$.
Sia
$$
HALT_{TM} = \{\langle M,w \rangle \mid M \text{ è una TM e si ferma su input } w \}.
$$

\paragraph{Teorema 5.1(orale)}
\label{teorema-5.1}
\text{}
\newline
$HALT_{TM}$ è indecidibile.
\vspace{1em}
\text{}
\newline
\hbox{\textbf{IDEA.}}
Questa dimostrazione è per assurdo.
Assumiamo che $HALT_{TM}$ sia decidibile e usiamo questa assunzione per dimostrare che $A_{TM}$ è decidibile, contraddicendo il \hyperref[teorema-4.11]{\textcolor{blue}{Teorema 4.11}}.
L'idea chiave è quella di mostrare che $A_{TM}$ è riducibile a $HALT_{TM}$.

Supponiamo di avere una TM $R$ che decide $HALT_{TM}$.
Usiamo quindi $R$ per costruire una TM $S$ che decide $A_{TM}$.
Per intuire come possiamo costruire $S$, immaginate di essere $S$.
Il vostro compito è decidere $A_{TM}$.
Ricevete un input $\langle M,w \rangle$.
Dovete dare in output \textit{accetta} se $M$ accetta $w$ e \textit{rifiuta} se $M$ non accetta $w$.
Provate a simulare $M$ su $w$.
Se accetta o rifiuta, fate lo stesso.
Ma potreste non essere in grado di determinare se $M$ è entrata in un ciclo, in questo caso la simulazione non terminerà.
Questo non va bene perchè voi siete un decisore e non vi è permesso ciclare.
Quindi questa idea non funziona.
Utilizziamo, invece, l'ipotesi che avete una TM $R$ che decide $HALT_{TM}$.
Con $R$, potete verificare se $M$ si ferma su $w$.
Se $R$ indica che $M$ non si ferma su $w$, rifiutate perchè $\langle M,w \rangle \notin A_{TM}$.
Tuttavia, se $R$ indica che $M$ si ferma su $w$, potete fare la simulazione senza alcun pericolo di ciclare.
Così, se la TM $R$ esistesse, potremmo decidere $A_{TM}$, ma sappiamo che $A_{TM}$ è indecidibile.
In virtù di questa contraddizione possiamo concludere che $R$ non esiste.
Pertanto $HALT_{TM}$ è indecidibile.
\vspace{1em}
\text{}
\newline
\hbox{\textbf{DIMOSTRAZIONE.}}
Assumiamo al fine di ottenere una contraddizione che la TM $R$ decide $HALT_{TM}$.
Costruiamo la TM $S$ per decidere $A_{TM}$, che opera come segue.
\vspace{1em}
\text{}
\newline
$S = $ "Su input $\langle M,w \rangle$, una codifica di una TM $M$ ed una stringa $w$:
\begin{enumerate}
    \item Esegue la TM $R$ su input $\langle M,w \rangle$.
    \item Se $R$ rifiuta, rifiuta.
    \item Se $R$ accetta, simula $M$ su $w$ finchè non si ferma.
    \item Se $M$ ha accettato, \textit{accetta}; se $M$ ha rifiutato, \textit{rifiuta}."
\end{enumerate}
Chiaramente, se $R$ decide $HALT_{TM}$, allora $S$ deve decidere $A_{TM}$.
Poichè $A_{TM}$ è indecidibile, $HALT_{TM}$ deve essere indecidibile.
\vspace{1em}
\text{}
\newline
Il \hyperref[teorema-5.1]{\textcolor{blue}{Teorema 5.1}} illustra il nostro metodo per dimostrare che un problema è indecidibile.
Questo metodo è comune alla maggior parte delle dimostrazioni di indecidibilità, tranne che per l'indecidibilità di $A_{TM}$ stesso, che viene dimostrata direttamente utilizzando il metodo della diagonalizzazione.

Presentiamo ora vari altri teoremi e le loro dimostrazioni come ulteriori esempi dell'uso della riducibilità per dimostrare l'indecidibilità.
Sia
$$
E_{TM} = \{\langle M \rangle \mid M \text{ è una TM e } L(M) = \emptyset \}.
$$
\paragraph{Teorema 5.2(orale)}
\label{teorema-5.2}
\text{}
\newline
$E_{TM}$ è indecidibile.
\vspace{1em}
\text{}
\newline
\hbox{\textbf{IDEA.}}
Seguiamo lo schema adottato nel \hyperref[teorema-5.1]{\textcolor{blue}{Teorema 5.1}}.
Assumuamo che $E_{TM}$ è decidibile e mostriamo che $A_{TM}$ è decidibile - una contraddizione.
Sia $R$ una TM che decide $E_{TM}$.
Utilizziamo $R$ per costruire la TM $S$ che decide $A_{TM}$.
Come funzionerà $S$ quando riceve in input $\langle M,w \rangle$?

Un'idea è che $S$ esegue $R$ su input $\langle M \rangle$ e vede se $R$ accetta.
Se lo fa, sappiamo che $L(M) = \emptyset$ e quindi $M$ non accetta $w$.
Ma, se $R$ rifiuta $\langle M \rangle$, tutto quello che sappiamo è che $L(M) \neq \emptyset$ e di conseguenza, che $M$ accetta qualche stringa - ma ancora non sappiamo se $M$ accetta la stringa $w$ in particolare.
Quindi abbiamo bisogno di utilizzare un'idea diversa.
Invece di eseguire $R$ su $\langle M \rangle$, eseguiamo $R$ su una modifica di $\langle M \rangle$.
Modifichiamo $\langle M \rangle$ così da garantire che $M$ rifiuta tutte le stringhe tranne $w$, ma su input $w$, funziona come al solito.
A questo punto, utilizziamo $R$ per determinare se la macchina modificata riconosce il linguaggio vuoto.
L'unica stringa che adesso la macchina può accettare è $w$, per cui il suo linguaggio sarà non vuoto sse accetta $w$.
Se $R$ accetta quando riceve in input la descrizione della macchina modificata, sappiamo che la macchina modificata non accetta nulla e che $M$ non accetta $w$.
\vspace{1em}
\text{}
\newline
\hbox{\textbf{DIMOSTRAZIONE.}}
Descriviamo la macchina modificata già descritta nell'idea di dimostrazione usando la nostra notazione standard.
Chiamiamola $M_1$.
\vspace{1em}
\text{}
\newline
$M_1 =$ "Su input $x$:
\begin{enumerate}
    \item Se $x \neq w$, rifiuta.
    \item Se $x = w$, esegui $M$ su $w$ e \textit{accetta} se $M$ accetta."
\end{enumerate}
\vspace{1em}
\text{}
\newline
Questa macchina ha la stringa $w$ come parte della sua descrizione.
Essa verifica se $x = w$ nel modo ovvio, attraverso la scansione dell'input e confrontandolo carattere per carattere con $w$ per determinare se coincidono.
Mettendo insieme tutto questo, assumiamo che la TM $R$ decide $E_{TM}$ e costruiamo la TM $S$ che decide $A_{TM}$ come segue.
\vspace{1em}
\text{}
\newline
$S = $ "Su input $\langle M,w \rangle$, una codifica di una TM $M$ ed una stringa $w$:
\begin{enumerate}
    \item Usa la descrizione di $M$ e $w$ per costruire la TM $M_1$ descritta sopra.
    \item Esegue la TM $R$ su input $\langle M_1 \rangle$.
    \item Se $R$ accetta, \textit{rifiuta}; se $R$ rifiuta, \textit{accetta}."
\end{enumerate}
Notate che $S$ deve essere effettivamente in grado di calcolare una descrizione di $M_1$ da una descrizione di $M$ e $w$.
Può farlo, perchè ha bisogno solo di aggiungere ad $M$ alcuni stati in più che svolgono il test $x = w$.
Se $R$ fosse un decisore per $E_{TM}$, allora $S$ sarebbe un decisore per $A_{TM}$.
Un decisore per $A_{TM}$ non può esistere, quindi $E_{TM}$ non può essere deciso.

Un altro interessante problema computazionale che concerne le macchine di Turing è quello di determinare se una data macchina di Turing riconosce un linguaggio che può essere riconosciuto anche da un modello di calcolo più semplice.
Ad esempio, sia $REGULAR_{TM}$ il problema di determinare se una machina di Turing ha un automa finito equivalente.
Questo problema equivale a determinarre se la macchina di Turing riconosce un linguaggio regolare.
Sia
$$
REGULAR_{TM} = \{\langle M \rangle \mid M \text{ è una TM e } L(M) \text{ è un linguaggio regolare} \}.
$$
\paragraph{Teorema 5.3}
\label{teorema-5.3}
\text{}
\newline
$REGULAR_{TM}$ è indecidibile.
\vspace{1em}
\text{}
\newline
\hbox{\textbf{IDEA.}} 
Come al solito per i teoremi sull'indecidibilità, questa dimostrazione consiste in una riduzione da $A_{TM}$.
Assumiamo che $REGULAR_{TM}$ sia deciso da una RM $R$ e utilizziamo quesra assunzione per costruire una TM $S$ che decide $A_{TM}$.
Questa volta risulta meno ovio come utilizzare la capacità di $R$ di assistere $S$ nel suo compito.
Tuttavia siamo in grado di farlo.
L'idea è che $S$ prenda il suo input $\langle M,w \rangle$ e modifichi $M$ in modo che la risultante TM riconosca un linguaggio regolare se e solo se $M$ accetta $w$.
Chiamiamo $M_2$ la macchina così modificata. Progettiamo $M_2$ in modo che riconosca il linguaggio non regolare $\{0^n1^n \mid n \geq 0\}$ se $M$ non accetta $w$, e riconosca il linguaggio regolare $\Sigma^*$ se $M$ accetta $w$.
Dobbiamo specificare come $S$ può costruire una tale $M_2$ da $M$ e $w$.
Qui, $M_2$ accetta automaticamente tutte le stringhe in $\{0^n1^n \mid n \geq 0\}$.
Inoltre, se $M$ accetta $w$, $M_2$ accetta tutte le altre stringhe.

Notate che la TM $M_2$ \textit{non} è costruita con lo scopo di essere eseguita su qualche input - un malinteso comune.
Costruiamo $M_2$ al solo scopo di dare in input la sua descrizione al decisore per $REGULAR_{TM}$ che abbiamo assunto esistere.
Quando tale decisore dà la sua risposta, possiamo usarla per rispondere se $M$ accetta $w$ o meno.
Quindi possiamo decidere $A_{TM}$, una contraddizione.
\vspace{1em}
\text{}
\newline
\hbox{\textbf{DIMOSTRAZIONE.}}
Definiamo $R$ come una TM che decide $REGULAR_{TM}$ e costruiamo una TM $S$ che decide $A_{TM}$.
Allora $S$ funziona come segue.
\vspace{1em}
\text{}
\newline
$S = $ "Su input $\langle M,w \rangle$, una codifica di una TM $M$ ed una stringa $w$:
\begin{enumerate}
    \item Costruisce la TM $M_2$ come segue.\newline $M_2 =$ "Sull'input $x$:
    \begin{enumerate}
        \item Se $x$ ha la forma $0^n1^n$ per qualche $n \geq 0$, \textit{accetta}.
        \item Se $x$ non ha tale forma, esegui $M$ su $w$ e \textit{accetta} se $M$ accetta."
    \end{enumerate}
    \item Esegue $R$ su input $\langle M_2 \rangle$.
    \item Se $R$ accetta, \textit{accetta}; se $R$ rifiuta, \textit{rifiuta}."
\end{enumerate}
\vspace{1em}
\text{}
\newline
Con prove simili si può dimostrare che i problemi di testare se il linguaggio di una macchina di Turing è un linguaggio context-free, un linguaggio decidibile o anche un linguaggio dinito sono indecidibili.
Infatti, un risultato generale, chiamato teorema di Rice,
\begin{figure}[H]
    \centering
    \includegraphics[width=0.4\textwidth]{Immagini/81.png}
    \caption{Teorema di Rice}
    \label{fig:rice_theorem}
\end{figure} 
afferma che determinare una \textit{qualsiasi proprietà} dei linguaggi riconosciuti da macchine di Turing è indecidibile.

Finora, la nostra strategia per dimostrare l'indecidibilità di linguaggi prevede una riduzione da $A_{TM}$.
A volte la riduzione da qualche altro linguaggio indecidibile è più conveniente per dimostrare che certi linguaggi sono indecidibili.
Il teorema 5.4 mostra che verificare l'equivalenza di due macchine di Turing è un problema indecidibile.
Potremmo provarlo con una riduzione da $A_{TM}$, ma abbiamo l'occasione di dare un esempio di una prova di incecidibilità mediante una riduzione da $E_{TM}$.
Sia
$$
EQ_{TM} = \{\langle M_1,M_2 \rangle \mid M_1 \text{ e } M_2 \text{ sono TM e } L(M_1) = L(M_2) \}.
$$
\paragraph{Teorema 5.4(orale)}
\label{teorema-5.4}
\text{}
\newline
$EQ_{TM}$ è indecidibile.
\vspace{1em}
\text{}
\newline
\hbox{\textbf{IDEA.}}
Mostriamo che se $EQ_{TM}$ fosse decidibile, allora $E_{TM}$ sarebbe decidibile eseguendo una riduzione da $E_{TM}$ a $EQ_{TM}$.
L'idea è semplice.
$E_{TM}$ è il problema di determinare se il linguaggio di una TM è vuoto.
$EQ_{TM}$ è il problema di determinare se i linguaggi delle due TM sono uguali.
Se uno di questi linguaggi è $\emptyset$, ci troviamo con il problema di determinare se il linguaggio dell'altra macchina è vuoto, cioè il problema $E_{TM}$.
Quindi, in un certo senso, il problema $E_{TM}$ è un caso speciale del problema $EQ_{TM}$ dove una delle macchine è fissata per riconoscere il linguaggio vuoto.
Questa idea rende la riduzione più facile.
\vspace{1em}
\text{}
\newline
\hbox{\textbf{DIMOSTRAZIONE.}}
Consideriamo una TM $R$ che decide $EQ_{TM}$ e costruiamo una TM $S$ che decide $E_{TM}$ come segue.
\vspace{1em}
\text{}
\newline
$S = $ "Su input $\langle M \rangle$, dove $M$ è una TM:
\begin{enumerate}
    \item Esegue $R$ su input $\langle M,M_1 \rangle$, dove $M_1$ è una TM che rifiuta ogni input.
    \item Se $R$ accetta, \textit{accetta}; se $R$ rifiuta, \textit{rifiuta}."
\end{enumerate}
\vspace{1em}
\text{}
\newline
Se $R$ decide $EQ_{TM}$, allora $S$ decide $E_{TM}$. Ma il \hyperref[teorema-5.2]{\textcolor{blue}{Teorema 5.2}} ci dice che $E_{TM}$ è indecidibile, quindi $EQ_{TM}$ deve essere indecidibile.

\subsection{Riducibilità mediante funzione}
Abbiamo mostrato come utilizzare la tecnica della riducibilità per dimostrare che alcuni problemi sono indecidibili.
In questa sezione formalizziamo il concetto di riducibilità.
Questo ci permette di usare la riducibilità in modo più raffinato, come per esempio per dimostrare che alcuni linguaggi non sono Turing-riconoscibili e per applicazioni in teoria della complessità.
La nozione di ridurre un problema ad un altro può essere definita formalmente in vari modi.
La scelta di quale usare dipende dall'applicazione. La nostra scelta ricade su un tipo semplice di riducibilità chiamato \textit{riducibilità mediante funzione}\footnote{E' chiamata riducibilità molti-a-uno in altri libri di testo}.

Informalmente, essere in grado di ridurre il problema $A$ al problema $B$ utilizzando una riduzione mediante funzione significa che esiste una funzione calcolabile che trasforma istanze del problema $A$ in istanze del problema $B$.
Se abbiamo una tale funzione, detta \textit{riduzione}, siamo in grado di risolvere $A$ risolvendo istanze di $B$.
Il motivo risiede nel fatto che una qualsiasi istanza di $A$ può essere risolta utilizzando prima la riduzione per convertirla in un'istanza di $B$ e poi risolvere tale istanza di $B$.
Nel breve daremo una definizione precisa di riducibilità mediante funzione.

\subsubsection{Funzioni calcolabili}
Una macchina di Turing calcola una funzione iniziando con l'input della funzione sul nastro e terminando con l'output della funzione sul nastro.

\begin{tcolorbox}[title=Definizione 5.17]
    Una funzione $f : \Sigma^* \rightarrow \Sigma^*$ è una \textbf{\textit{funzione calcolabile}} se esiste una macchina di Turing $M$ che, su qualsiasi input $w$, si ferma avendo solo $f(w)$ sul nastro.
\end{tcolorbox}

\begin{figure}[H]
    \centering
    \includegraphics[width=0.4\textwidth]{Immagini/82.png}
    \label{fig:reducibility_function}
\end{figure}
\begin{figure}[H]
    \centering
    \includegraphics[width=0.4\textwidth]{Immagini/83.png}
    \label{fig:reducibility_function}
\end{figure}

\subsubsection*{Definizione formale di riducibilità mediante funzione}
Definiamo ora la riducibilità mediante funzione. Come al solito, rappresentiamo problemi computazionali mediante linguaggi.

\begin{tcolorbox}[title=Definizione 5.120]
Un linguaggio $A$ si dice \textbf{\textit{riducibile mediante funzione}} a un linguaggio $B$, scritto $A \leq_m B$, se esiste una funzione calcolabile $f : \Sigma^* \rightarrow \Sigma^*$ tale che per ogni $w$,
$$
w \in A \text{ sse } f(w) \in B.
$$
La funzione $f$ è chiamata \textbf{\textit{riduzione}} da $A$ a $B$.
\end{tcolorbox}
La figura seguente illustra la riducibilità mediante funzione.
\begin{figure}[H]
    \centering
    \includegraphics[width=0.3\textwidth]{Immagini/84.png}
    \label{fig:reducibility_function}
\end{figure}

Una riduzione mediante funzione da $A$ a $B$ fornisce un modo per convertire problemi di appartenenza ad $A$ in problemi di appartenenza a $B$.
Per verificare se $w \in A$, usiamo la riduzione $f$ per mappare $w$ in $f(w)$ e verifichiamo se $f(w) \in B$ o meno.
Il termine \textit{riduzione mediante funzione} deriva dalla funzione che viene utilizzata per la riduzione.

Se un problema è riducibile mediante funzione ad un secondo problema già precedentemente risolto, possiamo ottenere da questo una soluzione al problema originale.

\paragraph{Teorema 5.22}
\label{teorema-5.22}
\text{}
\newline
Sia $A \leq_m B$ e $B$ è decidibile, allora A è decidibile.
\vspace{1em}
\text{}
\newline
\hbox{\textbf{DIMOSTRAZIONE.}}
Siano $M$ il decisore per $B$ ed $f$ la riduzione da $A$ a $B$.
Descriviamo un decisore $N$ per $A$ come segue.
\vspace{1em}
\text{}
\newline
$N = $ "Su input $w$:
\begin{enumerate}
    \item Computa $f(w)$.
    \item Esegue $M$ su input $f(w)$ e restituisce lo stesso output di $M$."
\end{enumerate}
Ovviamente, se $w \in A$, allora $f(w) \in B$ perchè $f$ è una riduzione da $A$ a $B$.
Quindi $M$ accetta $f(w)$ ogni volta che $w \in A$.
Quindi, $N$ svolge il compito desiderato.

Il seguente corollario al \hyperref[teorema-5.22]{\textcolor{blue}{Teorema 5.22}} rappresenta il nostro strumento principale nelle dimostrazioni di indecidibilità.
\vspace{1em}
\text{}
\newline
\begin{tcolorbox}[title=Corollario 5.23]
    Se $A \leq_m B$ e $A$ non è decidibile, allora $B$ non è decidibile.
\end{tcolorbox}
Ora rivediamo alcune delle nostre prove precedenti che hanno utilizzato il metodo della riducibilità per ottenere esempi di riduzione mediante funzione.
\begin{figure}[H]
    \centering
    \includegraphics[width=0.4\textwidth]{Immagini/85.png}
    \includegraphics[width=0.4\textwidth]{Immagini/87.png}
    \label{fig:example}
\end{figure}
\includegraphics[width=0.3\textwidth]{Immagini/86.png}
\newpage
\paragraph{Teorema 5.28}
\label{teorema-5.28}
\text{}
\newline
\begin{tcolorbox}[title=Teorema 5.28]
    Se $A \leq_m B$ e $B$ è Turing-riconoscibile, allora $A$ è Turing-riconoscibile.
\end{tcolorbox}
\vspace{1em}
\text{}
\newline
\paragraph*{Corollario 5.29}
\label{corollario-5.29}
\text{}
\newline
\begin{tcolorbox}[title=Corollario 5.29]
    Se $A \leq_m B$ e $A$ non è Turing-riconoscibile, allora $B$ non è Turing-riconoscibile.
\end{tcolorbox}
In una tipica applicazione di questo corollario, supponiamo che $A$ sia $\overline{A_{TM}}$, il complemento di $A_{TM}$.
Dal corollario \hyperref[corollario-4.23]{\textcolor{blue}{Corollario 4.23}}, sappiamo che non è Turing-riconoscibile.
La definizione di riduzione mediante funzione implica che $A \leq_m B$ ha lo stesso significato di $\overline{A} \leq_m \overline{B}$.
Per dimostrare che $B$ non è riconoscibile possiamo mostrare che $A_{TM} \leq_m B$.
Possiamo anche usare la riducibilità mediante funzione per dimostrare che alcuni problemi non sono nè Turing-riconoscibili nè co-Turing-riconoscibili, come nel seguente teorema.
\vspace{1em}
\text{}
\newline
\paragraph{Teorema 5.30}
\label{teorema-5.30}
\text{}
\newline
$EQ_{TM}$ non è ne Turing-riconoscibile nè co-Turing-riconoscibile.
\vspace{1em}
\text{}
\newline
\hbox{\textbf{DIMOSTRAZIONE.}}
Come primo passo, dimostriamo che $EQ_{TM}$ non è Turing-riconoscibile.
Lo facciamo dimostrando che $A_{TM}$ è riducibile a $\overline{EQ_{TM}}$.
La funzione di riduzione $f$ opera come segue.
\vspace{1em}
\text{}
\newline
$F =$ "Su input $\langle M,w \rangle$, dove $M$ è una TM e $w$ una stringa:
\begin{enumerate}
    \item Costruisce le seguenti due macchine, $M_1$ e $M_2$.\newline $M_1 =$ "Su ogni input: \begin{enumerate}
        \item \textit{Rifiuta}." \end{enumerate}$M_2 =$ "Su ogni input: \begin{enumerate}
        \item Esegue $M$ su $w$. Se accetta, \textit{accetta}." \end{enumerate}
    \item Restituisce $\langle M_1,M_2 \rangle$."
\end{enumerate}
Qui, $M_1$ non accetta alcuna stringa.
Se $M$ accetta $w$, $M_2$ accetta qualsiasi stringa, quindi le due macchine non sono equivalenti.
Vicecersa, se $M$ non accetta $w$, $M_2$ accetta solo la stringa vuota, quindi le due macchine sono equivalenti.
Perciò $f$ riduce $A_{TM}$ a $\overline{EQ_{TM}}$, come desiderato.

Per dimostrare che $\overline{EQ_{TM}}$ non è Turing-riconoscibile diamo una riduzione da $A_{TM}$ al complemento di $\overline{EQ_{TM}}$ cioè, $EQ_{TM}$.
In questo modo dimostriamo che $A_{TM} \leq_m EQ_{TM}$. La seguente $TM$ $G$ calcola la funzione di riduzione $g$. 
\vspace{1em}
\text{}
\newline
$G =$ "Su input $\langle M,w \rangle$, dove $M$ è una TM e $w$ una stringa:
\begin{enumerate}
    \item Costruisce le seguenti due macchine, $M_1$ e $M_2$.\newline $M_1 =$ "Su ogni input: \begin{enumerate}
        \item \textit{Accetta}." \end{enumerate}$M_2 =$ "Su ogni input: \begin{enumerate}
        \item Esegue $M$ su $w$. Se accetta, \textit{accetta}." \end{enumerate}
    \item Restituisce $\langle M_1,M_2 \rangle$."
\end{enumerate}
La sola differenza tra $f$ e $g$ si trova nella macchina $M_1$.
In $f$, la macchina $M_1$ rifiuta sempre, mentre in $g$ accetta sempre.
Sia in $f$ che in $g$, $M$ accetta $w$ sse $M_2$ accetta ogni input.
In $g$, $M$ accetta $w$ sse $M_1$ ed $M_2$ sono equivalenti.
Questo è il motivo per cui $g$ è una riduzione da $A_{TM}$ a $EQ_{TM}$.
\section{}
Serve per mantenere coerenza con il numero delle dimostrazioni da sapere indicate nel pdf (7.1, ecc...).
\section{Complessità di tempo}
Anche quando un problema è decidibile, e quindi in linea di principio computazionalmente risolvibile, può non essere risolvibile in pratica se la soluzione richiede una quantità eccessiva di tempo o di memoria.

Il nostro obiettivo in questo capitolo è presentare le basi della teoria della complessità di tempo.

\subsection{Misure di complessità}
Iniziamo con un esempio.
Consideraimo il linguaggio $A = \{0^k1^k \mid k \geq 0\}$.
Ovviamente, $A$ è decidibile.
Di quanto tempo necessita una macchina di Turing a nastro singllo per decidere $A$?
\newline
Esaminiamo la seguente TM $M_1$ per $A$.
Diamo la descrizione della macchina di Turing a basso livello, includendo l'esatto movimento della testina sul nastro in modo che possiamo contare il numero di passi che $M_1$ effettua quando lavora.
\vspace{1em}
\text{}
\newline
$M_1 =$ "Su input $w$:
\begin{enumerate}
    \item Scandisce il nastro e \textit{rifiuta} se trova uno $0$ a destra di un $1$.
    \item Ripete se il nastro contiene almeno un $0$ e un $1$:
    \item \quad Scandisce il nastro, cancellando uno $0$ e un $1$.
    \item Se rimane almeno uno $0$ dopo che ogni simbolo $1$  è stato cancellato, o se rimane almeno un $1$ dopo che ogni simbolo $0$ è stato cancellato, \textit{rifiuta}. Altrimenti, se non rimanfono nè simboli $0$ nè simboli $1$, \textit{accetta}."
\end{enumerate}
\textit{Analizzeremo} l'algoritmo della TM $M_1$ che decide $A$ per determinare la quantità di tempo che impiega.
Come prima cosa, introduciamo la terminologia e la notazione necessaria.
Il numero di passi che utilizza un algoritmo su un particolare input può dipendere da diversi parametri.
Per semplicità si calcola il tempo di esecuzione di un algoritmo semplicemente in funzione della linghezza della stringa che rappresenta l'input e non si conssiderano eventuali altri paramentri.
Nell'analisi del \textit{\textbf{caso peggiore}}, che noi considereremo, si valuta il tempo di esecuzione massimo tra tutti gli input di una determinata lunghezza.
Nell'analisi del \textit{\textbf{caso medio}}, si considera la media dei tempi di esecuzione du tutti gli inpud di una determinata lunghezza.
\newline
\subsubsection*{Definizione 7.1}
\label{definizione-7.1}
\begin{tcolorbox}[colback=yellow!10!white, colframe=yellow!50!black, title=Definizione 7.1]
    Sia $M$ una macchina di Turing deterministica che si ferma su tutti gli input. 
    Il \textbf{\textit{tempo di esecuzione}} o la \textbf{\textit{complessità di tempo}} di $M$ è la funzione $f : \mathbb{N} \rightarrow \mathbb{N}$, dove $f(n)$ è il massimo numero di passi che $M$ impiega su un input di lunghezza $n$.
    Se $f(n)$ è il tempo di esecuzione di $M$, diciamo che $M$ ha tempo di esecuzione $f(n)$ e che $M$ è una macchina di Turing di tempo $f(n)$.
    Abitualmente usiamo $n$ per rappresentare la lunghezza dell'input.
\end{tcolorbox}

\subsubsection{Notazione O-grande ed o-piccola}
Poichè il tempo esatto di esecuzione di un algoritmo è spesso un'espressione complessa, solitamente ci limitiamo ad ottenerne una stima.
L'utile metodo di stima, detto \textit{\textbf{analisi asintotica}}, permette di valutare il tempo di esecuzione dell'algoritmo quando viene eseguito su grandi input.
Lo facciamo prendendo in considerazione solo il termine di ordine maggiore dell'espressione del tempo di esecuzione dell'algoritmo, trascurando sia il coefficiente di tale termine, che tutti i termini di ordine inferiore, perchè il termine di ordine più alto domina gli altri termini quando l'input è grande.

Ad esempio, la funzione $f(n) = 6n^3 + 2n^2 + 20n + 45$ ha quattro termini ed il termine di ordine maggiore è $6n^3$.
Trascurando il coefficiente $6$, diciamo che $f$ è asintoticamente al più $n^3$.
La \textit{\textbf{notazione asintotica}} o notazione \textit{\textbf{O-grande}} per descrivere questo rapporto è $f(n) = O(n^3)$.

Formalizziamo tale nozione nella seguente definizione. Sia $\mathbb{R}^+$ l'insieme dei numeri reali non negativi.
\newline
\subsubsection*{Definizione 7.2}
\label{definizione-7.2}
\begin{tcolorbox}[colback=yellow!10!white, colframe=yellow!50!black, title=Definizione 7.2]
    Siano $f$ e $g$ funzioni da $f,g : \mathbb{N} \rightarrow \mathbb{R}^+$.
    Si dice che $f(n) = O(g(n))$ se esistono interi positivi $c$ e $n_0$ tali che per ogni $n \geq n_0$,
    $$
    f(n) \leq cg(n).
    $$
    Quando $f(n) = O(g(n))$, diciamo che $g(n)$ è un \textbf{\textit{limite superiore}} per $f(n)$, o più precisamente, che $g(n)$ è un limite superiore asintotico per $f(n)$, per sottolineare che stiamo ignorando le costanti.
\end{tcolorbox}
Intuitivamente, $f(n) = O(g(n))$ significa che $f$ è minore o uguale a $g$ se trascuriamo differenze fino ad un fattore costante.
Potete pensare ad O come rappresentazione implicita di una costante.
In pratica la maggior parte delle funzioni $f$ che potete incontrare hanno un termine di ordine più alto $h$, chiaramente individuabile.
In tal caso si scrive $f(n) = O(g(n))$, dove $g$ è $h$ senza il suo coefficiente.
\newline
\begin{figure}[H]
    \centering
    \includegraphics[width=0.4\textwidth]{Immagini/88.png}
    \label{fig:example_O_grande}
\end{figure}
\begin{figure}[H]
    \centering
    \includegraphics[width=0.4\textwidth]{Immagini/89.png}
    \label{fig:example_O_grande}
\end{figure}
\begin{tcolorbox}[colback=yellow!10!white, colframe=yellow!50!black, title=Definizione 7.5]
    Siano $f$ e $g$ funzioni da $f,g : \mathbb{N} \rightarrow \mathbb{R}^+$.
    Diciamo che $f(n) = o(g(n))$ se
    $$
    \lim_{n \to \infty} \frac{f(n)}{g(n)} = 0.
    $$
    In altri termini, $f(n) = o(g(n))$ significa che per ogni numero reale $c > 0$, esiste un intero positivo $n_0$ tale che per ogni $n \geq n_0$, 
    $$
    f(n) < cg(n).
    $$
\end{tcolorbox}

\begin{figure}[H]
    \centering
    \includegraphics[width=0.4\textwidth]{Immagini/90.png}
    \label{fig:example_O_grande}
\end{figure}

\subsubsection{Analisi degli algoritmi}
Analizziamo l'algoritmo $M_1$ dato per il linguaggio $A = \{0^k1^k \mid k \geq 0 \}$.
Riscriviamo qui l'algoritmo oer comodità.
\vspace{1em}
\text{}
\newline
$M_1 =$ "Su input $w$:
\begin{enumerate}
    \item Scandisce il nastro e \textit{rifiuta} se trova uno $0$ a destra di un $1$.
    \item Ripete se il nastro contiene almeno un $0$ e un $1$:
    \item \quad Scandisce il nastro, cancellando uno $0$ e un $1$.
    \item Se rimane almeno uno $0$ dopo che ogni simbolo $1$  è stato cancellato, o se rimane almeno un $1$ dopo che ogni simbolo $0$ è stato cancellato, \textit{rifiuta}. Altrimenti, se non rimancono nè simboli $0$ nè simboli $1$, \textit{accetta}."
\end{enumerate}
\vspace{1em}
\text{}
Per analizzare $M_1$, consideriamo ciascuna delle sue quattro fasi separatamente.
Nella fase 1, la macchina scansiona il nastro per verificare che l'input è del tipo $0^*1^*$.
Tale operazione di scansione usa $n$ passi.
Come abbiamo accennato in precedenza, di solito utilizziamo $n$ per rappresentare la lunghezza dell'input.
Per riposizionare la testina all'estremità sinistra del nastro utilizza ulteriori $n$ passi.
Per cui il totale di passi utilizzati in questa fase è $2n$ passi.
Nella notazione O-grande diciamo che questa fase usa $O(n)$ passi.
Si noti che non abbiamo fatto menzione del riposizionamento della testina del nastro nella descrizione della macchina.
L'utilizzo della notazione asintotica ci permette di omettere quei dettagli della descrizione della macchina che influenzano il tempo di esecuzione al più di un fattore costante.

Nelle fasi 2 e 3, la macchina esegue ripetutamente la scansione del nastro e cancella uno $0$ e un $1$ ad ogni scansione.
Ogni scansione utilizza $O(n)$ passi.
Poichè ogni scansione elimina due simboli, possono verificarsi al più $n/2$ scansioni.
Così il tempo totale impiegato dalle fasi 2 e 3 è di $(n/2)O(n) = O(n^2)$ passi.

Nella fase 4 la macchina fa una singola scansione per decidere se accettare o rifiutare.
Il tempo impiegato in questa fase è $O(n)$ passi.

Quindi il tempo totale di $M_1$ su un input di lunghezza $n$ è $O(n) + O(n^2) + O(n)$, ovvero $O(n^2)$.
In altre parole, il tempo di esecuzione è $O(n^2)$, il che completa l'analisi del tempo di esecuzione di questa macchina.

Vogliamo ora stabilire una notazione per classificare i linguaggi in base alla loro necessità di tempo.
\newline
\subsubsection*{Definizione 7.7}
\label{definizione-7.7}
\begin{tcolorbox}[colback=yellow!10!white, colframe=yellow!50!black, title=Definizione 7.7]
    Sia $t : \mathbb{N} \rightarrow \mathbb{R}^+$ una funzione.
    La \textbf{\textit{classe di complessità di tempo}} $\textbf{TIME(t(n))}$, è definita come l'insieme di tutti i linguaggi che sono decisi da una macchina di Turing in tempo $O(t(n))$.
\end{tcolorbox}
\vspace{1em}
\text{}
\newline
Ricordiamo il linguaggio $A = \{0^k1^k \mid k \geq 0\}$.
L'analisi precedente mostra che $A \in TIME(n^2)$ perchè $A$ è deciso dalla macchina di Turing $M_1$ in tempo $O(n^2)$.

Esiste una macchina di turing che decide $A$ in modo asintoticamente più veloce?
\newline
In altri termini, risulta $A$ in  $TIME(t(n))$ per $t(n) = o(n^2)$?
Possiamo migliorare il tempo di esecuzione, cancellando due simboli 0 e due simboli 1 ad ogni scansione invece di uno solamente, perchè questo riduce il numero di scansioni della metà.
Ma ciò migliora il tempo di esecuzione solo per un fattore 2 e non influenza il tempo di esecuzione asintotico.
La seguente macchina $M_2$, utilizza un metodo differente per decidere $A$ asintoticamente più velocemente.
Essa mostra che $A \in TIME(n logn)$.
\vspace{1em}
\text{}
\newline
$M_2 =$ "Su input $w$:
\begin{enumerate}
    \item Scandisce il nastro e \textit{rifiuta} se trova uno $0$ a destra di un $1$.
    \item Ripete se il nastro contiene almeno un $0$ e un $1$:
    \item \quad Scandisce il nastro, controllando se il numero totale di simboli 0 e di 1 rimasti è pari o dispari. Se è dispari, \textit{rifiuta}.
    \item \quad Scandisce nuovamente il nastro, cancellando prima ogni secondo 0 a partire dal primo 0, poi cancellando ogni secondo 1 a partire dal primo 1.
    \item Se nessuno 0 e nessun 1 rimangono sul nastro, \textit{accetta}. Altrimenti, \textit{rifiuta}."
\end{enumerate}

Prima di analizzare $M_2$, cerchiamo di verificare se effettivamente decide $A$.
In ogni scansione eseguitta nella fase 4, il numero totale di 0 rimanenti è ridotto della metà e ogni eventiale resto viene scartato.
Quindi, se abbiamo iniziato con 13 simboli 0, dopo una prima esecuzione della fase 4 rimangono solo 6 simboli 0.
Dopo le successive esecuzioni di questa fase ne restano 3, 1 e infine 0.
Questa fase ha lo stesso effetto sul numero di simboli 1.
Esaminiamo ora la parità del numero di simboli 0 e 1 ad ogni esecuzione della fase 3.
Si consideri ancora l'inizio con 13 simboli 0 e 13 simboli 1.
La prima esecuzione della fase 3 trova un numero dispari di simboli 0 (perchè 13 è un numero dispari) ed un numero dispari di simboli 1.
Nelle successive esecuzioni si ha un numero pari (6), poi un numero dispari (3), e un numero dispari (1).
Non eseguiamo questa fase su 0 simboli 0 o 0 simboli 1 in accordo alla condizione specificata nella fase 2.
Per la sequenza di parità trovata (dispari, pari, dispari, dispari), se sostituiamo pari con 0 e dispari con 1 e poi invertiamo la sequenza, otteniamo 1101, la rappresentazione binaria di 13, ossia il numero di simboli 0 e 1 iniziale.
La sequenza delle parità fornisce sempre l'inverso della rappresentazione binaria.

Quando la fase 3 controlla che il numero totale di 0 e 1 rimanenti è pari, in realtà sta controllando la coerenza della parità del numero di 0 con la parità del numero di 1.
Se le parità corrispondono, le rappresentazioni binarie dei numeri di 0 e di 1 corrispondono, e quindi i due numeri sono uhuali.

Per analizzare il tempo di esecuzione di $M_2$, per prima cosa osserviamo che ogni fase impiega un tempo $O(n)$.
Determiniamo poi il numero di volte in cui ognuna viene eseguita.
Le fasi 1 e 5 vengono eseguite una volta, impiegando un tempo totale di $O(n)$.
La fase 4 scarta almeno metà dei simboli 0 e 1 ogni volta che viene eseguita, quindi si verificano al massimo $1 + log_2 n$ iterazioni del ciclo prima di averli cancellati tutti.
Così il tempo totale delle fasi 2,3, e 4 è $(1 + log_2 n)O(n)$, o $O(nlogn)$. Il tempo di esecuzione di $M_2$ è $O(n) + O(nlogn) = O(nlogn)$.

In precedenza abbiamo dimostrato che $A \in TIME(n^2)$, ma ora abbiamo ottenuto un limite migliore - precisamente, $A \in TIME(nlogn)$.
Questo risultato non può essere ulteriormente migliorato su macchine di Turing a singolo nastro.
Infatti, ogni linguaggio che può essere deciso in tempo $o(nlogn)$ su una macchina di Turing a nastro singolo è regolare.

Possiamo decidere il linguaggio $A$ in tempo $O(n)$ (chiamato anche tempo lineare) se la macchina di Turing ha un secondo nastro.
La seguente TM $M_3$ a due nastri decide $A$ in tempo lineare.
La macchina $M_3$ opera diversamente dalle macchine precedenti per $A$.
Semplicemente copia tutti i simboli 0 sul suo secondo nastro e poi li accoppia con gli 1.
\vspace{1em}
\text{}
\newline
$M_3 =$ "Su input $w$:
\begin{enumerate}
    \item Scandisce il nastro 1 e \textit{rifiuta} se trova uno 0 a destra di un 1. 
    \item Scandisce i simboli 0 sul nastro 1 fino al primo 1. Contemporaneamente, copia ogni 0 sul nastro 2.
    \item Scandisce i simboli 1 sul nastro 1 fino alla fine dell'input.\newline Per ogni 1 letto sul nastro 1, cancella uno 0 sul nastro 2.\newline Se ogni 0 è stato cancellato prima di aver letto tutti gli 1, \textit{rifiuta}.
    \item Se tutti gli 0 sono stati cancellati \textit{accetta}. Se rimane qualche 0, \textit{rifiuta}."
\end{enumerate} 

Questa macchina è semplice da analizzare. 
Ciascuna delle quattro fasi utilizza $O(n)$ passi, in modo che il tempo di esecuzione complessivo risulta $O(n)$ e quindi lineare.
Si noti che questo tempo di esecuzione è il migliore possibile perché $n$ passi sono necessari semplicemente per leggere l'input.

Riassumiamo quello che abbiamo dimostrato circa la complessità temporale di $A$, la quantità di tempo necessaria per decidere $A$.
Abbiamo progettato una TM $M_1$ a nastro singolo che decide $A$ in tempo $O(n^2)$ ed una TM $M_2$ a nastro singolo che decide $A$ in tempo $O(nlogn)$.
Poi abbiamo progettato una TM $M_3$ a due nastri che decide $A$ in tempo $O(n)$.
Quindi la complessità di tempo di $A$ è $O(nlogn)$ si una TM a nastro singolo e $O(n)$ su una TM a due nastri.
Notate che la complessità di $A$ dipende dal modello di calcolo scelto.

Questa discussione evidenzia una differenza importante tra la teoria della complessità e la teoria della computabilità.
Nella teoria della computabilità, la tesi di Church-Turing implica che tutti i modelli computazionali sono equivalenti - ossia che decidono la stessa classe di linguaggi.
Nella teoria della complessità, la scelta del modello influisce sulla complessità di tempo dei linguaggi.
Nella teoria della complessità, classifichiamo i problemi computazionali secondo la loro complessità di tempo.
Ma con quale modalità misuriamo il tempo? 
Uno stesso linguaggio può avere differenti necessità di tempo in modelli diversi. 
Fortunatamente, i requisiti di tempo non differiscono molto per i modelli deterministici usuali.
Quindi, se il nostro sistema di classificazione non è molto sensibile a differenze relativamente piccole nella complessità, la scelta dello specifico modello deterministico non è importante.

\subsubsection{Relazioni di complessità tra modelli}
Qui esaminiamo come la scelta del modello di calcolo può influenzare la complessità di tempo dei linguaggi.
Consideriamo tre modelli: macchine di Turing a singolo nastro, macchine di Turing a due nastri e macchine di Turing non deterministiche.
\vspace{1em}
\paragraph{Teorema 7.8 (orale 24-30)}
\label{teorema-7.8}
\text{}
\newline
\begin{tcolorbox}[colback=yellow!10!white, colframe=yellow!50!black, title=Teorema 7.8]
    Sia $t(n)$ una funzione, tale che $t(n) \geq n$.
    Ogni macchina di Turing multinastro di tempo $t(n)$ ammette una macchina di Turing equivalente a nastro singolo di tempo $O(t^2(n))$.
\end{tcolorbox}

\hbox{\textbf{IDEA.}}
L'idea alla base della dimostrazione di questo teorema è molto semplice.
Ricordiamo che nel \hyperref[teorema-3.13]{\textcolor{blue}{Teorema 3.13}} abbiamo dimostrato che ogni macchina di Turing multinastro può essere simulata da una macchina di Turing a nastro singolo.
Ora analizziamo la simulazione per determinare la quantità di tempo supplementare richiesta.
Mostriamo che possiamo simulare ogni passo della macchina multinastro con $O(t(n))$ passi della macchina a nastro singolo.
\newline
Per cui il tempo totale impiegato è $O(t^2(n))$.
\vspace{1em}

\hbox{\textbf{DIMOSTRAZIONE.}}
Sia $M$ una TM a $k$-nastri avente tempo di esecuzione $t(n)$.
Costruiamo una TM $S$ a singolo nastro che ha tempo di esecuzione $O(t^2(n))$.
La macchina $S$ opera simulando $M$, come descritto nel \hyperref[teorema-3.13]{\textcolor{blue}{Teorema 3.13}}.
Nel rivedere tale simulazione, ricordiamo che $S$ utilizza il suo unico nastro per rappresentare il contenuto di tutti i $k$ nastri di $M$.
I nastri sono memorizzati consecutivamente, con le posizioni delle testine di $M$ masrcate nelle celle appropriate,
Inizialmente,$S$ mette il nastro nel formato che rappresenta tutti i nastri di $M$ e poi simula i passi di $M$.
Per simulare un passo, $S$ scorre tutte le informazioni memorizzate sul suo nastro per determinare i simboli presenti sotto le testine di $M$.
Poi $S$ esegue un'altra scansione del suo nastro per aggiornare il contenuto dei nastri e le posizioni delle testine.
Se una testina di $M$ si sposta a destra su una parte non letta del nastro, $S$ deve aumentare la quantità di spazio allocato per questo nastro.
Lo fa spostando una parte del suo nastro di una cella a destra.

Ora analizziamo questa simulazione.
Per ogni passo di $M$, la macchina $S$ fa due passi sulla parte attiva del suo nastro.
Il primo ottiene le informazioni necessarie per determinare la prossima mossa e il secondo la esegue.
La lunghezza della parte attiva del nastro di $S$ determina il tempo che $S$ impiega per eseguire la scansione, quindi dobbiamo determinare un limite superiore per questa lunghezza.
Per farlo prendiamo la somma delle lunghezze parti attive dei $k$ nastri di $M$.
Ciascuna di queste parti attive ha lunghezza al più $t(n)$, poichè $M$ utilizza $t(n)$ celle del nastro in $t(n)$ passi, se la testina si sposta verso destra ad ogni passo e anche meno se vi sono spostamenti di qualche testina a sinistra.
Quindi una scansione della parte di nastro attiva di $S$ impiega $O(t(n))$ passi.
Per simulare ciascuna delle fasi di $M$, $S$ esegue due scansioni ed eventualmente fino a $k$ spostamenti a destra.
Ognuno impiega un tempo $O(t(n))$, per cui il tempo totale per $S$ per simulare un passo di $M$ è $O(t(n))$.
Ora possiamo limitare il tempo totale impiegato dalla simulazione.
La fase iniziale, in cui $S$ mette il nastro nel formato corretto, usa $O(t(n))$ passi, per cui questa parte della simulazione utilizza $t(n) \times O(t(n)) = O(t^2(n))$ passi.
Quindi l'intera simulazione di $M$ utilizza $O(n) + O(t^2(n))$ passi.
Abbiamo assunto che $t(n) \geq n$ (un'ipotesi ragionevole perchè $M$ non potrebbe nemmeno leggere l'intero input in meno tempo). 
Pertanto il tempo di esecuzione di $S$ è $O(n) + O(t^2(n)) = O(t^2(n))$ e la dimostrazione è completa.
\vspace{1em}
\text{}
\newline
Ora vedremo un teorema analogo nel caso di macchine di Turing non deterministiche a nastro singolo.
Mostreremo che ogni linguaggio decidibile su tale macchina è anche decidibile su una macchina di Turing deterministica a nastro singolo che richiede molto più tempo.
Prima di farlo, dobbiamo definire il tempo di esecuzione di una macchina di Turing non deterministica.
Ricordiamo che una macchina di Turing non deterministica è un decisore se tutte le sue computazioni si fermano su tutti gli input.

\subsubsection{Definizione 7.9}
\label{definizione-7.9}
\begin{tcolorbox}[colback=yellow!10!white, colframe=yellow!50!black, title=Definizione 7.9]
    Sia $N$ una macchina di Turing non deterministica che sia anche un decisore.
    Il \textbf{\textit{tempo di esecuzione}} di $N$ è la funzione $f : \mathbb{N} \rightarrow \mathbb{N}$, tale che $f(n)$ è il massimo numero di passi che $N$ usa per ognuna delle computazioni su ogni input di òunghezza $n$, come mostrato nella figura seguente.
\end{tcolorbox}
\begin{figure}[H]
    \centering
    \includegraphics[width=0.4\textwidth]{Immagini/91.png}
    \caption{Misurazione del tempo nei casi deterministico e non deterministico.}
    \label{fig:example_O_grande}
\end{figure}

La definizione del tempo di esecuzione di una macchina di Turing non deterministica non è destinata a corrispondere ad un qualche dispositivo informatico reale.
Piuttosto, si tratta di un'utile definizione matematica che aiuta a caratterizzare la complessità di una classe importante di problemi computazionali, come dimostreremo a breve.

\subsubsection{Teorema 7.11 (orale 24-30)}
\label{teorema-7.11}
\begin{tcolorbox}[colback=yellow!10!white, colframe=yellow!50!black, title=Teorema 7.11]
    Sia $t(n)$ una funzione, tale che $t(n) \geq n$.
    Ogni macchina di Turing non deterministica a singolo nastro avente tempo di esecuzione $t(n)$ ammette una macchina di Turing deterministica a singolo nastro equivalente di tempo $2^{O(t(n))}$.
\end{tcolorbox}

\hbox{\textbf{DIMOSTRAZIONE.}}
Sia $N$ una TM non deterministica avente tempo di esecuzione $t(n)$.
Costruiamo una TM deterministica $D$ che simula $N$, come nel \hyperref[teorema-3.16]{\textcolor{blue}{Teorema 3.16}}, effettuando una ricerca sull'albero delle computazioni di $N$.
Ora analizziamo tale simulazione.
Su un input di lunghezza $n$, ogni ramificazione dell'albero delle computazioni si $N$ ha lunghezza al più $t(n)$.
Ogni nodo dell'albero può avere al più $b$ figli, dove $b$ è il massimo numero di scelte possibili in accordo alla funzione di transizione di $N$.
Così il numero totale di foglie nell'albero è al massimo $b^{t(n)}$.
La simulazione procede esplorando l'albero prima in ampiezza.
In altre parole, si visitano tutti i nodi a profondità $d$ prima di passare ad uno qualsiasi dei nodi a profondità $d + 1$.
L'algoritmo riportato nella dimostrazione del \hyperref[teorema-3.16]{\textcolor{blue}{Teorema 3.16}} inizia inefficientemente dalla radice e si sposta in basso verso un nodo ogni volta che visita il nodo stesso.
Tuttavia l'eliminazione di tale inefficienza non altera l'enunciato del Teorema, quindi la lasciamo in questa forma.
Il numero totale di nodi dell'albero è inferiore al doppio del numero di foglie, quindi è limitato da $O(b^{t(n)})$.
Il tempo per partire dalla radice e raggiungere un nodo è $O(t(n))$.
Pertanto il tempo di esecuzione di $D$ è $O(t(n)b^{t(n)}) = 2^{O(t(n))}$.

Come descritto nel \hyperref[teorema-3.16]{\textcolor{blue}{Teorema 3.16}}, la TM $D$ ha tre nastri.
Convertirla in una TM nastro singolo al più fa si che si elevi al quadrato il tempo di esecuzione, per il \hyperref[teorema-7.8]{\textcolor{blue}{Teorema 7.8}}.
Quindi il tempo di esecuzione del simulatore a nastro singolo è $(2^{O(t(n))})^2 = 2^{O(2t(n))} = 2^{O(t(n))}$ ed il teorema è dimostrato.

\subsection{La classe P}
I Teoremi \hyperref[teorema-7.8]{\textcolor{blue}{7.8}} e \hyperref[teorema-7.11]{\textcolor{blue}{7.11}} illustrano una distinzione importante.
Da una parte, abbiamo dimostrato una differenza al più quadratica o polinomiale tra le complessità di tempo dei problemi misurati su macchine di Turing deterministiche a singolo nastro e multinastro.
D'altra parte, abbiamo mostrato una differenza al più esponenziale tra la complessità temporale di problemi su macchine di Turing deterministiche e non deterministiche.

\subsubsection{Tempo polinomiale}
Diamo un'occhiata al motivo per cui abbiamo scelto di fare questa separazione tra polinomi ed esponenziali, piuttosto che tra le altre classi di funzioni.
\newline
Algoritmi aventi tempo polinomiale sono abbastanza veloci per molti scopi, ma algoritmi aventi tempo esponenziale sono raramente utili.
Algoritmi aventi tempo esponenziale si presentano in genere quando risolviamo problemi mediante una ricerca esaustiva nello spazio delle soluzioni, denominata \textit{\textbf{ricerca mediante forza bruta}}.
A volte, la ricerca mediante forza bruta può essere evitata attraverso una compensione più approfondita del problema, che può suggerire un algoritmo polinomiale di maggiore utilità.

Tutti i modelli computazionali deterministici ragionevoli sono \textit{\textbf{polinomialmente equivalenti}}.
Cioè, uno di essi può simularne un altro con aumento solo polinomiale del tempo di esecuzione.
Quando diciamo che tutti i modelli deterministici ragionevoli sono polinomialmente equivalenti, non cerchiamo di definire il termine \textit{ragionevole}.

Da qui in poi ci concentreremo sugli aspetti della teoria della complessità temporale che non sono influenzati da differenze polinomiali del tempo di esecuzione.
Ignorando tali differenze, possiamo sviluppare la teoria in un modo che non dipenda dalla scelta di un particolare modello di computazione.
Ricordiamo che il nostro obiettivo è presentare le proprietà fondamentali della \textit{computazione} , piuttosto che le proprietà delle macchine di Turing o di un qualsiasi altro modello particolare.

La nostra decisione di non tener conto delle differenze polinomiali non significa che consideriamo tali differenze non importanti.
Al contrario, certamente consideriamo importante la differenza tra il tempo $n$ ed il tempo $n^3$.
Ma alcune questioni, come ad esempio l'essere polinomiale o non polinomiale del problema della fattorizzazione, non dipendono da differenze di tipo polinomiale e sono ugualmente importanti.
Abbiamo semplicemente scelto di concentrarci su questo tipo di questioni.
\newline
Veniamo ora ad una definizione importante nella teoria della complessità.
\subsubsection{Definizione 7.12}
\label{definizione-7.12}
\begin{tcolorbox}[colback=yellow!10!white, colframe=yellow!50!black, title=Definizione 7.12]
    \textbf{P} è la classe di linguaggi che sono decidibili in tempo polinomiale su una macchina di Turing deterministica a singolo nastro. 
    In altre parole,
    $$
    P = \bigcup_{k} TIME(n^k).
    $$
\end{tcolorbox}
La classe P ha un ruolo centrale nella nostra teoria ed è importante perchè
\begin{enumerate}
    \item P è invariante per tutti i modelli di calcolo che sono polinomialmente equivalenti ad una macchina di Turing deterministica a nastro singolo, e
    \item P corrisponde approssimativamente alla classe dei problemi che sono realisticamente risolvibili su un computer.
\end{enumerate}
Il punto 1 indica che P è una classe matematicamente robusta.
Non è influenzata dai particolari del modello di computazione che stiamo usando.
\newline
Il punto 2 indica che P è importante da un punto di vista pratico.
Quando un problema è in P, abbiamo un metodo per risolverlo che viene eseguito in tempo $n^k$ per una costante $k$.
Se questo tempo di esecuzione risulta pratico dipende da $k$ e dall'applicazione.
Tuttavia, fissare la soglia di risolubilità in pratica al tempo polinomiale si è dimostrato utile.

\subsubsection{Esempi di problemi in P}
Quando analizziamo un algoritmo per dimostrare che esso viene eseguito in tempo polinomiale, abbiamo bisogno di fare due cose. 
In primo luogo, dobbiamo dare un limite superiore polinomiale (in genere nella notazione O-grande) sul numero di fasi che l'algoritmo esegue quando viene eseguito su un input di lunghezza $n$.
Poi, dobbiamo esaminare le singole fasi nella descrizione dell'algoritmo per essere sicuri che ognuna può essere implementata in tempo polinomiale su un modello deterministico ragionevole.
Utilizziamo le fasi nel descrivere l'algoritmo per rendere questa seconda
parte dell'analisi più semplice da eseguire.

Quando entrambi i compiti sono stati completati, possiamo concludere che l'algoritmo ha un tempo di esecuzione polinomiale perché abbiamo dimostrato che viene eseguito per un numero polinomiale di fasi, ciascuna delle quali può essere eseguita in tempo polinomiale, e la composizione di polinomi è ancora un polinomio.

Un punto che richiede attenzione è il metodo di codifica usato per i problemi. 
Continuiamo ad usare la notazione mediante parentesi angolari $\langle \cdot \rangle$ per indicare una codifica mediante una stringa ragionevole di uno o più oggetti, senza specificare un particolare metodo di codifica.

Molti dei problemi computazionali che incontrerete in questo capitolo contengono codifiche di grafi.
Una codifica ragionevole di un grafo consiste in un elenco dei suoi nodi ed archi. 
Un altro è la \textit{\textbf{matrice di adiacenza}}, dove l'elemento ($i, j$)-esimo è 1 se c'è un arco dal nodo $i$ al nodo $j$ e 0 se non c'è.
Quando analizziamo algoritmi su grafi, il tempo di esecuzione può essere calcolato in termini del numero di nodi invece della dimensione della rappresentazione del grafo.
In rappresentazioni ragionevoli di grafi, la dimensione della rappresentazione è un polinomio nel numero di nodi.
Così, se analizziamo un algoritmo e mostriamo che il suo tempo di esecuzione è polinomiale (o esponenziale) nel numero di nodi, sappiamo che esso è anche polinomiale (o esponenziale) nella dimensione dell'input.
Il primo problema riguarda grafi orientati. 
Un grafo orientato $G$ contiene nodi $s$ e $t$, come mostrato nella figura seguente. 

Il problema $PATH$ consiste nel determinare, se esiste, un cammino diretto $s$ a $t$. Sia
$$
PATH = \{ \langle G, s, t \rangle \mid G \text{ è un grafo orientato che contiene un cammino diretto da } s \text{ a } t \}.
$$
\begin{figure}[H]
    \centering
    \includegraphics[width=0.4\textwidth]{Immagini/92.png}
    \caption{Il problema $PATH$: esiste un cammino da $s$ a $t$?}
    \label{fig:directed_graph}
\end{figure}
\paragraph{Teorema 7.14}
\label{teorema-7.14}
\vspace{1em}
\text{}
\newline
$PATH \in P$.
\newline
\textbf{IDEA.}
Dimostriamo questo teorema presentando un algoritmo di tempo polinomiale che decide $PATH$.
Prima di descrivere l'algoritmo, notiamo che un algoritmo di forza bruta per questo problema non è abbastanza veloce. 
Un algoritmo di forza bruta per PATH procede esaminando tutti i potenziali cammini in $G$ e determinando se uno di essi è un cammino diretto da $s$ a $t$.
Un potenziale cammino è una sequenza di nodi in $G$ avente lunghezza al più $m$, dove $m$ è il numero di nodi in $G$.
(Se esiste un qualche cammino diretto da $s$ a $t$, ne deve esistere necessariamente almeno uno avente lunghezza al massimo $m$, perché non è mai necessario ripetere un nodo.) 
Tuttavia il numero di tali percorsi potenziali è approssimativamente $m^m$, che è esponenziale nel numero di nodi in $G$.
Pertanto questo algoritmo di forza bruta utilizza tempo esponenziale.
 
Per ottenere un algoritmo polinomiale per $PATH$, dobbiamo fare qualcosa che evita la forza bruta. 
Un modo è quello di utilizzare un metodo visita dei grafi quale la visita in ampiezza. 
In questo caso, contrassegniamo successivamente tutti nodi in $G$ che sono raggiungibili da $s$ mediante cammini diretti di lunghezza 1, poi 2, poi 3, fino ad $m$.
Possiamo facilmente limitare il tempo di esecuzione di questa strategia mediante un polinomio.
\vspace{1em}
\text{}
\newline
\textbf{DIMOSTRAZIONE.}
Un algoritmo in tempo polinomiale $M$ per $PATH$ procede come segue.
\newline
$M =$ "Su input $\langle G, s, t \rangle$, dove $G$ è un grafo orientato e $s$ e $t$ sono nodi di $G$:
\begin{enumerate}
    \item Marca il nodo $s$.
    \item Ripete il seguente passo fino a quando nessun nuovo nodo viene marcato:
    \item \quad Scandiona tutti gli archi di $G$. Se trova un arco $(a,b)$ che va da un nod $a$ marcato ad un nodo $b$ non marcato, marca il nodo $b$.
    \item Se il nodo $t$ è marcato, \textit{accetta}; altrimenti, \textit{rifiuta}."
\end{enumerate}
Ora analizziamo questo algoritmo per dimostrare che lavora in tempo polinomiale.
Ovviamente. le fasi 1 e 4 sono eseguite una sola volta. 
La fase 3 è eseguita al più $m$ volte perché ogni volta, tranne l'ultima, marca un nodo aggiuntivo in $G$.
Quindi il numero totale di fasi utilizzate è al massimo $1 + 1 + m$, che da un polinomio nella dimensione di $G$.
Le fasi 1 e 4 di $M$ sono facilmente implementate in tempo polinomiale
su un qualsiasi modello deterministico ragionevole. 
La fase 3 richiede una scansione dell'input ed un test che verifichi se certi nodi sono marcati o meno, ed è anch'essa facilmente implementabile in tempo polinomiale. 
Quindi $M$ è un algoritmo di tempo polinomiale per $PATH$.
\vspace{1em}
\text{}
\newline
Consideriamo ora un altro esempio di algoritmo avente tempo polinomiale.
Diciamo che due numeri sono relativamente primi se 1 è il più grande numero intero che li divide entrambi.
Per esempio, 10 e 21 sono relativamente primi, anche se nessuno dei due è esso stesso un numero primo, mentre 10 e 2 non sono relativamente primi perché entrambi sono divisibili per 2. 
Sia $RELPRIME$ il problema di verificare se due numeri sono relativamente primi. Quindi
$$
RELPRIME = \{ \langle a, b \rangle \mid a \text{ e } b \text{ sono relativamente primi} \}.
$$
\paragraph{Teorema 7.15}
\label{teorema-7.15}

\text{}
\newline
\textbf{IDEA.}
Un algoritmo che risolve questo problema effettua una ricerca tra tutti i divisori di entrambi i numeri e accetta se nessuno è maggiore di 1.
Tuttavia, la grandezza di un numero rappresentato in binario, o in qualsiasi altra notazione in base $k$ per $k \geq 2$, è esponenziale nella lunghezza della sua rappresentazione.
Quindi questo algoritmo di forza bruta ricerca attraverso un numero esponenziale di potenziali divisori e ha un tempo di esecuzione esponenziale.

Risolviamo invece questo problema con un antico procedimento numerico, chiamato \textit{algoritmo di Euclide}, per il calcolo del massimo comun divisore.
Il \textbf{\textit{massimo comun divisore}} dei numeri naturali $x$ e $y$, scritto gcd($x, y$), è il più grande numero intero che divide entrambi $x$ e $y$.
Ovviamente, $x$ e $y$ sono relativamente primi se il gcd($r,y$) = 1.
Nella dimostrazione indicheremo l'algoritmo di Euclide come l'algoritmo $E$.
Esso utilizza la funzione mod, per cui $x$ mod $y$ è il resto della divisione intera di $x$ per $y$.

\text{}
\newline
\textbf{DIMOSTRAZIONE.}
L'algoritmo euclideo $E$ è il seguente.

\text{}
\newline
$E =$ "Su input $\langle x, y \rangle$, dove $x$ e $y$ sono numeri naturali in binario:
\begin{enumerate}
    \item Ripete finchè $y = 0$:
    \item \quad Pone $x \leftarrow x mod y$.
    \item \quad Scambia $x$ e $y$.
    \item Output $x$."
\end{enumerate}
L'algoritmo $R$ risolve $RELPRIME$, usando $E$ come sottoprocedura.
\newline
$R =$ "Su input $\langle x, y \rangle$, dove $x$ e $y$ sono numeri naturali in binario:
\begin{enumerate}
    \item Esegue $E$ su $\langle x, y \rangle$.
    \item Se il risultato è 1, \textit{accetta}; altrimenti, \textit{rifiuta}."
\end{enumerate}
Chiaramente, se $E$ lavora correttamente in tempo polinomiale, così fa $R$ e quindi abbiamo solo bisogno di analizzare $E$ per valutarne il tempo e la correttezza.
Per analizzare la complessità di tempo di $E$, per prima cosa mostriamo che ogni esecuzione della fase 2 (tranne eventualmente la prima), taglia il valore di $x$ di almeno la metà.
Dopo l'esecuzione della la fase 2, risulta $x < y$ a causa della natura della funzione mod.
Dopo la fase 3, risulta $x > y$ perché i due valori sono stati scambiati.
Così, quando la fase 2 viene successivamente eseguita, $x > y$.
Se $x/2 \geq y$, allora $x mod y < y \leq x/2$ e $x$ diminuisce di almeno la metà.
Se $x/2 < y$, allora $x mod y = x - y < x/2$ e $x$ diminuisce di almeno la metà.

I valori di $x$ e $y$ sono scambiati ogni volta che viene eseguita la fase 3, così ognuno dei valori originali di $z$ e $y$ è ridotto di almeno la metà ad ogni iterazione del ciclo.
Quindi, il numero massimo di volte in cui le fasi 2 e 3 sono eseguite risulta pari al minore tra $2 log_2 x$ e $2 log_2 y$. 
Questi logaritmi sono proporzionali alle lunghezze delle rappresentazioni, il che dà il numero di fasi eseguite pari a $O(n)$. 
Ogni fase di $E$ utilizza solo tempo polinomiale, quindi il tempo totale di esecuzione è polinomiale.
\vspace{1em}
\text{}
\newline
L'ultimo esempio di algoritmo polinomiale mostra che ogni linguaggio context-free è decidibile in tempo polinomiale.
\paragraph{Teorema 7.16}
\label{teorema-7.16}
\text{}
\newline
Ogni linguaggio context-free è un elemento di P.

\text{}
\newline
\textbf{IDEA.}
Nel \hyperref[teorema-4.9]{\textcolor{blue}{Teorema 4.9}} abbiamo dimostrato che ogni CFL è decidibile.
Per fare ciò abbiamo dato un algoritmo che lo decide. 
Se questo algoritmo venisse eseguito in tempo polinomiale, questo teorema seguirebbe come corollario.
Ricapitoliamo l'algoritmo e vediamo se funziona abbastanza velocemente. 
Sia $L$ un CFL generato da una CFG $G$ in forma normale di Chomsky.
Qualsiasi derivazione di una stringa $w$ ha $2n - 1$ passi, dove $n$ è la lunghezza di $w$ perché $G$ è in forma normale di Chomsky.
Il decisore per $L$ funziona provando tutte le possibili derivazioni con $2n - 1$ passi quando il suo input è una stringa di lunghezza $n$. 
Se uno di questi è una derivazione di $w$, il decisore accetta, altrimenti, rifiuta.

Una rapida analisi di questo algoritmo dimostra che non viene eseguito in tempo polinomiale. 
Il numero di derivazioni con k passi può essere esponenziale in k, per cui questo algoritmo può richiedere tempo esponenziale.

Per ottenere un algoritmo di tempo polinomiale introduciamo una potente tecnica chiamata \textbf{\textit{programmazione dinamica}}.
Memorizziamo la soluzione ad ogni sottoproblema in modo da doverlo risolvere solo una volta. 
Lo facciamo creando una tabella di tutti i sottoproblemi e inserendo le loro soluzioni sistematicamente appena le troviamo.
Nel caso in oggetto, consideriamo i sottoproblemi consistenti nel determinare se ogni variabile in G genera ciascuna sottostringa di $w$.
L'algoritmo inserisce la soluzione a questo sottoproblema in una tabella $n \times n$.
Per $i \leq j$, la voce ($i, j$)-esima della tabella contiene la collezione di variabili che generano la sottostringa $w_1w_{i+1}...W_{j}$.
Per $i > j$, le voci della tabella non sono utilizzate. 
L'algoritmo riempie le voci della tabella per ogni sottostringa di $w$. 
In primo luogo si riempiono le voci corrispondenti alle sottostringhe di lunghezza 1, poi quelle di lunghezza 2 e così via.
Si utilizzano le voci per le lunghezze minori per determinare quelle per le lunghezze superiori.

Per esempio, si supponga che l'algoritmo abbia già stabilito quali varibili generano tutte le sottostringhe di lunghezza fino a $k$.
Per determinare se una variabile $A$ genera una particolare sottostringa di lunghezza $k + 1$, l'algoritmo suddivide la sottostringa in due parti non vuote in tutti i $k$ modi possibili. 
Per ogni suddivisione, l'algoritmo esamina ogni regola $A \rightarrow BC$ per determinare se $B$ genera la prima parte e $C$ genera la seconda parte, utilizzando le voci della tabella precedentemente calcolate.
Se entrambe $B$ e $C$ generano le rispettive parti, $A$ genera la sottostringa e quindi viene aggiunta la voce corrispondente nella tabella. 
L'algoritmo inizia il processo con le stringhe di lunghezza 1 esaminando la tabella per le regole $A \rightarrow b$.

\text{}
\newline
\textbf{DIMOSTRAZIONE.}
Il seguente algoritmo $D$ implementa l'idea della dimostrazione. 
Sia $G$ una CFG in forma normale di Chomsky che genera il CFL $L$.
Si supponga che $S$ sia la variabile iniziale. 
(Ricordiamo che la stringa vuota viene gestita in modo speciale nella grammatica in forma normale di Chomsky. 
L'algoritmo gestisce il caso particolare in cui $w = \varepsilon$ nella fase 1.) 
I commenti appaiono tra doppie parentesi quadre.
\newline
$D =$ "Su input $w = w_1w_2...w_n$:
\begin{enumerate}
    \item Per $w = \varepsilon$, se $S \rightarrow \varepsilon$ è una regola di $G$, \textit{accetta}; altrimenti, \textit{rifiuta}.\newline [caso $w = \varepsilon$]
    \item Per $i = 1$ a $n$: \qquad [esamina ogni sottostringa di lunghezza 1]
    \item \quad Per ogni variabile $A$:
    \item \qquad Testa se $A \rightarrow b$ è una regola, dove $b = w_i$.
    \item \qquad Se sì, inserisci $A$ in $table(i, i)$.
    \item Per $l = 2$ a $n$: \qquad [$l$ è la lunghezza della sottostringa]
    \item \quad Per $i = 1$ a $n - l + 1$: \qquad [$i$ è la posizione iniziale della sottostringa]
    \item \qquad \quad Sia $j = i + l - 1$. \qquad [$j$ è la posizione finale della sottostringa]
    \item \qquad \quad Per $k = i$ a $j - 1$: \qquad [$k$ è la posizione di suddivisione]
    \item \qquad \qquad Per ogni regola $A \rightarrow BC$:
    \item \qquad \qquad \quad Se $B$ è in $table(i, k)$ e $C$ è in $table(k + 1, j)$, inserisci $A$ in $table(i, j)$.
    \item Se $S$ è in $table(1, n)$, \textit{accetta}; altrimenti, \textit{rifiuta}."
\end{enumerate}
Ora analizziamo $D$. 
Ogni fase è facilmente implementata in modo da essere eseguita in tempo polinomiale. 
Le fasi 4 e 5 sono eseguite al più $nv$ volte, dove $v$ è il numero di variabili in $G$ ed è una costante fissa indipendente da $n$;
quindi queste fasi sono eseguite $O(n)$ volte. 
La fase 6 è eseguita al più $n$ volte. 
Ogni volta che la fase 6 viene eseguita, la fase 7 vieneeseguita al più $n$ volte.
Ogni volta che la fase 7 viene eseguita, le fasi 8 e 9 sono eseguite al più $r$ volte. 
Ogni volta che la fase 9 viene eseguita, la fase 10 è eseguita $r$ volte, dove $r$ è il numero di regole di $G$ ed è un'altra costante.
Quindi la fase 11, il ciclo interno dell'algoritmo, viene eseguito $O(n^3)$ volte.
La somma totale mostra che $D$ esegue $O(n^3)$ fasi.

\subsection{La classe NP}
Come abbiamo osservato nella precedente, per molti problemi possiamo evitare la ricerca mediante forza bruta ed ottenere una soluzione polinomiale.
Tuttavia, i tentativi di evitare la forza bruta nel caso di altri problemi, tra cui molti utili ed interessanti, non hanno avuto successo e non sono noti algoritmi polinomiali per risolvere tali problemi.
Come mai i tenta tivi di trovare algoritmi polinomiali per questi problemi non hanno avuto successo?
Non conosciamo la risposta a questa importante domanda.
Forse questi problemi ammettono algoritmi in tempo polinomiale che si basano su principi non ancora noti.
O forse alcuni di questi problemi semplicemente non possono essere risolti in tempo polinomiale.
Possono essere intrinsecamente difficili.
Una scoperta notevole riguardo a questa domanda dimostra che le complessità di molti problemi sono legate tra di loro.
Un algoritmo polinomiale per un certo problema può essere utilizzato per risolvere un'intera classe di problemi.
Per capire questo fenomeno, cominciamo con un esempio.
Un cammino Hamiltoniano in un grafo orientato G è un cammino orientato che attraversa ogni nodo del grafo esattamente una volta. 
Consideriamo il problema di verificare se un grafo orientato contiene un cammino Hamiltoniano che collega due nodi specificati, come mostrato nella seguente figura.
Sia
$$
HAMPATH = \{ \langle G, s, t \rangle \mid G \text{ è un grafo orientato che contiene un cammino Hamiltoniano da } s \text{ a } t \}.
$$
\begin{figure}[H]
    \centering
    \includegraphics[width=0.4\textwidth]{Immagini/93.png}
    \caption{Il problema $HAMPATH$: esiste un cammino Hamiltoniano da $s$ a $t$?}
    \label{fig:hampath}
\end{figure}
Possiamo facilmente ottenere un algoritmo avente tempo esponenziale per il problema $HAMPAT$ modificando l'algoritmo di forza bruta per $PATH$ dato nel \hyperref[teorema-7.14]{\textcolor{blue}{Teorema 7.14}}. 
Abbiamo bisogno solo di aggiungere un controllo per verificare che il cammino potenziale è Hamiltoniano. 
Nessuno sa se $HAMPATH$ è risolvibile in tempo polinomiale.
Il problema $HAMPATH$ ha una caratteristica chiamata verificabilità polinomiale che è importante per capire la sua complessità. 
Anche se non conosciamo una maniera veloce (ad esempio, in tempo polinomiale) per determinare se un grafo contiene un cammino Hamiltoniano, se un tale cammino è stato scoperto in qualche modo (magari utilizzando un algoritmo di tempo esponenziale), si potrebbe facilmente convincere qualcun altro della sua esistenza semplicemente esibendolo. 
In altre parole, verificare l'esistenza di un cammino Hamiltoniano può essere molto più facile che determinare la sua esistenza.

Un altro problema polinomialmente verificabile è l'essere composto. 
Ricordiamo che un numero naturale è \textit{\textbf{composto}} se è il prodotto di due numeri interi maggiori di 1 (cioè, un numero composto è un numero che non è primo).
Sia
$$
COMPOSITES = \{ x \mid x = pq \text{ per gli interi } p,q > 1\}.
$$
Alcuni problemi potrebbero non essere polinomialmente verificabili.
Per esempio, consideriamo $\overline{HAMPATH}$, il complemento del problema $HAMPATH$. 
Anche se riuscissimo a determinare (in qualche modo) che un grafo non ha un cammino Hamiltoniano, non conosciamo alcun modo per verificarne l'inesistenza senza utilizzare lo stesso algoritmo avente tempo esponenziale usato per la determinazione originaria. 
Una definizione formale è la seguente.

\paragraph{Definizione 7.18}
\label{definizione-7.18}
\vspace{1em}
\text{}
\newline
\begin{tcolorbox}[colback=yellow!10!white, colframe=yellow!50!black, title=Definizione 7.18]
    Un \textit{\textbf{verificatore}} per un linguaggio $A$ è un algoritmo $V$, dove
    $$
    A = \{ w \mid V \text{ accetta } \langle w, c \rangle \text{ per qualche stringa } c \}.
    $$
    Misuriamo il tempo di un verificatore solo in termini della lungezza di $w$, quindi un \textit{\textbf{verificatore in tempo polinomiale}} viene eseguito in tempo polinomiale nella lunghezza di $w$.
    Un linguaggio $A$ è \textit{\textbf{polinomialmente verificabile}} se ammette un verificatore in tempo polinomiale.
\end{tcolorbox}
Un verificatore usa ulteriori informazioni, rappresentate dal simbolo $c$ nella \hyperref[definizione-7.18]{\textcolor{blue}{Definizione 7.18}}, per verificare che una stringa $w$ è un elemento di $A$.
Questa indormazione è chiamata \textit{\textbf{certificato}}, o \textit{\textbf{prova}}, di appartenenza ad $A$.
Osservate che per i verificatori polinomiali, il certificato ha lunghezza polinomiale (nella lunghezza di $w$) perchè questo è tutto ciò cui il verificatore può accedere a causa del suo limite sul tempo.
Proviamo ad applicare questa definizione per i linguaggi $HAMPATH$ e $COMPOSITES$.
Per il problema $HAMPATH$, un certificato per una stringa $\langle G, s, t \rangle \in HAMPATH$ è un cammino Hamiltoniano da $s$ a $t$ in $G$.
Per il problema $COMPOSITES$, un certificato per il numero $x$ è uno dei suoi divisori.
In entrambi i casi, il verificatore, avendo il certificato, può verificare in tempo polinomiale che l'input è il linguaggio.
\paragraph{Definizione 7.19}
\label{definizione-7.19}
\vspace{1em}
\text{}
\newline
\begin{tcolorbox}[colback=yellow!10!white, colframe=yellow!50!black, title=Definizione 7.19]
    \textbf{NP} è la classe dei linguaggi che ammettono un verificatore in tempo polinomiale.
\end{tcolorbox}
La classe NP è importante perché contiene molti problemi di interesse pratico. 
Dalla discussione precedente, sia $HAMPATH$ che $COMPOSITES$ sono elementi di NP. 
Come abbiamo accennato, $COMPOSITES$ è anche elemento di P, che è un sottoinsieme di NP; 
ma dimostrare questo risultato più forte è più difficile.
Il termine NP proviene da \textit{\textbf{tempo polinomiale non deterministico}} e deriva da una caratterizzazione alternativa che utilizza macchine di Turing non deterministiche di tempo polinomiale.
I problemi in NP sono a volte chiamati problemi NP.

Quella che segue è una macchina di Turing non deterministica che decide
il problema $HAMPATH$ in tempo polinomiale non deterministico. 
Ricordiamo che nella \hyperref[definizione-7.9]{\textcolor{blue}{Definizione 7.9}}, abbiamo definito il tempo di una macchina non deterministica come il tempo usato dalla computazione corrispondente alla ramificazione più lunga.
\vspace{1em}
\text{}
\newline
$N_1 =$ "Su input $\langle G, s, t \rangle$, dove $G$ è un grafo orientato e $s$ e $t$ sono nodi di $G$:
\begin{enumerate}
    \item Scrive una lista di $m$ numeri, $p_1,...,p_m,$ dove $m$ è il numero di nodi in $G$. Ogni numero nella lista è scelto in modo non deterministico tra 1 e $m$.
    \item Controlla se vi sono tipetizioni nella lista. Se ne trova una, \textit{rifiuta}.
    \item Controlla se $s = p_1$ e $t = p_m$. Se una delle due è falsa, \textit{rifiuta}.
    \item Per ogni $i$ tra 1 e $m - 1$, controlla se ($p_i,p_{i+1}$) è un arco di $G$. Se non lo è \textit{rifiuta.}  Altrimenti tutti i test sono stati superati, quindi \textit{accetta}."
\end{enumerate}
Per analizzare questo algoritmo e verificare che viene eseguito in tempo polinomiale non deterministico, esaminiamo ogni sua fase. 
Nella fase 1, la selezione non deterministica richiede chiaramente tempo polinomiale. 
Nelle fasi 2 e 3, ogni parte è un semplice controllo, quindi entrambe richiedono tempo polinomiale.
Quindi, questo algoritmo viene eseguito in tempo polinomiale non deterministico.
\vspace{1em}
\text{}
\newline
\paragraph{Teorema 7.20(orale)}
\label{teorema-7.20}
\text{}
\newline
\begin{tcolorbox}[colback=white, colframe=black, title=Teorema 7.20]
    Un linguaggio è in NP sse esso viene deciso in tempo polinomiale da una macchina di Turing non deterministica.
\end{tcolorbox}
\vspace{1em}
\text{}
\newline
\textbf{IDEA.}
Mostriamo come convertire un verificatore di tempo polinomiale in una NTM equivalente e viceversa. 
La NTM simula il verificatore per indovinare il certificato. 
Il verificatore simula la NTM utilizzando il ramo di computazione accettante come certificato.
\newline
\textbf{DIMOSTRAZIONE.}
Per la parte diretta di questo teorema, assumiamo
$A \in$ NP e mostriamo che $A$ è deciso da una NTM $N$ avente tempo polinomiale. 
Sia $V$ il verificatore polinomiale per $A$ la cui esistenza è assicurata dalla definizione di NP. 
Si supponga che $V$ è una TM avente tempo di
esecuzione $n^k$ e costruiamo $N$ come segue.
\vspace{1em}
\text{}
\newline
$N =$ "Su input $w$ di lunghezza $n$:
\begin{enumerate}
    \item In maniera non deterministica seleziona una stringa $c$ di lunghezza al più $n^k$
    \item Esegue $V$ su input $\langle w, c \rangle$.
    \item Se $V$ accetta, \textit{accetta}; altrimenti, \textit{rifiuta}."
\end{enumerate}
Per dimostrare la direzione inversa del teorema, supponiamo che $A$ è deciso da una NTM $N$ avente tempo polinomiale e costruiamo come segue un verificatore $V$ avente tempo polinomiale.
\vspace{1em}
\text{}
\newline
$V =$ "Su input $\langle w, c \rangle$, dove $w$ e $c$ sono stringhe:
\begin{enumerate}
    \item Simula $N$ su input $w$, trattando ogni simbolo di $c$ come la descrizione di una scelta non deterministica da formulare ad ogni passo (come nella dimostrazione del \hyperref[teorema-3.16]{\textcolor{blue}{Teorema 3.16}}).
    \item Se questo ramo di computazione di $N$ accetta, \textit{accetta}; altrimenti, \textit{rifiuta}."
\end{enumerate}
Definiamo la classe di complessità di tempo non deterministico $NTIME(t(n))$ in modo analogo alla classe di complessità di tempo deterministico $TIME(t (n))$.
\paragraph{Definizione 7.21}
\label{definizione-7.21}
\vspace{1em}
\text{}
\newline
\begin{tcolorbox}[colback=yellow!10!white, colframe=yellow!50!black, title=Definizione 7.21]
    $NTIME(t(n)) = \{L \mid L \text{ è un linguaggio deciso da una macchina di Turing non deterministica di tempo } O(t(n))\}$
\end{tcolorbox}
\paragraph{Corollario 7.22}
\label{corollario-7.22}
\vspace{1em}
\text{}
\newline
$NP = \bigcup_{k} NTIME(n^k)$.
\vspace{1em}

La classe NP è insensibile alla scelta di un modello computazionale ragionevole non deterministico in quanto tutti questi modelli sono polinomialmente equivalenti. 
Nel descrivere e analizzare algoritmi non deterministici aventi tempo polinomiale, seguiamo le convenzioni precedenti per gli algoritmi di tempo polinomiale deterministici. 
Ogni fase di un algoritmo non deterministico di tempo polinomiale deve avere una chiara implementazione in un tempo polinomiale non deterministico su un modello di calcolo non deterministico ragionevole. 
Analizziamo l'algoritmo per dimostrare che ogni ramo utilizza al massimo un numero polinomiale di fasi.

\subsubsection{Esempi di problemi in NP}
Una \textit{\textbf{clique}} in un grafo non orientato è un sottografo, in cui ogni due nodi sono collegati da un arco. 
Una \textit{\textbf{$k$-clique}} è una clique che contiene $k$ nodi.
La seguente illustra un grafo con una 5-clique.
\begin{figure}[H]
    \centering
    \includegraphics[width=0.4\textwidth]{Immagini/94.png}
    \caption{Un grafo con una 5-clique}
    \label{fig:5-clique}
\end{figure}
Il problema della clique consiste nel determinare se un grafo contiene una clique di una dimensione specificata. 
Sia
$$
CLIQUE = \{ \langle G, k \rangle \mid G \text{ è un grafo che contiene una $k$-clique} \}.
$$
\paragraph{Teorema 7.24}
\label{teorema-7.24}

\text{}
\newline
$CLIQUE \in NP$.

\text{}
\newline
\textbf{IDEA.} La clique è il certificato.

\text{}
\newline
\textbf{DIMOSTRAZIONE.} Il seguente algoritmo è un verificatore $V$ per $CLIQUE$.

\text{}
\newline
$V =$ "Su input $\langle \langle G, k \rangle, c \rangle$:
\begin{enumerate}
    \item Controlla se $c$ è sottografo con $k$ nodi in $G$.
    \item Controlla se $G$ contiene tutti gli archi tra i nodi di $c$.
    \item Se entrambi i controlli sono veri, \textit{accetta}; altrimenti, \textit{rifiuta}."
\end{enumerate}
\vspace{1em}
\text{}
\newline
\textbf{DIMOSTRAZIONE ALTERNATIVA.}
Se preferite pensare a NP in termini di macchine di Turing di tempo polinomiale non deterministico, si può dimostrare questo teorema dandone una che decide $CLIQUE$. 
Osservate la somiglianza tra le due prove.

\text{}
\newline
$N =$ "Su input $\langle G, k \rangle$ dove $G$ è un grafo:
\begin{enumerate}
    \item Seleziona non deterministicamente un sottoinsieme $c$ di $k$ nodi di $G$.
    \item Controlla se $G$ contiene tutti gli archi tra i nodi di $c$.
    \item Se sì, \textit{accetta}; altrimenti, \textit{rifiuta}."
\end{enumerate}

\text{}
\newline
Come altro esempio. consideriamo il problema di aritmetica sugli interi $SUBSET-SUM$. 
Ci viene dato un insieme di numeri $x_1,...,x_k$ ed un valore obiettivo $t$.
Vogliamo determinare se l'insieme contiene un sottoinsieme che somma a $t$. Quindi,
$$
SUBSET-SUM = \{ \langle S, t \rangle \mid S = \{x_1,...,x_k\}, \text{ e per qualche } \{y_1,...,y_l\} \subseteq \{x_1,...,x_k\}, \text{ si ha } \Sigma y_i = t \}.
$$
Per esempio. $\langle \{4, 11, 16, 21, 27\}, 25 \rangle \in SUBSET-SUM \text{ perché } 4 + 21 = 25$. 
Notate che $\{x_1,...,x_k\} \text{ e } \{y_1,...,y_l\}$ sono considerati \textbf{\textit{multinsiemi}} e quindi permettono la ripetizione di elementi.
\paragraph{Teorema 7.25}
\label{teorema-7.25}
\vspace{1em}
\text{}
\newline
$SUBSET-SUM$ è in NP.

\text{}
\newline
\textbf{IDEA.} Il sottoinsieme è il certificato.

\text{}
\newline
\textbf{DIMOSTRAZIONE.} L'algoritmo seguente è un verificatore $V$ per $SUBSET-SUM$.

\text{}
\newline
$V =$ "Su input $\langle \langle S, t \rangle, c \rangle$:
\begin{enumerate}
    \item Verifica se $c$ è una collezione di numeri la cui somme è $t$.
    \item Verifica se $S$ contiene tutti i numeri in $c$.
    \item Se entrambi i controlli sono veri, \textit{accetta}; altrimenti, \textit{rifiuta}."
\end{enumerate}
\vspace{1em}
\text{}
\newline
\textbf{DIMOSTRAZIONE ALTERNATIVA.} Possiamo anche dimostrare questo teorema esibendo una macchina di Turing di tempo polinomiale non deterministico per $SUBSET-SUM$ come segue.

\text{}
\newline
$N =$ "Su input $\langle S, t \rangle$:
\begin{enumerate}
    \item Seleziona non deterministicamente un sottoinsieme $c$ dei numeri in $S$.
    \item Verifica se $c$ è una collezione di numeri la cui somma è $t$.
    \item Se sì, \textit{accetta}; altrimenti, \textit{rifiuta}."
\end{enumerate}

\text{}
\newline
Si osservi che i complementi di questi insiemi, $\overline{CLIQUE}$ e $\overline{SUBSET-SUM}$, non sono elementi di NP. 
Verificare che qualcosa \textit{non} è presente sembra essere più difficile di verificare che è presente. 
Definiamo una classe di complessità separata, denominata \textbf{coNP}, che contiene i linguaggi che sono il complemento di un linguaggio in NP. 
Non sappiamo se coNP è diversa da NP.

\subsubsection*{La questione P = NP}
Come abbiamo detto, NP è la classe dei linguaggi che sono risolubili in tempo polinomiale con una macchina di Turing non deterministica; 
o, equivalentemente, è la classe dei linguaggi per cui l'appartenenza può essere verificata in tempo polinomiale. 
P è la classe dei linguaggi per cui l'appartenenza può essere decisa in tempo polinomiale. 
Riassumiamo queste informazioni come segue, dove informalmente ci riferiamo ad un problema risolvibile in tempo polinomiale come risolvibile "velocemente."
\begin{center}
    P = la classe dei linguaggi per cui l'apparteneza è \textit{decidibile} velocemente.

    NP = la classe dei linguaggi per cui l'appartenenza è \textit{verificabile} velocemente.
\end{center}
Abbiamo presentato esempi di linguaggi, quali $HAMPATH$ e $CLIQUE$, che
sono elementi di NP, ma di cui non è nota l'appartenenza a P. 
Il potere della verificabilità polinomiale sembra essere molto maggiore di quello della decidibilità polinomiale. 
Ma, anche se difficile da immaginare, P e NP potrebbero essere uguali. 
Non siamo in grado di dimostrare l'esistenza di un solo linguaggio in NP che non è in P.

La domanda se P = NP è uno dei maggiori problemi irrisolti dell'informatica teorica e della matematica contemporanea. 
Se queste classi fossero uguali, qualsiasi problema polinomialmente verificabile sarebbe polinomialmente decidibile. 
La maggior parte dei ricercatori ritengono che le due classi non sono uguali, perché molte persone hanno investito, senza successo, enormi sforzi per trovare algoritmi aventi tempo polinomiale per problemi in NP. 
I ricercatori hanno anche tentato di dimostrare che le due classi sono diverse, ma ciò comporterebbe dimostrare che non esiste un algoritmo veloce che può sostituire la ricerca esaustiva.
Cio è fuori dalla portata scientifica attuale. 
La figura seguente illustra le due possibilità.
\begin{figure}[H]
    \centering
    \includegraphics[width=0.4\textwidth]{Immagini/95.png}
    \caption{La questione P = NP}
    \label{fig:p-np}
\end{figure}
Il miglior metodo deterministico attualmente noto per decidere se un linguaggio è in NP utilizza tempo esponenziale. 
In altre parole, possiamo dimostrare che
$$
NP \subseteq EXPTIME = \bigcup_{k} TIME(2^{n^k}).
$$
ma non sappiamo se NP è contenuta in una classe di complessità deterministica più piccola.

\subsection{NP-COMPLETEZZA}
Un progresso importante sulla questione P diverso da NP ci fu all'inizio degli anni '70 con il lavoro di Stephen Cook e Leonid Levin. 
Essi scoprirono vari problemi appartenenti a NP la cui complessità individuale è correlata a quella dell'intera classe. 
Se esistesse un algoritmo di tempo polinomiale per uno qualsiasi di essi, tutti i problemi in NP diventerebbero risolvibili in tempo polinomiale. 
Questi problemi vengono detti \textit{\textbf{NP-completi}}.

Se un qualsiasi problema appartenente a NP richiede tempo più che polinomiale, lo stesso vale per uno NP-completo.
Quindi, provare che il problema è NP-completo costituisce un'evidenza forte della sua non polinomialità. 

Il primo problema NP-completo che presentiamo è il \textbf{\textit{problema della soddisfacibilità}}.

Una formula booleana è un'espressione che coinvolge variabili e operazioni booleane. 
Per esempio,
$$ 
\phi = (\overline{x} \land y) \lor (x \land \overline{y})
$$
è una formula booleana. Una formula booleana è soddisfacibile se qualche assegnamento di 0 e di 1 alle variabili fa sì che la formula valga 1.
La formula precedente è soddisfacibile perchè l'assegnamento $x = 0$, $y = 1$ e $z = 0$ fa si che $\phi$ valga 1.
Diciamo che l'assegnamento \textit{soddisfa} $\phi$.
Il \textit{\textbf{problema della soddisfacibilità}} consiste nel determinare se una formula booleana è soddisfacibile.
Sia 
$$
SAT = \{ \langle \phi \rangle \mid \phi \text{ è una formula booleana soddisfacibile} \}.
$$
Il teorema che ora enunciamo lega la complessità del problema $SAT$ alle complessità di tutti i problemi appartenenti a NP.
\paragraph{Teorema 7.27}
\label{teorema-7.27}
\vspace{1em}
\text{}
\newline
$SAT \in P$ se e solo se $P = NP$.

\subsubsection{Riducibilità in tempo polinomiale}
Quando il problema $A$ si riduce al problema $B$, una soluzione per $B$ può essere usata per risolvere $A$. 
Ora definiamo una versione della riducibilità che tiene conto dell'efficienza della computazione. 
Quando il problema $A$ è riducibile \textit{efficientemente} al problema $B$, una soluzione efficiente per $B$ può essere usata per risolvere $A$ efficientemente.
\paragraph{Definizione 7.28}
\label{definizione-7.28}
\vspace{1em}
\text{}
\newline
\begin{tcolorbox}[colback=yellow!10!white, colframe=yellow!50!black, title=Definizione 7.28]
    Una funzione $f: \Sigma^* \rightarrow \Sigma^*$ è una \textit{\textbf{funzione calcolabile in tempo polinomiale}} se esiste una macchina di Turing di tempo polinomilae $M$ che si arresta con $f(w)$ soltanto sul suo nastro, quando ha iniziato con un qualsiasi input $w$.
\end{tcolorbox}

\paragraph{Definizione 7.29}
\label{definizione-7.29}
\vspace{1em}
\text{}
\newline
\begin{tcolorbox}[colback=yellow!10!white, colframe=yellow!50!black, title=Definizione 7.29]
    Il linguaggio $A$ è \textit{\textbf{riducibile mediante funzione in tempo polinomiale}}, o semplicemente \textit{\textbf{riducibile in tempo polinomiale}}, al linguaggio $B$, scritto $A \leq_p B$, se esiste una funzione calcolabile in tempo polinomiale $f: \Sigma^* \rightarrow \Sigma^*$ tale che per ogni $w$,
    $$
    w \in A \iff f(w) \in B.
    $$
    La funzione $f$ è chiamata \textit{\textbf{riduzione in tempo polinomiale}} di $A$ a $B$.
\end{tcolorbox}

Come con un'ordinaria riduzione mediante funzione, una riduzione in tempo polinomiale di $A$ a $B$ fornisce un modo per convertire la verifica dell'appartenenza ad $A$ nella verifica dell'appartenenza a $B$ - ma ora la conversione viene realizzata efficientemente.
Per verificare se $w \in A$, utilizziamo la
riduzione $f$ per associare $w$ ad $f(w)$ e verificare se $f(u) \in B$.

Se un linguaggio è riducibile in tempo polinomiale ad un linguaggio per cui già si sa che possiede una soluzione di tempo polinomiale, si ottiene una soluzione di tempo polinomiale per il linguaggio originale, come prova il teorema seguente.
\paragraph{Teorema 7.31 (orale)}
\label{teorema-7.31}
\vspace{1em}
\text{}
\newline
Se $A \leq_p B$ e $B \in P$, allora $A \in P$.

\text{}
\newline
\textbf{DIMOSTRAZIONE.} 
Sia $M$ l'algoritmo di tempo polinomiale che decide $B$ ed $f$ la riduzione di tempo polinomiale di $A$ a $B$. 
Descriviamo un algoritmo di tempo polinomiale $N$ che decide $A$ come segue.

\text{}
\newline
$N =$ "Su input $w$:
\begin{enumerate}
    \item Calcola $f(w)$.
    \item Esegue $M$ con input $f(w)$ e dai in output qualsiasi cosa $M$ da in output."
\end{enumerate}
Risulta $w \in A$ ogni volta che $f(w) \in B$ perché $f$ è una riduzione di $A$ a
$B$. Pertanto, $M$ accetta $f(w)$ ogni volta che $w \in A$. 
Inoltre, $N$ computa in tempo polinomiale perché ciascuna delle sue due fasi viene eseguita in tempo polinomiale. 
Si noti che la fase 2 viene eseguita in tempo polinomiale perché la composizione di due polinomi è un polinomio.

\text{}
\newline
Prima di mostrare una riduzione di tempo polinomiale, introduciamo $3SAT$, unn caso speciale di problema di soddisfacibilità per cui tutte le formule sono in una forma speciale.
Un \textit{\textbf{letterale}} è una variabile booleana o una variabile booleana negata, come $x$ o $\overline{x}$. 
Una \textit{\textbf{clausola}} consiste in diversi letterali connessi tramite operatori $\lor$, come ($x_1 \lor \overline{x_2} \lor \overline{x_3} \lor x_4$).
Una formula booleana è in \textit{\textbf{forma normale congiuntiva}}, ed è detta una \textit{\textbf{formula cnf}}, se comprende diverse clausole connesse tramite operatori $\land$, come
$$
(x_1 \lor \overline{x_2} \lor \overline{x_3} \lor x_4) \land (x_3 \lor \overline{x_5} \lor x_6) \land (x_3 \lor \overline{x_6}).
$$
Essa è una formula \textit{\textbf{3cnf}} se tutte le clausole hanno tre letterali come in 
$$
(x_1 \lor \overline{x_2} \lor \overline{x_3}) \land (x_3 \lor \overline{x_5} \lor x_6) \land (x_3 \lor \overline{x_6} \lor x_4) \land (x_4 \lor x_5 \lor x_6).
$$
Sia $3SAT = \{ \langle \phi \rangle \mid \phi \text{ è una formula 3cnf soddisfacibile} \}$.
Se un assegnamento soddisfa una formula cnf, ciascuna clausola deve contenere almeno un letterale che vale 1.

Il teorema seguente presenta una riduzione di tempo polinomiale dal problema $3SAT$ al problema $CLIQUE$.

\paragraph{Teorema 7.32 (orale)}
\label{teorema-7.32}
\vspace{1em}
\text{}
\newline
$3SAT$ è riducibile in tempo polinomiale a $CLIQUE$.

\text{}
\newline
\textbf{IDEA.}
La riduzione di tempo polinomiale $f$ che mostriamo di $3SAT$ a $CLIQUE$ converte formule in grafi. 
Nei grafi costruiti, clique di una dimensione specificata corrispondono ad assegnamenti soddisfacenti per la formula. 
Le strutture all'interno del grafo sono progettate per simulare il comportamento delle variabili e delle clausole.

\text{}
\newline
\textbf{DIMOSTRAZIONE.}
Sia $\phi$ una formula con $k$ clausole complemento
$$
\phi = (a_1 \lor b_1 \lor c_1) \land (a_2 \lor b_2 \lor c_2) \land ... \land (a_k \lor b_k \lor c_k).
$$
La riduzione $f$ genera la stringa $\langle G, k \rangle$, dove $G$ è un grafo non orientato definito come segue.

I nodi in $G$ sono organizzati in $k$ gruppi di tre nodi ciascuno, detti triple, $t_1,..., t_k$. 
Ciascuna tripla corrisponde ad una delle clausole in $\phi$, e ciascun nodo in una tripla corrisponde ad un letterale nella clausola associata. 
Si etichetti ciascun nodo di $G$ con il suo letterale corrispondente in $\phi$.

Gli archi di $G$ connettono tutti meno due tipi di coppie di nodi in $G$. 
Non
ci sono archi presenti tra nodi nella stessa tripla, e non ci sono archi presenti
tra nodi con etichette complementari, come $x_2$ ed $\overline{x_2}$. 
La seguente illustra questa costruzione quando $\phi = (x_1 \lor x_1 \lor x_2) \land (\overline{x_1} \lor \overline{x_2} \lor \overline{x_2}) \land (\overline{x_1} \lor x_2 \lor x_2)$.
\begin{figure}[H]
    \centering
    \includegraphics[width=0.3\textwidth]{Immagini/96.png}
    \caption{Un grafo costruito da una formula 3cnf}
    \label{fig:3cnf}
\end{figure}
Facciamo vedere ora perché questa costruzione funziona. 
Dimostriamo che $\phi$ è soddisfacibile se e solo se $G$ ha una clique di dimensione $k$.

Si supponga che $\phi$ abbia un assegnamento che la soddisfa.
In questo assegnamento soddisfacente, almeno un letterale è vero in ogni clausola.
In ciascuna tripla di $G$, si selezioni un nodo corrispondente al letterale vero nell'assegnamento soddisfacente. 
Se più di un letterale è vero in una particolare clausola, si scelga uno dei letterali veri arbitrariamente. 
I nodi appena selezionati formano una clique di dimensione $k$. 
Il numero di nodi selezionati è $k$ perché ne è stato scelto uno per ognuna delle $k$ triple.
Ciascuna coppia dei nodi selezionati è collegata da un arco perché nessuna coppia rientra in una delle eccezioni descritte in precedenza. 
Non possono essere della stessa tripla perché ne è stato scelto soltanto uno per tripla.
Non possono avere etichette complementari perché i letterali associati erano entrambi veri nell'assegnamento soddisfacente. 
Pertanto, $G$ contiene una clique di dimensione $k$.

Si supponga che $G$ abbia una clique di dimensione $k$. 
Nessuna coppia di nodi di una clique occorre nella stessa tripla perché i nodi nella stessa tripla non sono connessi da archi. 
Pertanto, ciascuna delle $k$ triple contiene esattamente uno dei $k$ nodi della clique. 
Si assegnino valori di verità alle
variabili di $\phi$, in modo tale che ciascun letterale che etichetta un nodo della clique risulti vero.
Questa operazione è sempre possibile perché due nodi etichettati in modo complementare non sono connessi da un arco e, quindi, non possono essere entrambi nella clique.
L'assegnamento così definito per
le variabili soddisfa $\phi$ perché ciascuna tripla contiene un nodo della clique e, quindi, ciascuna clausola contiene un letterale a cui è stato assegnato VERO. 
Pertanto, $\phi$ è soddisfacibile.

I Teoremi \hyperref[teorema-7.31]{\textcolor{blue}{7.31}} e \hyperref[teorema-7.32]{\textcolor{blue}{7.32}} insieme mostrano che se $CLIQUE$ è risolvibile in tempo polinomiale lo è anche $3SAT$.

A questo punto volgiamo l'attenzione ad una definizione che ci permette in modo simile di collegare le complessità di un'intera classe di problemi.

\subsubsection{Definizione di NP-completezza}
\paragraph{Definizione 7.34}
\label{definizione-7.34}
\vspace{1em}
\text{}
\newline
\begin{tcolorbox}
    Un linguaggio $B$ è \textit{\textbf{NP-completo}} se soddisfa due condizioni:
    \begin{enumerate}
        \item $B \in NP$.
        \item Ogni linguaggio $A \in NP$ è riducibile in tempo polinomiale a $B$.
    \end{enumerate}
\end{tcolorbox}

\paragraph{Teorema 7.35}
\label{teorema-7.35}
\vspace{1em}
\text{}
\newline
Se un linguaggio $B$ è NP-completo e $B \in P$, allora $P = NP$.

\text{}
\newline
\textbf{DIMOSTRAZIONE.}
Il teorema discende direttamente dal Teorema \hyperref[teorema-7.31]{\textcolor{blue}{Teorema 7.31}}.

\paragraph{Teorema 7.36}
\label{teorema-7.36}
\vspace{1em}
\text{}
\newline
Se $B$ è NP-completo e $B \leq_p C$ per qualche $C \in NP$, allora $C$ è NP-completo.

\text{}
\newline
\textbf{DIMOSTRAZIONE.}
Già sappiamo che $C$ appartiene a NP, quindi dobbiamo far vedere che ogni $A$ appartenente a NP è riducibile in tempo polinomiale a $C$. 
Poiché $B$ è NP-completo, ogni linguaggio appartenente a NP è riducibile in tempo polinomiale a $B$, e $B$ a sua volta è riducibile in tempo polinomiale a $C$. 
Le riduzioni di tempo polinomiale possono essere composte; cioè, se $A$ è riducibile in tempo polinomiale a $B$ e $B$ è riducibile in tempo polinomiale a $C$, allora $A$ è riducibile in tempo polinomiale a $C$. 
Pertanto, ogni linguaggio appartenente a NP è riducibile in tempo polinomiale a $C$.

\subsubsection{Il teorema di Cook e Levin (e mo so cazzi)}
Una volta che disponiamo di un problema NP-completo, possiamo ottenerne altri attraverso riduzioni polinomiali da esso. 
Tuttavia, determinare il primo problema NP-completo è più difficile.
Lo facciamo ora, provando che $SAT$ è NP-completo.

\subsubsection*{Teorema 7.37 (30L)}
\label{teorema-7.37}
$SAT$ è NP-completo.

\text{}
\newline
Questo teorema implica il \hyperref[teorema-7.27]{\textcolor{blue}{Teorema 7.27}}.

\text{}
\newline
\textbf{IDEA.}
Mostrare che $SAT$ appartiene a NP è facile, e lo faremo velocemente. 
La parte difficile della dimostrazione è far vedere che qualsiasi linguaggio appartenente a NP è riducibile in tempo polinomiale a $SAT$.

Per far ciò, costruiremo una riduzione di tempo polinomiale per ciascun linguaggio $A$ appartenente a NP a $SAT$. 
La riduzione per $A$ prende una stringa $w$ e produce una formula booleana $\phi$ che simula la macchina NP per $A$ su input $w$.
Se la macchina accetta, $\phi$ ha un assegnamento che la soddisfa che corrisponde ad una computazione accettante. 
Se la macchina non accetta, nessun assegnamento soddisfa $\phi$.
Pertanto, $w$ appartiene ad $A$ se e solo se $\phi$ è soddisfacibile.

In realtà, costruire la riduzione per farla funzionare in questo modo è un compito concettualmente semplice, sebbene debbano essere gestiti diversi dettagli. 
Una formula booleana può contenere le operazioni booleane AND, OR, e NOT, e queste operazioni costituiscono la base per la circuiteria usata nei calcolatori elettronici. 
Quindi, il fatto che possiamo progettare una formula booleana per simulare una macchina di Turing non è sorprendente.
\newline
I dettagli sono nell'implementazione di questa idea.

\text{}
\newline
\textbf{DIMOSTRAZIONE.}
Prima di tutto, mostriamo che $SAT$ appartiene a NP. 
Una macchina non deterministica di tempo polinomiale può ipotizzare un assegnamento per una data formula $\phi$ ed accettare se l'assegnamento soddisfa $\phi$.

Successivamente, prendiamo un qualsiasi linguaggio $A$ appartenente a NP e facciamo vedere che $A$ è riducibile in tempo polinomiale a $SAT$.
Sia $N$ una macchina di Turing non deterministica che decide $A$ in tempo $n^k$ per qualche costante $k$. 
(Per comodità in effetti assumiamo che $N$ computi in tempo $n^k - 3$; ma soltato i lettori interessati ai dettagli dovrebbero preoccuparsi di questo punto minore.)
La nozione che segue aiuta a descrivere la riduzione.
Un \textbf{\textit{tableau}} per $N$ su $w$ è una tabella $n^k x n^k$ le cui righe sono le configurazioni di una diramazione della computazione di $N$ su input $w$, come mostrato nella figura seguente.
\begin{figure}[H]
    \centering
    \includegraphics[width=0.4\textwidth]{Immagini/97.png}
    \caption{Un tableau per una macchina di Turing non deterministica}
    \label{fig:tableau}
\end{figure}
Per convenienza nel seguito assumiamo che ciascuna configurazione inizi e finisca con un simbolo \#.
Pertanto, la prima e l'ultima colonna di un tableau sono tutti \#.
La prima riga di un tableau è la configurazione iniziale di $N$
su $w$, e ciascuna riga segue dalla precedente in accordo alla funzione di transizione di $N$.
Un tableau è accettante se qualche riga del tableau è una configurazione accettante.

Ogni tableau accettante per $N$ su $w$ corrisponde ad una diramazione della computazione accettante di $N$ su $w$. 
Quindi, il problema di stabilire se $N$ accetta $w$ è equivalente al problema di stabilire se esiste un tableau accettante per $N$ su $w$.

Passiamo ora alla descrizione della riduzione di tempo polinomiale $f$ di
$A$ a $SAT$. Su input $w$, la riduzione produce una formula $\phi$.
Iniziamo descrivendo le variabili di $\phi$.
Siano $Q$ e $\Gamma$ l'insieme degli stati e l'alfabeto del nastro di $N$, rispettivamente.
Sia $C = Q \cup \Gamma \cup \{ \# \}$.
Per ciascun $i$ e $j$ tra 1 e $n^k$ e per ciascun $s$ in $C$, introduciamo una variabile $X_{i,j,s}$.

Ciascuna delle $(n^k)^2$ entrate di un tableau è chiamata \textbf{\textit{cella}}.
La cella in riga $i$ e colonna $j$ viene denotata con $cell[i, j]$ e contiene un simbolo di $C$.
Rappresentiamo i contenuti delle celle con le variabili di $\phi$. 
Se $x_{i,j,s}$ assume il valore 1, significa che $cell[i, j]$ contiene $s$.
\newline
Progettiamo a questo punto $\phi$ in modo tale che un assegnamento che soffisfa le variabili di $\phi$ corrisponda ad un tableau accettante per $N$ su $w$.
La formula $\phi$ è l'AND di quanttro parti: $\phi = \phi_{cell} \land \phi_{start} \land \phi_{move} \land \phi_{accept}$.
Descriviamo ciascuna parte, una per volta.

Come anticipato precedentemente, porre a 1 la variabile $x_{i,j,s}$ corrisponde a posizionare il simbolo $s$ in $cell[i.j]$.
La prima cosa che dobbiamo garantire al fine di ottenere una corrispondenza tra un assegnamento ed un tableau è che l'assegnamento ponga ad 1 esattamente una variabile per ciascuna cella.
La formula $\phi_{cell}$ garantisce questo requisito esprimendolo in termini di operazioni booleane:
$$
\phi_{cell} = \bigwedge_{i \leq i,j \leq n^k}  \left[ \left( \bigvee_{s \in C} x_{i,j,s} \right) \land \left(\bigwedge_{\substack{s,t \in C \\s \neq t}} (\overline{x_{i,j,s}} \lor \overline{x_{i,j,t}}) \right) \right].
$$
I simboli $\wedge$ e $\vee$ sono usati per le operazioni AND e OR.
Per esempio, l'espressione nella formula precedente
$$
\bigvee_{s \in C} x_{i,j,s}
$$
è un'abbreviazione per 
$$
x_{i,j,s_1} \lor x_{i,j,s_2} \lor ... \lor x_{i,j,s_{l}}.
$$
dove $C = \{s_1, s_2, ..., s_l\}$.
Quindi, $\phi_{cell}$ è in realtà un'espressione grande che contiene un frammento per ciascuna cella nel tableau, poiché $i$ e $j$ variano da 1 a $n^k$.
La prima parte del frammento stabilisce che almeno una variabile assume valore 1 nella cella corrispondente. 
La seconda parte di ciascun frammento stabilisce che non più di una variabile assume valore  1 (letteralmente, stabilisce che, in ogni coppia di variabili, almeno una assume valore 0) nella cella corrispondente. 
Questi frammenti sono collegati attraverso operazioni $\wedge$.

La prima parte di $\phi_{cell}$ all'interno delle parentesi garantisce che almeno una variabile che è associata con ciascuna cella vale 1, laddove la seconda parte garantisce che non più di una variabile vale 1 per ciascuna cella.
Qualsiasi assegnamento alle variabili che soddisfa $\phi$ (e di conseguenza $\phi{cell}$) deve avere esattamente una variabile ad 1 per ogni cella.
Pertanto, qualsiasi assegnamento che soddisfa $\phi$ specifica un simbolo in ciascuna cella della tabella. 
Le parti $\phi_{start}, \phi_{move}, \text{ e } \phi{accept}$ garantiscono che questi simboli corrispondano realmente ad un tableau accettante come segue.

La formula $\phi_{start}$, assicura che la prima riga della tabella sia la configurazione iniziale di $N$ su $w$, richiedendo esplicitamente che le variabili corrispondenti siano ad 1:
$$
\phi_{start} = x_{1,1,\#} \land x_{1,2,q_0} \land 
x_{1,3,w_1} \land x_{1,4,w_2} \land ... \land x_{1,n+2,w_n} \land 
x_{1,n+3,\sqcup} \land ... \land x_{1,{n^k-1,\sqcup}} \land x_{1,n^k, \#}.
$$
La formula $\phi_{accept}$ garantisce che una configurazione accettante sia presente nel tableau.
Essa assicura che $q_{accept}$, il simbolo per lo stato di accettazione, compaia in una delle celle del tableau, richiedendo che una delle variabili corrispondenti sia a 1:
$$
\phi_{accept} = \bigvee_{1 \leq i,j \leq n^k} x_{i,j,q_{accept}}.
$$
Infine, la formula $\phi_{move}$ garantisce che ciascuna riga del tableau corrisponda ad una configurazione che segue legittimamente dalla configurazione della riga precedente, in accordo alle regole di $N$. 
Lo fa assicurando che ogni finestra di celle $2 \times 3$ sia lecita. 
Diciamo che una finestra $2 \times 3$ è lecita se non viola le azioni specificate dalla funzione di transizione di $N$. 
In altre parole, una finestra è lecita se può esser presente quando una configurazione segue correttamente da un'altra.

Per esempio, siano $a, b,$ e $c$ elementi dell'alfabeto di nastro, e $q_1$ e $q_2$ stati di $N$. 
Si assuma che quando nello stato $q_1$ con la testina che legge una $a$, $N$ scrive una $b$, resta nello stato $q_1$, e muove a destra; e che quando nello stato $q_1$ con la testina che legge una $b$, $N$ non deterministicamente
\begin{enumerate}
    \item scrive una $c$, entra in $q_2$ e muove a sinistra, oppure
    \item scrive una $a$, entra in $q_2$ e muove a destra.
\end{enumerate}
In termini formali, $\delta(q_1,a) = \{(q_1,b,R)\}$ e $\delta(q_1,b) = \{(q_2,c,L), (q_2,a,R)\}$.
Esempi di finestre lecite sono mostrati nella figura seguente.
\begin{figure}[H]
    \centering
    \includegraphics[width=0.4\textwidth]{Immagini/98.png}
    \caption{Finestre lecite}
    \label{fig:finestre}
\end{figure}
Nella figura le finestre (a) e (b) sono lecite perché la funzione di transizione permette ad $N$ di muovere nella direzione indicata.
La finestra (c) è lecita perché, con $q_1$ presente sul lato destro della riga di sopra, non sappiamo su quale simbolo la testina sia.
Detto simbolo potrebbe essere una $a$, e $q_1$ potrebbe cambiarlo in una $b$ e muovere a destra. 
Una tale possibilità genererebbe questa finestra, quindi essa non viola le regole di $N$.
La finestra (d) è ovviamente lecita perché la riga di sopra e quella di sotto sono identiche, situazione che si verifica se la testina non è adiacente alla posizione della finestra. 
Si noti che \# in una finestra lecita può comparire a sinistra o a destra sia nella riga superiore che in quella inferiore.
La finestra (e) è lecita perché lo stato $q_1$ potrebbe essere immediatamente a destra della riga superiore e la macchina, leggendo una $b$, si sarebbe mossa a sinistra nello stato $q_2$ che ora compare all'estremità destra della riga inferiore. 
Infine, la finestra (f) è lecita perché lo stato $q_1$ potrebbe essere stato immediatamente a sinistra della riga superiore, e la macchina potrebbe aver cambiato la $b$ in una $c$ ed effettuato una mossa a sinistra.

Le finestre mostrate nella figura seguente non sono lecite per la macchina $N$.
\begin{figure}[H]
    \centering
    \includegraphics[width=0.4\textwidth]{Immagini/99.png}
    \caption{Finestre non lecite}
    \label{fig:finestre-non-lecite}
\end{figure}
Nella finestra (a), il simbolo centrale nella riga superiore non può cambiare perché non c'è uno stato ad esso adiacente. 
La finestra (b) non è lecita perché la funzione di transizione specifica che la $b$ viene cambiata in una $c$ ma non in una $a$.
La finestra (c) non è lecita perché due stati compaiono nella riga inferiore.

\text{}
\newline
\paragraph*{Fatto 7.41}
\label{fatto-7.41}
\vspace{1em}
\text{}
\newline
Se la riga superiore del tableau è la configurazione iniziale ed ogni finestra nel tableau è lecita, ciascuna riga del tableau è una configurazione che segue legittimamente dalla precedente.

\text{}
\newline
Proviamo l'asserto considerando ogni coppia di configurazioni adiacenti nel tableau, indicate come la configurazione superiore e la configurazione inferiore. 
Nella configurazione superiore, ogni cella che contiene un simbolo di nastro e non è adiacente ad un simbolo di stato rappresenta la cella centrale superiore in una finestra in cui la riga superiore non contiene stati. 
Pertanto, questo stesso simbolo deve comparire al centro in basso nella finestra.
Quindi, esso è presente nella stessa posizione nella configurazione inferiore.

La finestra contenente il simbolo di stato nella cella centrale superiore garantisce che le tre posizioni corrispondenti vengano aggiornate consistentemente tramite la funzione di transizione.
Pertanto, se la configurazione superiore è una configurazione lecita, altrettanto lo è la configurazione inferiore, e quella inferiore discende da quella superiore in accordo alle regole di $N$.
Si noti che questa dimostrazione, seppur immediata, dipende in maniera cruciale dalla nostra scelta di utilizzare una finestra di dimensione $2 \times 3$.

Torniamo ora alla costruzione di $\phi_{move}$.
Essa garantisce che tutte le finestre nel tableau sono lecite. 
Ciascuna finestra contiene sei celle, che possono essere impostate in un numero fissato di modi per produrre una finestra lecita.
La formula move stabilisce che le impostazioni delle sei celle devono essere uno di questi modi, ossia
$$
\phi_{move} = \bigwedge_{1 \leq i \leq n^k, 1<j<n^k} \left( \text{la finestra } (i,j) \text{ è lecita} \right).
$$
La finestra $(i,j)$ ha $cell[i,j]$ in posizione centrale in alto.
Sostituiamo il testo "la finestra $(i,j)$ è lecita" in questa formula con la formula seguente.
Denotiamo il contenuto delle sei celle di una finestra con $a_1,...,a_6$.
$$
\bigvee_{\substack{a1,...,a_6 \\ \text{è una finestra lecita}}} \left( x_{i,j-1,a_1} \land x_{i,j,a_2} \land x_{i,j+1,a_3} \land x_{i+1,j-1,a_4} \land x_{i+1,j,a_5} \land x_{i+1,j+1,a_6} \right).
$$
Nel seguito analizziamo la complessità della riduzione per mostrare che essa opera in tempo polinomiale. 
Per far ciò, esaminiamo la dimensione di $\phi$.
Prima di tutto, stimiamo il numero di variabili che ha. 
Si ricordi che il tableau è una tabella $n^k \times n^k$, quindi contiene $n^{2k}$ celle.
Ciascuna cella ha $l$ variabili associate ad essa, dove $l$ è il numero di simboli in $C$.
Poiché $l$ dipende soltanto dalla TM $N$ e non dalla lunghezza dell'input $n$, il numero totale di variabili è $O(n^{2k})$.

Stimiamo la dimensione di ciascuna delle parti di $\phi$.
La formula $\phi_{cell}$ contiene un frammento di dimensione fissata della formula per ciascuna cella del tableau, quindi la sua dimensione è $O(n^{2k})$. 
La formula $\phi{start}$ ha un frammento per ciascuna cella nella riga superiore, quindi la sua dimensione è $O(n^{k})$. 
Le formule $\phi{move}$ e $\phi{accept}$ contengono ciascuna un frammento di dimensione fissata della formula per ciascuna cella del tableau, quindi la loro dimensione è $O(n^{2k})$.
Pertanto, la dimensione complessiva di è $O(n^{2k})$.
Tale limite è sufficiente per i nostri scopi perché dimostra che la dimensione di $\phi$ è polinomiale in $n$.
Se fosse stata più che polinomiale, la riduzione non avrebbe avuto alcuna possibilità di generarla in tempo polinomiale.
(In realtà, le nostre stime sono più basse di un fattore $O(logn)$, poiché ciascuna variabile ha indici che possono arrivare fino a $n^k$ e, quindi, possono richiedere $O(logn)$ simboli da scrivere nella formula, ma questo fattore addizionale non cambia la polinomialità del risultato.)
Per rendersi conto che la formula può essere generata in tempo polinomiale, si noti la sua natura altamente ripetitiva.
Ciascun componente della formula è composto da molti frammenti quasi identici, che differiscono soltanto negli indici in modo molto semplice. 
Pertanto, possiamo costruire facilmente una riduzione che produce o in tempo polinomiale dall'input $w$.

\text{}
\newline
Quindi, abbiamo concluso la dimostrazione del teorema di Cook e Levin, mostrando che $SAT$ è NP-completo.
Mostrare la NP-completezza di altri linguaggi generalmente non richiede una dimostrazione così lunga. 
Al contrario, la NP-completezza può essere provata con una riduzione di tempo polinomiale da un linguaggio che è già noto essere NP-completo. 
Possiamo usare $SAT$ per questo scopo; ma usare $3SAT$, il caso speciale di $SAT$ è solitamente più facile. 
Si ricordi che le formule appartenenti a $3SAT$ sono in forma normale congiuntiva (cnf) con tre letterali per clausola. 
Prima di tutto, dobbiamo dimostrare che $3SAT$ stesso è NP-completo. 
Proviamo questo asserto come corollario del \hyperref[teorema-7.37]{\textcolor{blue}{Teorema 7.37}}.

\text{}
\newline
\paragraph{Corollario 7.42}
\label{corollario-7.42}
\vspace{1em}
\text{}
\newline
$3SAT$ è NP-completo.

\text{}
\newline
\textbf{DIMOSTRAZIONE.}
Ovviamente $3SAT$ appartiene a NP, quindi dobbiamo provare solamente che tutti i linguaggi appartenenti a NP si riducono a $3SAT$ in tempo polinomiale.
Un modo per farlo è facendo vedere che $SAT$ si riduce in tempo polinomiale a $3SAT$.
Invece, preferiamo farlo modificando la dimostrazione del \hyperref[teorema-7.37]{\textcolor{blue}{Teorema 7.37}}, in modo tale che produca direttamente una formula in forma normale congiuntiva con tre letterali per clausola.

Il \hyperref[teorema-7.37]{\textcolor{blue}{Teorema 7.37}} produce una formula che è già quasi in forma congiuntiva normale.
La formula $\phi_{cell}$ è un grosso AND di sottoformule, ciascuna delle quali contiene un grosso OR ed un grosso AND di diversi OR. 
Quindi, $\phi_{cell}$ è un AND di clausole e, di conseguenza. è già in forma cnf. 
La formula $\phi{start}$ è un grosso AND di variabili.
Prendendo ciascuna di queste variabili come clausole di dimensione 1, notiamo che $\phi_{start}$ è in forma cnf.
La formula $\phi_{accept}$ è un grosso OR di variabili, ed è perciò una singola clausola.
La formula $\phi_{move}$ è l'unica che non è già in forma cnf, ma possiamo facilmente convertirla in una formula che è in forma cnf come segue.

Si ricordi che $\phi_{move}$ è un grosso AND di sottoformule, ciascuna delle quali è un OR di diversi AND che descrive tutte le possibili finestre lecite.
La legge distributiva, stabilisce che possiamo sostituire un OR di vari AND con un equivalente AND di vari OR.
Fare ciò può incrementare significativamente la dimensione di ciascuna sottoformula, ma tale operazione può incrementare la dimensione complessiva di
$\phi_{move}$ solamente di un fattore costante, poiché la dimensione di ciascuna sottoformula dipende solo da $N$.
Il risultato è una formula che è in forma normale congiuntiva.

Ora che abbiamo scritto la formula in forma cof, la convertiamo in una con tre letterali per clausola. 
In ciascuna clausola che correntemente ha uno o due letterali, duplichiamo uno dei letterali, fino a quando il numero totale diventa tre. 
Ciascuna clausola che ha più di tre letterali la dividiamo in più claisole e aggiungiamo ulteriori variabili per preservare la soddisfacibilità o non soddisfacibilità della clausola originale.
Per esempio, sostituiamo la clausola ($a_1 \vee a_2 \vee a_3 \vee a_4$), dove ciascun $a_i$
è un eletterale, con l'espressione di due clausolea ($a_1 \vee a_2 \vee z$) e ($\overline{z} \vee a_3 \vee a_4$), dove $z$ è una nuova variabile.
Se qualche impostazione degli $a_i$ soddisfa la clausola originale, possiamo trovare una impostazione di $z$ in modo tale che le due nuove clausole siano soddisfatte e viceversa. 
In generale, se la clausola contiene $l$ letterali,
$$
(a_1 \vee a_2 \vee ... \vee a_l)
$$
possiamo sostituirla con le $l-2$ clausolea
$$
(a_1 \vee a_2 \vee z_1) \land (\overline{z_1} \vee a_3 \vee z_2) \land (\overline{z_2} \vee a_4 \vee z_3) \land ... \land (\overline{z_{l-3}} \vee a_{l-1} \vee a_l).
$$
È possibile verificare facilmente che la nuova formula è soddisfacibile se e solo se la formula originale lo era, quindi la dimostrazione è completa.

\newpage
\subsection{Ulteriori problemi NP-completi}
Per mostrare che un problema $C$ è \textit{NPC}, dobbiamo:
\begin{enumerate}
    \item Provare che appartiene a \textit{NP}; e
    \item Trovare un $B \in \textit{NPC}$ tale che $B \leq_P C$.
\end{enumerate}

Noi applicheremo questo procedimento per dimostrare che tre problemi sono \textit{NPC}, partendo dal fatto che $\textit{SAT} \in \textit{NPC}$.

In particolare:
\[
\textit{SAT} \leq_P \textit{3-CNF-SAT} \leq_P \textit{CLIQUE} \leq_P \textit{VERTEX-COVER}
\]

Ad oggi, i problemi \textit{NPC} sono molte migliaia, alcuni anche molto lontani da $\textit{SAT}$ (originariamente mostrato essere completo per $\textit{NP}$).

\begin{itemize}
    \item Un \textbf{letterale} in una formula booleana è un’occorrenza di una variabile o della sua negazione.
    \item Una formula booleana è in \textbf{forma normale congiuntiva (CNF)} se è espressa come un AND di clausole, ognuna delle quali è un OR di uno o più letterali.
    \item \textbf{OSSERVAZIONE:} Un assegnamento soddisfa una formula in CNF se e solo se ogni clausola contiene almeno un letterale il cui valore è 1.
    \item Una formula booleana è in \textbf{3-CNF} se ogni clausola ha esattamente tre letterali distinti.
\end{itemize}

\textbf{Definizione:} 
\[
\textit{3-CNF-SAT} = \{\langle \phi \rangle \,|\, \phi \text{ è una formula in 3-CNF soddisfacibile} \}
\]

\paragraph{Teorema 34.10(24-30)}
\label{teorema-34.10}
\vspace{1em}
\text{}
\newline
$\textit{3-CNF-SAT} \in \textit{NPC}$.

\text{}
\newline
\textbf{DIMOSTRAZIONE:}
\begin{itemize}
    \item $\textit{3-CNF-SAT} \in \textit{NP}$: il certificato è l’assegnamento; la verifica consiste nel sostituire ogni variabile con il valore assegnatole e nel calcolare il valore della formula.
    \item Mostriamo ora che $\textit{SAT} \leq_P \textit{3-CNF-SAT}$, cioè che ogni formula booleana $\phi$ può essere portata in forma 3-$CNF$ senza modificarne la soddisfacibilità.
\end{itemize}

\subsubsection*{Passo 1: Trasformazione in Forma 3-CNF}
\begin{enumerate}
    \item Costruiamo un albero di parsing binario per la formula in input, con letterali come foglie e connettivi come nodi interni.
    \item Se la formula contiene una clausola che è l’OR/AND di più letterali, usiamo l’associatività per parentesizzare l’espressione, in modo che ogni nodo interno dell’albero risultante abbia al più 2 figli.
    \item Introduciamo una (nuova) variabile $y_i$ per l’output di ogni nodo interno.
    \item La nuova formula $\phi^{'}$ è la congiunzione tra la variabile alla radice dell’albero ed una sottoformula per ogni nodo interno (che ne descrive l’operazione).
    \item Nota: Ogni sottoformula ha al più 3 letterali.
\end{enumerate}
\newpage
\textbf{Esempio:}
\begin{figure}[H]
    \centering
    \includegraphics[width=0.4\textwidth]{Immagini/100.png}
    \label{fig:esempio-3-cnf}
\end{figure}

\textbf{Passo 2: Conversione in CNF}
\begin{itemize}
    \item Siccome $\phi^{'}$ è già una congiunzione di sottoformule, basta convertire ognuna in CNF.
    \item Ogni sottoformula ha al più 3 letterali, possiamo costruire la tavola di verità per ognuna di esse (con al più 8 righe ognuna) e applicare la procedura per il calcolo della forma normale congiuntiva.
\end{itemize}

\textbf{Esempio:}
\begin{figure}[H]
    \centering
    \includegraphics[width=0.4\textwidth]{Immagini/101.png}
    \label{fig:esempio-3-cnf}
\end{figure}

\textbf{Passo 3: Uniformazione delle Clausole}
\begin{itemize}
    \item Per ogni clausola $C$ in $\phi^{''}$, trasformiamo $\phi^{''}$ in $\phi^{'''}$ in cui tutte le clausole hanno esattamente 3 letterali:
    \begin{itemize}
        \item Se $C$ ha 3 letterali, includiamo $C$ in $\phi^{'''}$.
        \item Se $C = l_1 \lor l_2$, includiamo $(l_1 \lor l_2 \lor p) \land (l_1 \lor l_2 \lor \neg p)$ in $\phi^{'''}$.
        \item Se $C = l$, includiamo $(l \lor p \lor q) \land (l \lor p \lor \neg q) \land (l \lor \neg p \lor q) \land (l \lor \neg p \lor \neg q)$ in $\phi^{'''}$.
    \end{itemize}
\end{itemize}

\textbf{Conclusione:}
\begin{itemize}
    \item Ogni passo preserva la soddisfacibilità.
    \item Ogni passo può essere fatto in tempo polinomiale:
    \begin{itemize}
        \item Ottenere $\phi^{'}$ da $\phi$ introduce al più 1 variabile e 1 clausola per ogni connettivo in $\phi$
        \item Ottenere $\phi^{''}$ da $\phi^{'}$ può introdurre al più 8 clausole per ogni clausola di $\phi^{'}$
        \item Ottenere $\phi^{'''}$ da $\phi^{''}$ introduce al più 4 clausole per ogni clausola di $\phi^{''}$
    \end{itemize}
    \item Quindi, la dimensione di $\phi^{'''}$ è polinomiale nella dimensione di $\phi$.
\end{itemize}

\subsubsection{CLIQUE}
\textbf{Definizione:} Dato un grafo non diretto $G = (V, E)$, una \textit{clique} è $V' \subseteq V$ tale che ogni coppia di vertici in $V'$ è connessa da un arco di $E$.

\begin{itemize}
    \item La dimensione di una \textit{clique} è il numero di vertici che contiene.
    \item Il problema della \textit{clique} consiste nel trovare una \textit{clique} di dimensione massima.
    \item Questo problema può essere tradotto in un problema decisionale chiedendo se esiste una \textit{clique} di una data dimensione $k$.
\end{itemize}

\textbf{Definizione Formale:}
\[
\textit{CLIQUE} = \{\langle G, k \rangle \,|\, G \text{ è un grafo non diretto che contiene una clique di dimensione } k \}.
\]

\paragraph{Teorema 34.11(24-30)}
\label{teorema-34.11}
\vspace{1em}
\text{}
\newline
$\textit{CLIQUE} \in \textit{NPC}$.

\text{}
\newline
\textbf{DIMOSTRAZIONE:}
\begin{itemize}
    \item $\textit{CLIQUE} \in \textit{NP}$: il certificato è il sottinsieme $V'$; la verifica consiste nel controllare che ogni $v \in V'$ appartenga a $V$, che $V'$ abbia dimensione $k$, e che, per ogni $u, v \in V'$, $\{u, v\}$ appartenga a $E$. Ciò richiede $O(|V|^2)$ operazioni.
\end{itemize}
Mostriamo che $\textit{3-CNF-SAT} \leq_P \textit{CLIQUE}$, cioè che, data una formula 3-$CNF$ $\phi$, si trova (in tempo polinomiale) una coppia ($G,k$) tale che $\phi$ è soddisfacibile se e solo se $G$ ammette una clique di dimensione $k$.

\textbf{Costruzione del Grafo:}
Sia $\phi = C_1 \land ... \land C_k$, dove ogni $C_r = (l_{1}^r \lor l_{2}^r \lor l_{3}^r)$.

Definiamo $G = (V,E)$ come 
\begin{itemize}
    \item $V = \bigcup_{r=1}^{k} \{v_{1}^r, v_{2}^r, v_{3}^r\}$ //un vertice per ogni letterale di ogni clausola
    \item $E = \{\{v_{i}^r, v_{j}^s\} \,|\, r \neq s \text{ e } l_{i}^r \neq \neg l_{j}^s\}$  //un arco tra ogni coppia di letterali che stanno in clausole diverse e non sono una la negazione dell’altro
\end{itemize}

\textbf{Esempio:}
\begin{figure}[H]
    \centering
    \includegraphics[width=0.4\textwidth]{Immagini/102.png}
    \label{fig:clique-example}
\end{figure}

\textbf{Dimostrazione della corrispondenza tra $\phi$ e $G$:}
\begin{enumerate}
    \item \textbf{Assumiamo che $\phi$ abbia un assegnamento che la soddisfa:}
    \begin{itemize}
        \item Ogni clausola $C_r$ contiene almeno un letterale $l_i^r$ che vale 1.
        \item Consideriamo l’insieme formato scegliendo esattamente un tale letterale da ogni clausola.
        \item Questo forma un sottinsieme $V'$ di $V$ di dimensione $k$. Mostriamo che $V'$ è una clique:
        \begin{itemize}
            \item Per ogni coppia di vertici $v_i^r, v_j^s \in V'$, per costruzione di $V'$ abbiamo che $r \neq s$ e che $l_i^r$ ed $l_j^s$ valgono entrambi 1 (quindi, i letterali non sono uno la negazione dell’altro).
            \item Per costruzione di $G$, l’arco $\{v_i^r, v_j^s\}$ appartiene a $E$.
        \end{itemize}
    \end{itemize}
    
    \item \textbf{Assumiamo che $G$ abbia una clique $V'$ di dimensione $k$:}
    \begin{itemize}
        \item Nessun arco di $G$ connette vertici della stessa tripla, quindi $V'$ contiene esattamente un vertice per ogni tripla.
        \item Assegniamo 1 ad ogni letterale $l_i^r$ tale che $v_i^r \in V'$ (variabili che non corrispondono a vertici della clique possono assumere qualsiasi valore).
        \begin{itemize}
            \item Non assegneremo mai 1 ad un letterale e al suo negato, siccome $G$ non ha archi tra letterali complementari.
        \end{itemize}
        \item Ogni clausola è soddisfatta, quindi $\phi$ è soddisfacibile.
    \end{itemize}
\end{enumerate}
\textbf{Conclusione:} $\textit{3-CNF-SAT} \leq_P \textit{CLIQUE}$. Q.E.D.

\textbf{Esempio:}
\begin{figure}[H]
    \centering
    \includegraphics[width=0.4\textwidth]{Immagini/103.png}
    \caption{Esempio di clique}
    \label{fig:clique-example-2}
\end{figure}

Ogni clique di dimensione 3 corrisponde ad un assegnamento che soddisfa la formula, e viceversa. Per esempio:
\begin{itemize}
    \item La clique identificata dagli archi in rosso $\Rightarrow x_1 = x_2 = x_3 = 1$.
    \item La clique identificata dagli archi in verde $\Rightarrow x_1 = x_2 = 1$ ($x_3$ può avere qualsiasi valore).
    \item La clique identificata dagli archi in arancio $\Rightarrow x_1 = x_2 = 0$ e $x_3 = 1$.
    \item La clique identificata dagli archi in blu $\Rightarrow x_2 = 0$ e $x_3 = 1$ ($x_1$ può avere qualsiasi valore).
\end{itemize}

\subsubsection{VERTEX-COVER}
\textbf{Definizione:} Dato un grafo non diretto $G = (V, E)$, un \textit{vertex cover} è un $V' \subseteq V$ tale che, per ogni $\{u,v\} \in E$, o $u \in V'$ o $v \in V'$ (o entrambe).

\begin{itemize}
    \item Ogni vertice copre gli archi che lo hanno come estremo; un \textit{vertex cover} è un insieme che copre tutti gli archi.
\end{itemize}

\begin{figure}[H]
    \centering
    \includegraphics[width=0.4\textwidth]{Immagini/104.png}
    \caption{Esempio di vertex cover}
    \label{fig:vertex-cover-example}
\end{figure}
\begin{itemize}
    \item L’insieme $\{w, z\}$ è un \textit{vertex-cover}.
\end{itemize}
La dimensione di un \textit{vertex-cover} è il numero di vertici che contiene. 

\textbf{Problema decisionale:}
\begin{itemize}
    \item Il problema del \textit{vertex-cover} è di trovare un \textit{vertex cover} di dimensione minima.
    \item Può essere tradotto in un problema decisionale, chiedendo se un grafo ha un \textit{vertex cover} di una certa dimensione $k$.
\end{itemize}

\textbf{Definizione Formale:}
\[
\textit{VERTEX-COVER} = \{\langle G, k \rangle \,|\, G \text{ è un grafo non diretto con un vertex cover di dimensione } k \}.
\]

\paragraph{Teorema 34.12(orale)}
\label{teorema-34.12}
\vspace{1em}
\text{}
\newline
$\textit{VERTEX-COVER} \in \textit{NPC}$.

\text{}
\newline
\textbf{DIMOSTRAZIONE:}
\begin{enumerate}
    \item $\textit{VERTEX-COVER} \in \textit{NP}$: il certificato è il sottinsieme $V'$. La verifica consiste nel controllare che ogni $v \in V'$ appartiene a $V$, che $V'$ ha dimensione $k$, e che, per ogni $\{u,v\} \in E$, o $u$ o $v$ appartiene a $V'$. Ciò richiede $O(|V| + |E|)$ operazioni.
    \item Mostriamo che $\textit{CLIQUE} \leq_P \textit{VERTEX-COVER}$:
    \begin{itemize}
        \item Dato un grafo $G = (V, E)$ e un intero $k$, definiamo una nuova coppia $(G', k')$ tale che $G$ ha una clique di dimensione $k$ se e solo se $G'$ ha un \textit{vertex cover} di dimensione $k'$.
        \item $G'$ è il complemento di $G$, con $G' = (V, \bar{E})$ dove $\bar{E} = \{\{u,v\} \,|\, u, v \in V, u \neq v, \{u,v\} \notin E\}$.
    \end{itemize}
\end{enumerate}

Dato ($G, k$) per $CLIQUE$, la coppia da considerare per $VERTEX-COVER$ è ($\overline{G}, |V| - k$).

\begin{enumerate}
    \item \textbf{Assumuamo che $G$ abbia una clique $V'$ di dimensione $k$:}
    \begin{itemize}
        \item Mostriamo che $V \setminus V'$ è un \textit{vertex cover} per $\overline{G}$.
        \item Sia $\{u, v\}$ un qualsiasi arco in $\bar{E}$. Allora:
        \begin{itemize}
            \item Per definizione, $\{u, v\} \notin E$.
            \item Essendo $V'$ una clique, almeno uno tra $u$ e $v$ non appartiene a $V'$, poiché ogni coppia di vertici di $V'$ è connessa da un arco di $E$.
            \item Quindi, almeno uno tra $u$ e $v$ appartiene a $V \setminus V'$, come richiesto.
        \end{itemize}
        \item Infine, $|V \setminus V'| = |V| - k$, essendo $|V'| = k$.
    \end{itemize}

    \item \textbf{assumiamo che $\overline{G}$ ha un \textit{vertex cover} $V'$ di dimensione $|V| - k$:}
    \begin{itemize}
        \item Mostriamo che $V \setminus V'$ è una clique per $G$.
        \item Sia $u, v$ una qualsiasi coppia di vertici distinti in $V \setminus V'$. Allora:
        \begin{itemize}
            \item $\{u, v\}$ non può appartenere a $\bar{E}$, altrimenti non sarebbe coperto da $V'$ (che invece è un \textit{vertex cover}).
            \item Quindi, per definizione, $\{u, v\} \in E$, come richiesto.
        \end{itemize}
        \item Infine, $|V \setminus V'| = k$, essendo $|V'| = |V| - k$.
    \end{itemize}
\end{enumerate}

\textbf{Conclusione:} $\textit{CLIQUE} \leq_P \textit{VERTEX-COVER}$. Q.E.D.


\textbf{Esempio:}
\begin{itemize}
    \item Consideriamo il grafo $G$.
    \item Il suo complemento $G'$ è rappresentato nella figura seguente
\end{itemize}

\begin{figure}[H]
    \centering
    \includegraphics[width=0.4\textwidth]{Immagini/105.png}
    \caption{Esempio di vertex cover}
    \label{fig:vertex-cover-example-2}
\end{figure}

\subsection{Approssimazione}
Molti problemi pratici sono \textit{NPC}, ma troppo rilevanti per essere abbandonati solo perché non sappiamo come trovare una soluzione ottimale in tempo polinomiale.

\textbf{Tre approcci per aggirare l’NP-completezza:}
\begin{enumerate}
    \item Utilizzare un algoritmo di tempo esponenziale per input sufficientemente piccoli.
    \item Isolare sottocasi importanti risolvibili in tempo polinomiale.
    \item Trovare soluzioni vicine all’ottimo in tempo polinomiale (sia nel caso pessimo che medio).
\end{enumerate}

Un algoritmo che restituisce soluzioni prossime all’ottimo sarà chiamato \textbf{algoritmo di approssimazione}.

\subsection*{Approximation Ratio}
Immaginiamo di avere un problema di ottimizzazione in cui ogni potenziale soluzione ha un costo positivo.

\begin{tcolorbox}[colback=yellow!10!white, colframe=yellow!50!black, title=Definizione]
    Un algoritmo ha un \textit{approximation ratio} di $\rho(n)$ se, per ogni input di dimensione $n$, il costo $C$ della soluzione prodotta differisce dal costo $C^*$ della soluzione ottimale per al più un fattore $\rho(n)$:
    \[
    \text{Per un problema di massimizzazione: } 0 < C \leq C^*, \quad \rho(n) = \frac{C^*}{C}.
    \]
    \[
    \text{Per un problema di minimizzazione: } 0 < C^* \leq C, \quad \rho(n) = \frac{C}{C^*}.
    \]

    Se un algoritmo raggiunge un \textit{approximation ratio} di $\rho(n)$, lo chiameremo un \textbf{\textit{$\rho(n)$-algoritmo di approssimazione}}.
\end{tcolorbox}

\textbf{Nota Bene:}
\begin{itemize}
    \item L’approximation ratio non è mai minore di 1.
    \item Un 1-algoritmo di approssimazione produce una soluzione ottimale.
    \item Un approximation ratio maggiore restituisce una soluzione peggiore dell’ottimale.
\end{itemize}

Per molti problemi, abbiamo algoritmi di approssimazione polinomiali con
approximation ratios che sono costanti (anche piccole).

Per altri problemi, i migliori algoritmi di approssimazione polinomiali hanno
approximation ratio che crescono in funzione della dimensione dell’input n.

Alcuni problemi NPC ammettono algoritmi di approssimazione polinomiali che
possono raggiungere approssimazioni sempre migliori usando più tempo
computazionale.

$\Rightarrow$ si può negoziare il tempo computazionale per la qualità dell’approssimazione.

\subsubsection{Vertex Cover}
\textbf{Definizione:} Un \textit{vertex cover} in un grafo non diretto $G = (V, E)$ è un sottinsieme $V' \subseteq V$ tale che, per ogni $\{u, v\} \in E$, o $u \in V'$ o $v \in V'$ (o entrambe).

\begin{itemize}
    \item La dimensione di un vertex cover è $|V'|$.
    \item Il problema consiste nel trovare un vertex cover di dimensione minima, detto \textbf{\textit{optimal vertex cover}}.
\end{itemize}

\begin{tcolorbox}[colback=red!10!white, colframe=red!50!black, title=Osservazione]
    Questo problema è la versione ottimizzata di \textit{VERTEX-COVER} ($\in$\textit{NPC}), quindi non si può risolvere in tempo polinomiale a meno che $P = NP$.
\end{tcolorbox}

\textbf{IDEA:} Ad ogni passo, selezioniamo il nodo che «copre» più archi.

\begin{figure}[H]
    \centering
    \includegraphics[width=0.4\textwidth]{Immagini/106.png}
    \label{algo_vertex_cover}
\end{figure}

\textbf{Esempio:}

\begin{figure}[H]
    \centering
    \includegraphics[width=0.4\textwidth]{Immagini/107.png}
    \label{algo_vertex_cover-example}
\end{figure}

\hbox{In questo esempio, il rapporto tra la soluzione calcolata dall'algoritmo e l'ottimo è 4/3.}

$\Rightarrow$ ma questo deve valere \textit{per ogni} grafo di input!

\text{}
\newline
\hbox{\textbf{CONTROESEMPIO:} consideriamo il seguente grafo:}

\begin{figure}[H]
    \centering
    \includegraphics[width=0.4\textwidth]{Immagini/108.png}
    \label{fig:controesempio}
\end{figure}

Per applicare \textit{GREEDY-VERTEX-COVER} a questo grafo:
\begin{enumerate}
    \item Selezioniamo i nodi con il grado maggiore, in ordine decrescente.
    \item Ripetiamo fino a coprire tutti gli archi.
\end{enumerate}

\begin{figure}[H]
    \centering
    \includegraphics[width=0.4\textwidth]{Immagini/109.png}
    \label{fig:example_image}
\end{figure}

Quindi, l'algoritmo \textit{GREEDY-VERTEX-COVER} restituisce tutti i nodi colorati , che sono
$$
\sum_{i=2}^{k} \lfloor \frac{k}{i} \rfloor \geq k(\ln k - 2).
$$

\text{}
\newline
In realtà, il vertex cover ottimale sarebbe quello formato solo dai nodi neri (quindi, di dimensione $k$).
\newline
Quindi, in questo caso l'approximation ratio è almeno $O(\log k)$ (dove $k$ è una funzione di $n$).
\vspace{1em}

\hbox{\textbf{N.B.:} si può provare che questo upper bound è preciso, quindi questo è il caso pessimo.}
\vspace{1em}

$\Rightarrow$ si può fare meglio con un \textit{approximation ratio} costante.

\textbf{Esempio di un algoritmo migliorato:}

\begin{figure}[H]
    \centering
    \includegraphics[width=0.5\textwidth]{Immagini/110.png}
    \label{fig:algo-migliorato}
\end{figure}

\paragraph{Teorema 35.1(orale)}
\label{teorema-35.1}
\vspace{1em}
\text{}
\newline
\textit{APPROX-VERTEX-COVER} è un 2-algoritmo di approssimazione polinomiale.

\text{}
\newline
\textbf{DIMOSTRAZIONE:}
\begin{itemize}
    \item \textit{APPROX-VERTEX-COVER} richiede tempo $O(|V| + |E|)$:
    \begin{itemize}
        \item Ogni nodo è aggiunto a $C$ al massimo una volta.
        \item Ogni arco è rimosso da $E'$ al massimo una volta (o perchè selezionato o perchè cancellato).
    \end{itemize}
    \item L’insieme $C$ restituito è un vertex cover, siccome l'algoritmo cicla finchè ogni arco di $E'$ è stato coperto da un vertice in $C$.
    \item \textit{APPROX-VERTEX-COVER} restituisce un insieme la cui dimensione è al più doppia della dimensione di un vertex cover ottimo:
    \begin{itemize}
        \item Sia $A$ l’insieme degli archi scelti nella linea4 (dello pseudo-codice).
        \item Poiché non esistono coppie in $A$ con estremi condivisi, $|C| = 2|A|$.
        \item Ogni vertex cover ottimale $C^*$ deve includere almeno un estremo per ogni arco di $A$: $|A| \leq |C^*|$.
        \item Quindi, $|C| = 2|A| \leq 2|C^*|$.
    \end{itemize}
\end{itemize}

\section{Complessità di Spazio}
In questo capitolo valutiamo la complessità dei problemi computazionali in termini della quantità di spazio, o memoria, che essi richiedono.
Tempo e spazio sono due dei fattori più importanti quando ricerchiamo soluzioni pratiche per numerosi problemi computazionali.
La complessità di spazio condivide molte caratteristiche della complessità di tempo e rappresenta un modo ulteriore per classificare i problemi in accordo alla propria difficoltà computazionale.

Come per la complessità di tempo, abbiamo necessità di selezionare un modello per misurare lo spazio utilizzato da un algoritmo.
Continuiamo a servirci del modello della macchina di Turing per la stessa ragione per cui l'abbiamo usato per misurare il tempo.
Le macchine di Turing sono matematicamente semplici ed abbastanza vicine ai calcolatori reali da fornire risultati significativi.

\begin{tcolorbox}[colback=yellow!5!white, colframe=yellow!50!black, title=\textbf{Definizione 8.1}]
    Sia $M$ una macchina di Turing deterministica che si arresta
    su tutti gli input. 
    La \textit{\textbf{complessità di spazio}} di $M$ è la funzione $f: \mathbb{N} \rightarrow \mathbb{N}$, dove  $f(n)$ è il numero massimo di celle del nastro che $M$ scandisce su ogni input di lunghezza $n$.
    Se la complessità di spazio di $M$ è $f(n)$, diciamo anche che $M$ computa in spazio $f(n)$.
    Se $M$ è una macchina di Turing non deterministica dove tutte le diramazioni si arrestano su tutti gli input, definiamo la sua complessità di spazio $f(n)$ come il massimo numero di celle del nastro che $M$ scandisce su qualsiasi diramazione della sua computazione per ogni input di lunghezza $n$.
\end{tcolorbox}

Tipicamente stimiamo la complessità di spazio delle macchine di Turing attraverso la notazione asintotica.

\paragraph*{Definizione 8.2}
\label{definizione-8.2}
\vspace{1em}
\text{}
\newline
Sia $f : \mathbb{N} \rightarrow \mathbb{R}^+$ una funzione.
\[
\text{SPACE}(f(n)) = \{L \mid L \text{ è un linguaggio deciso da una macchina di Turing deterministica con spazio } O(f(n))\}.
\]
\[
\text{NSPACE}(f(n)) = \{L \mid L \text{ è un linguaggio deciso da una macchina di Turing non deterministica con spazio } O(f(n))\}.
\]

\begin{figure}[H]
    \centering
    \includegraphics[width=0.3\textwidth]{Immagini/111.png}
    \includegraphics[width=0.3\textwidth]{Immagini/112.png}
    \label{fig:space-complexity}
\end{figure}

\subsection{Teorema di Savitch (e mo so cazzi pt.2)}
Il teorema di Savitch è uno dei primi risultati riguardanti la complessità di spazio. 
Mostra che le macchine deterministiche possono simulare le macchine non deterministiche usando una quantità di spazio sorprendentemente molto piccola.
Per la complessità di tempo, tale simulazione sembra richiedere un incremento esponenziale in tempo.
Per la complessità di spazio, il teorema di Savitch mostra che ogni TM non deterministica che usa spazio $f(n)$ può essere convertita in una TM deterministica che usa soltanto spazio $f^2(n)$.

\paragraph{Teorema 8.5(30L)}
\label{teorema-8.5}
\vspace{1em}
\text{}
\newline
\begin{tcolorbox}[colback=white, colframe=black!50!red, title=\textbf{Teorema di Savitch}]
    Per ogni funzione $f: \mathbb{N} \rightarrow \mathbb{R}^+$ dove $f(n)$, dove $f(n) \geq n$,
    \[
    \text{NSPACE}(f(n)) \subseteq \text{SPACE}(f^2(n)).
    \]
\end{tcolorbox}

\textbf{IDEA:}
Dobbiamo simulare deterministicamente una NTM con complessità di spazio $f(n)$.
Un approccio ingenuo consiste nel procedere tentando tutte le diramazioni seguite dalla computazione della NTM, una per una. 
La simulazione deve tener traccia di quale diramazione la macchina sta tentando al momento per poter passare alla successiva.
Ma una diramazione che usa spazio $f(n)$ può richiedere $2^{O(f(n))}$ passi e ciascun passo potrebbe consistere in una scelta non deterministica.
L'esplorazione sequenziale delle diramazioni richiederebbe la memorizzazione di tutte le scelte effettuate su una particolare diramazione per poter individuare la diramazione successiva.
Pertanto, questo approccio può usare spazio $2^{O(f(n))}$, eccedendo il
nostro obiettivo di usare spazio $O(f^2(n))$.

Invece, usiamo un approccio diverso, considerando il problema più generale che segue. 
Date due configurazioni della NTM, $c_1$ e $c_2$, e un numero $t$, verifichiamo se la NTM può transire da $c_1$ a $c_2$ in al più $t$ passi, usando soltanto spazio $f(n)$.
Chiamiamo questo problema il \textit{\textbf{problema della resa}}.
Risolvendo il problema della resa, dove $c_1$, è la configurazione iniziale, $c_2$ è la configurazione finale, e $t$ è il numero massimo di passi che la macchina non deterministica può effettuare, possiamo stabilire se la macchina accetta il proprio input.

A tal fine forniamo un algoritmo ricorsivo deterministico che risolve il problema della resa.
L'algoritmo opera cercando una configurazione intermedia $c_m$: e verificando ricorsivamente se 
(1) $c_1$ può portare a $c_m$ in al
più $t/2$ passi, e 
(2) se $c_m$ può portare a $c_2$ in al più $t/2$ passi.
La riutilizzazione dello spazio per ognuno dei due test ricorsivi permette un risparmio di spazio significativo.

Questo algoritmo richiede spazio per memorizzare le chiamate ricorsive. 
Ciascun livello della ricorsione usa spazio $O(f(n))$ per memorizzare una configurazione.
La profondità della ricorsione è $\log t$, dove $t$ è il tempo massimo che la macchina non deterministica può usare su ogni diramazione. 
Risulta $t = 2^{O(f(n))}$, quindi $\log t = O(f(n))$.
Pertanto la simulazione deterministica usa spazio $O(f^2(n))$.

\text{}
\newline
\textbf{DIMOSTRAZIONE:}
Sia $N$ una NTM che decide il linguaggio $A$ in spazio $f(n)$.
Costruiamo una TM deterministica $M$ che decide $A$.
La macchina $M$ utilizza la procedura $CANYIELD$, che verifica se una delle configurazioni di $N$ può darne un'altra entro uno specificato numero di passi.
Questa procedura risolve il problema della resa descritto nell'idea della prova.

Sia $w$ una stringa di input per $N$.
Per le configurazioni $c_1$ e $c_2$ di $N$, e l'intero $t$, $CANYIELD(c_1, c_2, t)$ da in output \textit{accetta} se $N$ può passare dalla configurazione $c_1$ alla configurazione $c_2$ in $t$ o meno passi attraverso un qualche cammino non deterministico.
Altrimenti, $CANYIELD$ da in output \textit{rifiuta}.
Per convenienza, assimiamo che $t$ sia una potenza di 2.

\text{}
\newline
$CANYIELD = $ "Su input $c_1, c_2, t$:
\begin{enumerate}
    \item Se $t = 1$, allora verifica direttamente se $c_1 = c_2$ o se $c_1$ può passare a $c_2$ in un passo, in accordo alle regole di transizione di $N$. \textit{Accetta} se uno dei due test ha successo; altrimenti, \textit{rifiuta}.
    \item Se $t > 1$, allora per ogni configurazione $c_m$ di $N$ usando spazio $f(n)$:
    \item $\quad$ Esegui $CANYIELD(c_1, c_m, t/2)$.
    \item $\quad$ Esegui $CANYIELD(c_m, c_2, t/2)$.
    \item $\quad$ Se i passi 3 e 4 somo entrambi accettanti, allora \textit{accetta}.
    \item Se non ha ancora accettato, \textit{rifiuta}."
\end{enumerate}

Ora definiamo $M$ per simulare $N$ come segue.
Prima modifichiamo $N$ in modo tale che, quando accetta, cancella il proprio nastro e muove la testina sulla cella più a sinistra - entrando con ciò in una configurazione detta $c_{accept}$. 
Denotiamo con $c_{start}$ la configurazione iniziale di $N$ su $w$.
Selezioniamo una costante $d$ in modo tale che $N$ abbia al più $2^{df(n)}$ configurazioni usando un nastro di $f(n)$ celle, dove $n$ è la lunghezza di $w$.
Allora sappiamo che $2^{df(n)}$ fornisce un limite superiore al tempo di esecuzione di ogni diramazione di $N$ su $w$.
    
\text{}
\newline
$M = $ "Su input $w$:
\begin{enumerate}
    \item Fornisci in output il risultato di $CANYIELD(c_{start}, c_{accept}, 2^{df(n)})$."
\end{enumerate}

L'algoritmo $CANYIELD$ ovviamente risolve il problema della resa e, di conseguenza, $M$ simula correttamente $N$.
Dobbiamo analizzarne l'esecuzione per verificare che $M$ lavora in spazio $O(f^2(n))$.

Ogni qualvolta $CANYIELD$ invoca se stessa ricorsivamente, memorizza il numero della fase corrente e i valori di $c_1, c_2$, e $t$ in una pila, in modo tale che questi valori possano essere recuperati dalla chiamata ricorsiva.
Ciascun livello della ricorsione usa perciò uno spazio aggiuntivo pari a $O(f(n))$. 
Inoltre, ciascun livello della ricorsione divide la taglia di $t$ a metà.
Inizialmente $t$ è uguale a $2^{df(n)}$, e quindi la profondità della ricorsione è $O (\log 2^{df(n)})$ ovvero $O(f(n))$. 
Pertanto, lo spazio totale usato è $O(f^2(n))$, come asserito.
Una difficoltà tecnica sorge in questo argomento perché l'algoritmo $M$ deve conoscere il valore di $f(n)$ quando invoca $CANYIELD$.
Possiamo gestire questa difficoltà modificando $M$ in modo tale che tenti con $f(n) = 1, 2, 3, \ldots$ .
Per ciascun valore $f(n) = i$, l'algoritmo modificato usa $CANYIELD$ per stabilire se la configurazione di accettazione è raggiungibile.
In aggiunta, l'algoritmo usa $CANYIELD$ per stabilire se $N$ usa almeno spazio $i + 1$ verificando se $N$ può raggiungere qualcuna delle configurazioni di lunghezza $i + 1$ dalla configurazione iniziale.

Se la configurazione finale è raggiungibile, $M$ accetta; 
se nessuna configurazione di lunghezza $i + 1$ è raggiungibile, $M$ rifiuta; 
e altrimenti, $M$ continua con $f(n) = i + 1$. 
(Avremmo potuto gestire questa difficoltà in
un altro modo, assumendo che $M$ possa calcolare $f(n)$ in spazio $O(f(n))$,
ma poi avremmo dovuto aggiungere questa assunzione all'enunciato del teorema.)

\subsection{LA CLASSE PSPACE}
In analogia con la classe P, definiamo la classe PSPACE per la complessità di spazio.

\paragraph{Definizione 8.6}
\label{definizione-8.6}
\vspace{1em}
\text{}
\newline
\begin{tcolorbox}[colback=white, colframe=black!50!red, title=\textbf{Definizione}]
    \textbf{\textit{PSPACE}} è la classe dei linguaggi che sono decidibili in spazio polinomiale con una macchina di Turing deterministica.
    \newline
    In altre parole,
    \[
    \textit{PSPACE} = \bigcup_{k} \textit{SPACE}(n^k).
    \]
\end{tcolorbox}

Definiamo NPSPACE, la controparte non deterministica di PSPACE, in termini delle classi NSPACE.
Tuttavia, in virtù del teorema di Savitch, PSPACE = NPSPACE, perché il quadrato di un polinomio è ancora un polinomio.

Negli esempi 8.3 e 8.4, abbiamo mostrato che $SAT$ è contenuto in SPACE$(n)$ e che $ALL_NFA$ è contenuto in coNSPACE($n$) e, quindi, per il teorema di Savitch, in SPACE($n^2$), perché le classi di complessità di spazio deterministiche sono chiuse rispetto al complemento.
Pertanto, entrambi i linguaggi sono in PSPACE.

Esaminiamo ora la relazione di PSPACE con P ed NP.
Osserviamo che P $\subseteq$ PSPACE, perchè una macchina che esegue velocemente non può usare una quantità di spazio grande.
Precisamente, per $t(n) \geq n$, qualsiasi macchina che opera in tempo $t(n)$ può usare al più spazio $t(n)$, perchè una macchina può esplorare al più una nuova cella ad ogni passo della propria esecuzione.
Similmente, NP $\subseteq$ NPSPACE e, pertanto, NP $\subseteq$ PSPACE.

Viceversa, possiamo limitare la complessità di tempo di una macchina di Turing in termini della dua complessità di spazio.
Per $f(n) \geq n$, una TM che usa spazio $f(n)$ può avere al più $2^{O(f(n))}$ configurazioni differenti, attraverso una semplice generalizzazione della priva del \hyperref[lemma-5.8]{\textcolor{blue}{Lemma 5.8}}.
Una computazione di una TM che si arresta non può ripetere una configurazione.
Pertanto, una TM che usa spazio $f(n)$ deve computare in tempo $f(n)2^{O(f(n))}$, quindi PSPACE $\subseteq$ EXPTIME = $\bigcup_{k} \text{TIME}(2^{n^k})$.

Riassumiamo le nostre conoscenze circa le relazioni tra le classi di complessità definite fino ad ora nella serie di inclusioni
\[
\text{P} \subseteq \text{NP} \subseteq \text{PSPACE} = \text{NPSPACE} \subseteq \text{EXPTIME}.
\]
Non sappiamo se una qualsiasi di queste inclusioni è in realtà un'uguaglianza.
Qualcuno potrebbe trovare una simulazione simile a quella usata nel teorema di Savitch che fonde alcune di queste classi nella stessa classe.
In realtà, molti ricercatori ritengono che tutte le inclusioni siano proprie.
Il diagramma seguente raffigura le relazioni tra queste classi, assumendo che siano tutte differenti.

\begin{figure}[H]
    \centering
    \includegraphics[width=0.5\textwidth]{Immagini/113.png}
    \label{fig:relazioni-classi-complessita}
\end{figure}


\end{document}